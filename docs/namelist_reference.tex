%%%%%%%%%%%%%%%%%%%%%%%%%%%%%%%%%%%%%%%%%%%%%%%%%%%%%%%%%%%%%%%%%%%%%%%%%
% RAMSES / cuRAMSES-kjhan  --  Namelist Parameter Reference
%
% Standalone-compilable:   pdflatex namelist_reference.tex
% Includable from main.tex:  %%%%%%%%%%%%%%%%%%%%%%%%%%%%%%%%%%%%%%%%%%%%%%%%%%%%%%%%%%%%%%%%%%%%%%%%%
% RAMSES / cuRAMSES-kjhan  --  Namelist Parameter Reference
%
% Standalone-compilable:   pdflatex namelist_reference.tex
% Includable from main.tex:  %%%%%%%%%%%%%%%%%%%%%%%%%%%%%%%%%%%%%%%%%%%%%%%%%%%%%%%%%%%%%%%%%%%%%%%%%
% RAMSES / cuRAMSES-kjhan  --  Namelist Parameter Reference
%
% Standalone-compilable:   pdflatex namelist_reference.tex
% Includable from main.tex:  %%%%%%%%%%%%%%%%%%%%%%%%%%%%%%%%%%%%%%%%%%%%%%%%%%%%%%%%%%%%%%%%%%%%%%%%%
% RAMSES / cuRAMSES-kjhan  --  Namelist Parameter Reference
%
% Standalone-compilable:   pdflatex namelist_reference.tex
% Includable from main.tex:  \input{namelist_reference}
%   (when included, comment out the \documentclass .. \begin{document}
%    and \end{document} lines, or wrap them with \ifx\STANDALONE\undefined)
%%%%%%%%%%%%%%%%%%%%%%%%%%%%%%%%%%%%%%%%%%%%%%%%%%%%%%%%%%%%%%%%%%%%%%%%%

\documentclass[11pt]{article}

%----------------------------------------------------------------------
%  Packages
%----------------------------------------------------------------------
\usepackage[top=2.5cm,bottom=2.5cm,left=2.8cm,right=2.8cm,a4paper]{geometry}
\usepackage[utf8]{inputenc}
\usepackage[T1]{fontenc}
\usepackage{lmodern}
\usepackage{microtype}
\usepackage{xcolor}
\usepackage{hyperref}
\usepackage{enumitem}
\usepackage{amsmath,amssymb}
\usepackage{booktabs}
\usepackage{longtable}
\usepackage{fancyhdr}
\usepackage{titlesec}
\usepackage{parskip}
\usepackage{listings}

%----------------------------------------------------------------------
%  Colour palette (CAMB-readthedocs inspired)
%----------------------------------------------------------------------
\definecolor{apibg}{RGB}{248,249,251}       % light background
\definecolor{apiborder}{RGB}{200,210,225}    % border
\definecolor{apiname}{RGB}{0,64,128}         % parameter name
\definecolor{apitype}{RGB}{100,100,100}      % type annotation
\definecolor{apidefault}{RGB}{60,130,60}     % default value
\definecolor{apinote}{RGB}{180,80,40}        % warning / note
\definecolor{sectioncolor}{RGB}{0,80,160}    % section headings
\definecolor{codebg}{RGB}{243,245,248}
\definecolor{codeframe}{RGB}{190,200,215}

%----------------------------------------------------------------------
%  Hyperref
%----------------------------------------------------------------------
\hypersetup{
  colorlinks=true,
  linkcolor=sectioncolor,
  urlcolor=sectioncolor!80!black,
  citecolor=sectioncolor,
  pdfauthor={Juhan Kim},
  pdftitle={RAMSES Namelist Parameter Reference},
}

%----------------------------------------------------------------------
%  Section formatting
%----------------------------------------------------------------------
\titleformat{\section}
  {\Large\bfseries\sffamily\color{sectioncolor}}
  {\thesection}{1em}{}
  [\vspace{-0.6em}\textcolor{sectioncolor!40}{\rule{\linewidth}{0.8pt}}]

\titleformat{\subsection}
  {\large\bfseries\sffamily\color{sectioncolor!80!black}}
  {\thesubsection}{0.8em}{}

%----------------------------------------------------------------------
%  Header / footer
%----------------------------------------------------------------------
\pagestyle{fancy}
\fancyhf{}
\fancyhead[L]{\small\sffamily\textcolor{apitype}{RAMSES Namelist Reference}}
\fancyhead[R]{\small\sffamily\textcolor{apitype}{\nouppercase{\leftmark}}}
\fancyfoot[C]{\small\sffamily\thepage}
\renewcommand{\headrulewidth}{0.4pt}

%----------------------------------------------------------------------
%  Code listings
%----------------------------------------------------------------------
\lstset{
  basicstyle=\ttfamily\small,
  backgroundcolor=\color{codebg},
  frame=single,
  rulecolor=\color{codeframe},
  framesep=4pt,
  xleftmargin=6pt,
  xrightmargin=6pt,
  breaklines=true,
  columns=fullflexible,
  keepspaces=true,
  showstringspaces=false,
}

%----------------------------------------------------------------------
%  Custom environments for parameter documentation.
%
%  We use a "paramblock" environment instead of a macro so that
%  verbatim-like content (lstlisting) can appear inside parameter
%  descriptions.
%----------------------------------------------------------------------

% --- parameter header bar (name + type + default) ---
\newcommand{\paramheader}[3]{%
  \vspace{0.7em}%
  \noindent
  \colorbox{apibg}{%
    \parbox{\dimexpr\linewidth-2\fboxsep}{%
      \vspace{4pt}%
      \hspace{2pt}%
      {\large\textcolor{apiname}{\texttt{\textbf{#1}}}}%
      \hfill
      {\small\textcolor{apitype}{\textsf{#2}}}%
      \quad
      {\small\textcolor{apidefault}{\textsf{default:~\texttt{#3}}}}%
      \vspace{4pt}%
    }%
  }%
  \nopagebreak[4]\par\nopagebreak[4]\vspace{0.25em}%
}

% --- required parameter header bar (no default) ---
\newcommand{\paramheaderreq}[2]{%
  \vspace{0.7em}%
  \noindent
  \colorbox{apibg}{%
    \parbox{\dimexpr\linewidth-2\fboxsep}{%
      \vspace{4pt}%
      \hspace{2pt}%
      {\large\textcolor{apiname}{\texttt{\textbf{#1}}}}%
      \hfill
      {\small\textcolor{apitype}{\textsf{#2}}}%
      \quad
      {\small\textcolor{apinote}{\textsf{required}}}%
      \vspace{4pt}%
    }%
  }%
  \nopagebreak[4]\par\nopagebreak[4]\vspace{0.25em}%
}

%----------------------------------------------------------------------
%  Note / Warning / Example helpers (usable at top level)
%----------------------------------------------------------------------
\newcommand{\apinote}[1]{%
  \par\vspace{0.3em}%
  \noindent\textcolor{sectioncolor}{\textsf{\textbf{Note:}}} #1%
  \par\vspace{0.2em}%
}

\newcommand{\apiwarning}[1]{%
  \par\vspace{0.3em}%
  \noindent\textcolor{apinote}{\textsf{\textbf{Warning:}}} #1%
  \par\vspace{0.2em}%
}

%----------------------------------------------------------------------
%  Cross-reference helper
%  Usage: \paramref{cosmo}  or  \paramref{memory_balance}
%  The label name uses the raw form (with literal underscores);
%  the displayed text uses \texttt with escaped underscores.
%----------------------------------------------------------------------
\makeatletter
\newcommand{\paramref}[1]{%
  \begingroup
  \def\tmp{#1}%
  \hyperref[param:\tmp]{\texttt{\detokenize{#1}}}%
  \endgroup
}
\makeatother

%======================================================================
\begin{document}

%----------------------------------------------------------------------
%  Title
%----------------------------------------------------------------------
\begin{center}
{\fontsize{28}{34}\selectfont\sffamily\bfseries
  \textcolor{sectioncolor}{RAMSES Namelist Parameter Reference}}\\[0.8cm]
{\large\sffamily cuRAMSES-kjhan -- February 2026}\\[0.3cm]
{\normalsize\sffamily Juhan Kim}\\[0.2cm]
{\small\sffamily\textcolor{apitype}{%
  Based on RAMSES by Romain Teyssier}}
\end{center}

\vspace{0.5cm}
\noindent\textcolor{sectioncolor!40}{\rule{\linewidth}{1pt}}

\vspace{0.6cm}
\noindent
This document provides a complete reference for every namelist parameter
accepted by RAMSES and the cuRAMSES-kjhan extensions.  Parameters are
grouped by their Fortran namelist block (\texttt{\&RUN\_PARAMS},
\texttt{\&AMR\_PARAMS}, etc.).  Each entry specifies the parameter name,
Fortran type, default value, and a detailed description including valid
ranges and interactions with other parameters.

The namelist file uses standard Fortran namelist syntax.  Each block
begins with \texttt{\&BLOCK\_NAME} and ends with a single~\texttt{/}.

\vspace{0.4cm}
\tableofcontents
\newpage

%======================================================================
\section{\texttt{\&RUN\_PARAMS} -- Global Run Control}
\label{sec:run_params}
%======================================================================

This mandatory block controls the physics modules to activate, restart
behaviour, domain decomposition strategy, and general simulation
parameters.

% ------------------------------------------------------------------
\paramheader{cosmo}{logical}{.false.}
\label{param:cosmo}

\begin{quote}
Enable cosmological mode.  When \texttt{.true.}, RAMSES uses comoving
(super-comoving) coordinates with the expansion factor $a(t)$ as the
time variable.  The box length is interpreted in $h^{-1}\,$Mpc.
Friedmann equations are integrated internally.

Enabling this flag also activates expansion-factor--based output
scheduling (see \paramref{aout} in
Section~\ref{sec:output_params}).  Cosmological initial conditions
(GRAFIC2 format) must be provided via \paramref{initfile}.
\end{quote}

% ------------------------------------------------------------------
\paramheader{pic}{logical}{.false.}
\label{param:pic}

\begin{quote}
Enable the Particle-In-Cell (PIC) method for collisionless $N$-body
dynamics (dark matter, stars).  Particles are deposited onto the AMR
grid using cloud-in-cell (CIC) interpolation, and forces are
interpolated back to particle positions.

Required for any simulation containing dark matter particles.  Usually
combined with \paramref{poisson}\texttt{=.true.}
\end{quote}

% ------------------------------------------------------------------
\paramheader{poisson}{logical}{.false.}
\label{param:poisson}

\begin{quote}
Enable the self-gravity Poisson solver.  RAMSES uses an adaptive
multigrid (MG) method on the AMR hierarchy with V-cycles and
red--black Gauss--Seidel smoothing.  Convergence is controlled by
\paramref{epsilon} in \texttt{\&POISSON\_PARAMS}.

Must be \texttt{.true.}\ whenever \paramref{pic}\texttt{=.true.}\ or
whenever gas self-gravity is desired.
\end{quote}

% ------------------------------------------------------------------
\paramheader{hydro}{logical}{.false.}
\label{param:hydro}

\begin{quote}
Enable the hydrodynamics (or MHD) solver.  RAMSES employs a
second-order MUSCL--Hancock scheme with approximate Riemann solvers
(see \paramref{scheme}, \paramref{riemann} in
Section~\ref{sec:hydro_params}).

Set to \texttt{.true.}\ for any simulation involving baryonic gas.
\end{quote}

% ------------------------------------------------------------------
\paramheader{nrestart}{integer}{0}
\label{param:nrestart}

\begin{quote}
Restart from checkpoint (output snapshot) number \texttt{nrestart}.
\begin{itemize}[nosep]
  \item \texttt{nrestart=0} -- fresh start from initial conditions.
  \item \texttt{nrestart=}$N$ -- load \texttt{output\_}$N$\texttt{/} and resume.
\end{itemize}
The number of MPI processes must match the run that produced the
checkpoint.  RAMSES reads all AMR, hydro, particle, and gravity data
from the snapshot directory.
\end{quote}

% ------------------------------------------------------------------
\paramheader{nremap}{integer}{5}
\label{param:nremap}

\begin{quote}
Load-balancing frequency: perform domain decomposition every
\texttt{nremap} coarse time steps.  Recommended value: \textbf{5}
(balances redistribution overhead against growing load imbalance).
Set to \texttt{0} to disable load balancing entirely.

\apinote{Benchmarks (200\,M particles, 12~ranks, 10~steps) show that
\texttt{nremap=5} reduces total runtime by 18\%\ compared to
\texttt{nremap=1}, with load-balance overhead at 6.3\%\ of wall time.
Larger values (e.g.\ 10) save overhead but allow imbalance to grow.}
\end{quote}

% ------------------------------------------------------------------
\paramheader{nsubcycle}{integer array}{1,1,2}
\label{param:nsubcycle}

\begin{quote}
Time sub-cycling factors per AMR level.  The $i$-th entry gives the
number of fine time steps per coarse step at level
$\texttt{levelmin}+i-1$.  Typical usage:
\begin{itemize}[nosep]
  \item \texttt{1} for coarse levels (no sub-cycling).
  \item \texttt{2} for fine levels (halve the time step at each
        finer level).
\end{itemize}
The array has up to \texttt{levelmax\,--\,levelmin+1} entries.  Any
unspecified trailing entries default to~\texttt{1}.
\end{quote}

\noindent\textcolor{apidefault}{\textsf{\textbf{Example:}}}
\begin{lstlisting}
nsubcycle = 1, 1, 1, 2, 2, 2, 2
\end{lstlisting}

% ------------------------------------------------------------------
\paramheader{ncontrol}{integer}{1}
\label{param:ncontrol}

\begin{quote}
Print control output (energy diagnostics, timing) every
\texttt{ncontrol} coarse time steps to standard output.
\end{quote}

% ------------------------------------------------------------------
\paramheader{nstepmax}{integer}{1000000}
\label{param:nstepmax}

\begin{quote}
Maximum number of coarse time steps.  The simulation stops when
\texttt{nstep} reaches this value, even if the final output time has
not been reached.  Useful for short test runs.
\end{quote}

% ------------------------------------------------------------------
\paramheader{ordering}{character}{`hilbert'}
\label{param:ordering}

\begin{quote}
Domain decomposition ordering strategy.
\begin{description}[style=nextline,leftmargin=2.5cm,font=\ttfamily]
  \item[`hilbert'] Hilbert space-filling curve.  Standard choice for
    moderate core counts ($\lesssim 1000$).
  \item[`ksection'] K-section tree-based decomposition.  Provides
    $O(k)$~message scaling (where $k$ is the branching factor) for
    large core counts.  Enables hierarchical MPI exchanges and
    memory-based load balancing (see \paramref{memory_balance}).
\end{description}
When \texttt{ordering='ksection'}, the communication pattern in ghost
zone exchanges, multigrid solvers, and \texttt{build\_comm} all switch
to ksection tree routing automatically.
\end{quote}

% ------------------------------------------------------------------
\paramheader{memory\_balance}{logical}{.false.}
\label{param:memory_balance}

\begin{quote}
Enable memory-based load balancing.  When \texttt{.true.}, the
bisection histogram weights each cell by its memory footprint (grid
metadata + attached particles) instead of uniform cell count.

\textbf{Requires} \paramref{ordering}\texttt{='ksection'}.

The cell cost function is:
\[
  C_{\text{cell}} = \frac{\texttt{mem\_weight\_grid}}{\texttt{twotondim}}
                + n_{\text{part}} \times
                  \frac{\texttt{mem\_weight\_part}}{\texttt{twotondim}}
\]
where $n_{\text{part}}$ is the number of particles attached to the parent
grid.  The weight parameters \texttt{mem\_weight\_grid} (default~270) and
\texttt{mem\_weight\_part} (default~12) are set in the same namelist block.

\apinote{All histogram variables (\texttt{bisec\_hist},
\texttt{bisec\_cpu\_load}, \texttt{cell\_cost}) use 64-bit integers
(\texttt{integer(i8b)}) and \texttt{MPI\_INTEGER8} to avoid overflow at
high particle counts.}
\end{quote}

% ------------------------------------------------------------------
\paramheader{sink}{logical}{.false.}
\label{param:sink}

\begin{quote}
Enable sink particle formation and evolution.  Sink particles represent
compact objects (e.g.\ black holes, protostars) that accrete gas from
their surroundings.  When active, cells exceeding a density threshold
at \texttt{levelmax} can spawn sink particles.

See also \paramref{sink_AGN}, \paramref{bondi}, \paramref{Mseed} in
Section~\ref{sec:physics_params}.
\end{quote}

% ------------------------------------------------------------------
\paramheader{sinkprops}{logical}{.false.}
\label{param:sinkprops}

\begin{quote}
Output detailed sink particle properties (mass, position, velocity,
accretion rate, spin) to dedicated files at each snapshot.
\end{quote}

% ------------------------------------------------------------------
\paramheader{lightcone}{logical}{.false.}
\label{param:lightcone}

\begin{quote}
Enable lightcone output mode.  When \texttt{.true.}, particles and/or
cells crossing the observer's past lightcone are written to special
output files during the simulation.  See
Section~\ref{sec:lightcone_params} for additional parameters.
\end{quote}

% ------------------------------------------------------------------
\paramheader{verbose}{logical}{.false.}
\label{param:verbose}

\begin{quote}
Enable verbose output during initialization and evolution.  Prints
additional diagnostics (grid counts, memory usage, load balance
statistics) to standard output at each coarse step.
\end{quote}


%======================================================================
\section{\texttt{\&AMR\_PARAMS} -- Adaptive Mesh Refinement}
\label{sec:amr_params}
%======================================================================

This mandatory block controls the AMR grid hierarchy, memory allocation
sizes, and the simulation box geometry.

% ------------------------------------------------------------------
\paramheaderreq{levelmin}{integer}
\label{param:levelmin}

\begin{quote}
Minimum (base) AMR level.  The base grid has $2^{\texttt{levelmin}}$
cells per dimension.

\noindent\textcolor{apidefault}{\textsf{\textbf{Example:}}}
\texttt{levelmin=7} produces a $128^3$ base grid.
\texttt{levelmin=9} produces a $512^3$ base grid.

This level is fully covered -- every cell at \texttt{levelmin} exists
on exactly one MPI process.
\end{quote}

% ------------------------------------------------------------------
\paramheaderreq{levelmax}{integer}
\label{param:levelmax}

\begin{quote}
Maximum AMR level.  Determines the finest attainable resolution:
\[
  \Delta x_{\min} = \frac{L_{\text{box}}}{2^{\texttt{levelmax}}}
\]
For cosmological zoom-in simulations, this controls the physical
resolution at $z=0$.  The number of refinement levels beyond
\texttt{levelmin} is \texttt{levelmax\,--\,levelmin}.

Refinement criteria (\paramref{m_refine}, \paramref{ivar_refine})
determine which cells actually refine up to this level.
\end{quote}

% ------------------------------------------------------------------
\paramheader{nexpand}{integer array}{1}
\label{param:nexpand}

\begin{quote}
Number of buffer (guard) cell layers per level to ensure smooth
transitions between refinement levels.  The $i$-th entry applies to
level $\texttt{levelmin}+i-1$.  Typical value: \texttt{1} for all
levels.

Larger values produce wider buffer zones around refined patches,
improving solution quality at the cost of more cells.
\end{quote}

% ------------------------------------------------------------------
\paramheaderreq{ngridtot}{integer(i8b)}
\label{param:ngridtot}

\begin{quote}
Total number of AMR grids (octs) allocated across all MPI processes.
Each process receives $\texttt{ngridmax} = \texttt{ngridtot} /
\texttt{ncpu}$ grids.  Each grid (oct) contains $2^{\texttt{ndim}}$
cells (8~cells in 3D).

\apiwarning{RAMSES allocates full arrays at startup based on
\texttt{ngridmax}.  The virtual memory footprint is approximately
$\texttt{ngridmax} \times 20\;\text{bytes} \times \texttt{nvar}$.
This must not exceed the system's \texttt{CommitLimit} (typically
RAM~$\times$~\texttt{overcommit\_ratio}/100).}

\textbf{Rule of thumb:} For $N$~processes on a node with $M$\,GB of
RAM,
\[
  \texttt{ngridtot} < \frac{M \times 0.5}{20 \times \texttt{nvar}}
  \times N
\]
\end{quote}

% ------------------------------------------------------------------
\paramheaderreq{nparttot}{integer(i8b)}
\label{param:nparttot}

\begin{quote}
Total particle allocation across all MPI processes.  Each process gets
$\texttt{npartmax} = \texttt{nparttot} / \texttt{ncpu}$.  Should be
at least $2\times$ the total number of DM + star particles expected
during the simulation (to accommodate load imbalance and new star
particle creation).
\end{quote}

\noindent\textcolor{apidefault}{\textsf{\textbf{Example:}}}
\begin{lstlisting}
! 100M DM particles, allow for stars
nparttot = 300000000
\end{lstlisting}

% ------------------------------------------------------------------
\paramheader{boxlen}{real(dp)}{1.0}
\label{param:boxlen}

\begin{quote}
Box length in code units.  For cosmological runs, the box size is
typically read from the IC header (in $h^{-1}\,$Mpc) and this
parameter is overridden.  For non-cosmological (idealised) setups,
\texttt{boxlen} defines the physical domain extent.
\end{quote}


%======================================================================
\section{\texttt{\&OUTPUT\_PARAMS} -- Snapshot Output}
\label{sec:output_params}
%======================================================================

Controls when and how simulation snapshots are written to disk.

% ------------------------------------------------------------------
\paramheader{noutput}{integer}{1}
\label{param:noutput}

\begin{quote}
Number of output snapshots requested.  The corresponding times or
expansion factors must be listed in \paramref{tout} (non-cosmological)
or \paramref{aout} (cosmological).
\end{quote}

% ------------------------------------------------------------------
\paramheader{aout}{real(dp) array}{---}
\label{param:aout}

\begin{quote}
Scale factors at which to write output snapshots (cosmological mode
only, i.e.\ when \paramref{cosmo}\texttt{=.true.}).  The array must
contain \paramref{noutput} entries, in ascending order.
\end{quote}

\noindent\textcolor{apidefault}{\textsf{\textbf{Example:}}}
\begin{lstlisting}
noutput = 4
aout    = 0.1, 0.2, 0.5, 1.0
! Outputs at z = 9, 4, 1, 0
\end{lstlisting}

% ------------------------------------------------------------------
\paramheader{tout}{real(dp) array}{---}
\label{param:tout}

\begin{quote}
Output times in code units (non-cosmological mode).  The array must
contain \paramref{noutput} entries, in ascending order.
\end{quote}

% ------------------------------------------------------------------
\paramheader{foutput}{integer}{1000000}
\label{param:foutput}

\begin{quote}
Write an output snapshot every \texttt{foutput} coarse time steps,
regardless of the \paramref{aout}/\paramref{tout} schedule.  Useful
for periodic checkpointing in long runs.  Set to a very large number
to effectively disable.
\end{quote}

% ------------------------------------------------------------------
\paramheader{outformat}{character}{`original'}
\label{param:outformat}

\begin{quote}
Output file format for snapshots.
\begin{description}[style=nextline,leftmargin=2.5cm,font=\ttfamily]
  \item[`original'] Standard RAMSES per-CPU binary format.  Each MPI
    process writes separate files (\texttt{amr\_NNNNN.outNNNNN},
    \texttt{hydro\_NNNNN.outNNNNN}, etc.).
  \item[`hdf5'] Single HDF5 file per snapshot (\texttt{data\_NNNNN.h5}).
    Uses MPI parallel I/O for collective writes.  The HDF5 file stores
    all AMR, hydro, gravity, particle, and sink data in a hierarchical
    group structure.  \textbf{Requires compilation with}
    \texttt{make~HDF5=1}.
\end{description}

\apinote{The standard auxiliary files (\texttt{info\_NNNNN.txt},
\texttt{header\_NNNNN.txt}, \texttt{compilation.txt},
\texttt{makefile.txt}, \texttt{namelist.txt}) are always written
regardless of \texttt{outformat}.}
\end{quote}

% ------------------------------------------------------------------
\paramheader{informat}{character}{`original'}
\label{param:informat}

\begin{quote}
Input (restart) file format.
\begin{description}[style=nextline,leftmargin=2.5cm,font=\ttfamily]
  \item[`original'] Read from standard per-CPU binary files.  The
    number of MPI processes must match the run that produced the
    checkpoint.
  \item[`hdf5'] Read from the single HDF5 file
    (\texttt{data\_NNNNN.h5}).  Currently requires the same number of
    MPI processes as the original run.  \textbf{Requires compilation
    with} \texttt{make~HDF5=1}.
\end{description}

\apinote{\texttt{informat} and \texttt{outformat} can be set
independently, allowing cross-format conversion (e.g.\ restart from
binary and output to HDF5, or vice versa).}
\end{quote}


%======================================================================
\section{\texttt{\&INIT\_PARAMS} -- Initial Conditions}
\label{sec:init_params}
%======================================================================

Specifies the format and location of initial condition files.

% ------------------------------------------------------------------
\paramheader{filetype}{character}{`grafic'}
\label{param:filetype}

\begin{quote}
Initial condition file format.
\begin{description}[style=nextline,leftmargin=2.5cm,font=\ttfamily]
  \item[`grafic'] GRAFIC2 binary format (Bertschinger 2001).  Each
    level's IC directory contains binary files for density
    perturbations, velocities, and (optionally) particle
    displacements.
  \item[`ascii'] Text-based initial conditions (for simple test
    problems).
\end{description}
\end{quote}

% ------------------------------------------------------------------
\paramheader{initfile}{character array}{---}
\label{param:initfile}

\begin{quote}
Paths to IC directories, one per AMR level.
\texttt{initfile(1)} corresponds to \paramref{levelmin},
\texttt{initfile(2)} to $\texttt{levelmin}+1$, and so on.

Each directory must contain the following binary files:
\begin{itemize}[nosep]
  \item \texttt{ic\_deltab} -- baryon density perturbation field
  \item \texttt{ic\_velbx}, \texttt{ic\_velby}, \texttt{ic\_velbz}
        -- baryon velocity fields
  \item \texttt{ic\_velcx}, \texttt{ic\_velcy}, \texttt{ic\_velcz}
        -- dark matter (CDM) velocity fields
  \item \texttt{ic\_poscx}, \texttt{ic\_poscy}, \texttt{ic\_poscz}
        -- dark matter displacement fields (optional, for multi-level
        zoom-in)
  \item \texttt{ic\_tempb} -- baryon temperature perturbation
        (optional)
  \item \texttt{ic\_pvar\_00001}, \ldots\ -- passive scalar fields
        (optional, for zoom-in refinement tagging;
        see \paramref{ivar_refine})
  \item \texttt{ic\_refmap} -- refinement map (optional)
\end{itemize}
\end{quote}

\noindent\textcolor{apidefault}{\textsf{\textbf{Example:}}}
\begin{lstlisting}
initfile = '/data/IC/level_07'
         , '/data/IC/level_08'
         , '/data/IC/level_09'
\end{lstlisting}


%======================================================================
\section{\texttt{\&REFINE\_PARAMS} -- Refinement Criteria}
\label{sec:refine_params}
%======================================================================

Controls which cells are refined in the AMR hierarchy.  These
parameters are \textbf{critical for zoom-in simulations}, where
background regions must remain coarse while the zoom region refines
to high resolution.

% ------------------------------------------------------------------
\paramheader{m\_refine}{real(dp) array}{$-1$}
\label{param:m_refine}

\begin{quote}
Quasi-Lagrangian mass threshold per level.  The $i$-th entry applies
to level $\texttt{levelmin}+i-1$.  A cell is flagged for refinement
when the effective mass indicator $\phi \geq \texttt{m\_refine}(i)$.

Typical value: \textbf{8.0} for all levels (refine when the
equivalent of $\geq 8$ particles occupies a cell).  Provide one
entry for each level from \texttt{levelmin} to \texttt{levelmax}.

Interacts with \paramref{ivar_refine} and
\paramref{mass_cut_refine} to determine which particles contribute
to the density used for the refinement decision.
\end{quote}

\noindent\textcolor{apidefault}{\textsf{\textbf{Example:}}}
\begin{lstlisting}
! 6 levels of refinement (levelmin=7, levelmax=13)
m_refine = 8., 8., 8., 8., 8., 8., 8.
\end{lstlisting}

% ------------------------------------------------------------------
\paramheader{ivar\_refine}{integer}{$-1$}
\label{param:ivar_refine}

\begin{quote}
Variable index controlling the refinement criterion in
\texttt{poisson\_refine}.  This parameter fundamentally changes how
refinement regions are selected:
\begin{description}[style=nextline,leftmargin=1em]
  \item[\texttt{ivar\_refine = 0}:]
    Use \texttt{cpu\_map2} for refinement control.  During
    initialization, \texttt{cpu\_map2} is set by
    \texttt{init\_refmap} from \texttt{ic\_refmap} (if present);
    during evolution, it is updated by \texttt{rho\_fine} based on
    the local density field.  This is the standard quasi-Lagrangian
    approach.

    \apiwarning{In zoom-in simulations, this can cause uncontrolled
    AMR expansion into background regions if \texttt{cpu\_map2} is
    not properly restricted by \paramref{mass_cut_refine}.}

  \item[\texttt{ivar\_refine > 0} (e.g.\ 11):]
    During initialization, use passive scalar criterion:
    \texttt{uold(cell, ivar\_refine) / uold(cell, 1)} $>$
    \paramref{var_cut_refine}.

    \textbf{Recommended for zoom-in:} set
    \texttt{ivar\_refine=NVAR} (the last hydro variable), and create
    \texttt{ic\_pvar\_NNNNN} files with value 1.0 inside the zoom
    region and 0.0 in the background.  After initialization,
    \texttt{cpu\_map2} (set by \texttt{rho\_fine} with
    \paramref{mass_cut_refine} filtering) takes over.

  \item[\texttt{ivar\_refine < 0} (default):]
    Pure density-based refinement at both initialization and runtime.
    A cell is refined when
    \texttt{uold(cell, 1)} $\geq$ \texttt{m\_refine}
    $\times\; m_{\text{sph}} / V_{\text{cell}}$.
\end{description}
\end{quote}

% ------------------------------------------------------------------
\paramheader{var\_cut\_refine}{real(dp)}{$-1$}
\label{param:var_cut_refine}

\begin{quote}
Threshold for passive-scalar-based refinement when
\paramref{ivar_refine} $> 0$.  A cell is refined only if
\[
  \frac{\texttt{uold}(\text{cell},\;\texttt{ivar\_refine})}%
       {\texttt{uold}(\text{cell},\;1)}
  > \texttt{var\_cut\_refine}
\]
Typical value: \textbf{0.01} for zoom geometry tagging (the passive
scalar is 1.0 inside the zoom region, 0.0 outside).
\end{quote}

% ------------------------------------------------------------------
\paramheader{mass\_cut\_refine}{real(dp)}{$-1$}
\label{param:mass_cut_refine}

\begin{quote}
Particle mass threshold for quasi-Lagrangian refinement.  In
\texttt{rho\_fine}, dark matter particles with mass
$\geq \texttt{mass\_cut\_refine}$ are \emph{excluded} from the
density computation that drives cell refinement.  This prevents
heavy (coarse-level) background particles from triggering spurious
refinement.

Set this to the DM particle mass at the finest IC level.  Reference
values for a $100\,h^{-1}\,$Mpc box
($\Omega_m = 0.3$, $h = 0.68$):

\begin{center}
\small
\begin{tabular}{@{}ll@{}}
\toprule
\textbf{IC finest level} & \textbf{mass\_cut\_refine} \\
\midrule
8  & \texttt{1.19209e-07} \\
9  & \texttt{1.49012e-08} \\
10 & \texttt{1.86265e-09} \\
11 & \texttt{2.32831e-10} \\
12 & \texttt{2.91038e-11} \\
13 & \texttt{3.63798e-12} \\
\bottomrule
\end{tabular}
\end{center}

\apinote{This parameter interacts with \paramref{ivar_refine} and
\paramref{m_refine}.  All three should be set consistently for
zoom-in simulations.}
\end{quote}

% ------------------------------------------------------------------
\paramheader{interpol\_var}{integer}{0}
\label{param:interpol_var}

\begin{quote}
Interpolation variable type used when prolongating (interpolating)
data from coarse to fine grids.
\begin{description}[style=nextline,leftmargin=1.5cm,font=\ttfamily]
  \item[0] Conservative variables ($\rho$, $\rho v$, $E$).
  \item[1] Primitive variables ($\rho$, $v$, $P$).
        \textbf{Recommended} for cosmological simulations to avoid
        interpolation artefacts in low-density regions.
\end{description}
\end{quote}

% ------------------------------------------------------------------
\paramheader{interpol\_type}{integer}{1}
\label{param:interpol_type}

\begin{quote}
Interpolation slope limiter for prolongation.
\begin{description}[style=nextline,leftmargin=1.5cm,font=\ttfamily]
  \item[0] MinMod limiter -- more diffusive, more robust.
  \item[1] MonCen (monotonised central) limiter -- less diffusive.
        \textbf{Recommended}.
\end{description}
\end{quote}

% ------------------------------------------------------------------
\paramheader{sink\_refine}{logical}{.false.}
\label{param:sink_refine}

\begin{quote}
Force maximum refinement around sink particles.  When \texttt{.true.},
a contribution equal to \paramref{m_refine} is added to the
refinement indicator $\phi$ for every cell containing a sink particle,
ensuring refinement up to \paramref{levelmax}.
\end{quote}

% ------------------------------------------------------------------
\paramheader{jeans\_ncells}{real(dp)}{$-1$}
\label{param:jeans_ncells}

\begin{quote}
Jeans refinement criterion.  If $> 0$, cells are refined to resolve
the local Jeans length by at least this many cells:
\[
  \Delta x < \frac{\lambda_J}{\texttt{jeans\_ncells}}
\]
Enabling this also activates a polytropic equation-of-state floor to
prevent artificial fragmentation (Truelove criterion).  Typical
value: \textbf{4} (minimum of 4 cells per Jeans length).
\end{quote}


%======================================================================
\section{\texttt{\&HYDRO\_PARAMS} -- Hydrodynamics Solver}
\label{sec:hydro_params}
%======================================================================

Controls the gas dynamics solver configuration.

% ------------------------------------------------------------------
\paramheader{gamma}{real(dp)}{$5/3$}
\label{param:gamma}

\begin{quote}
Adiabatic index $\gamma$ of the ideal gas equation of state,
$P = (\gamma - 1) \rho e$.
Standard value: $5/3$ for a monatomic ideal gas.  Use $7/5$ for
diatomic gas or $4/3$ for radiation-dominated flow.
\end{quote}

% ------------------------------------------------------------------
\paramheader{courant\_factor}{real(dp)}{0.8}
\label{param:courant_factor}

\begin{quote}
Courant--Friedrichs--Lewy (CFL) number for time step control.  The
time step at each level is
$\Delta t = \texttt{courant\_factor} \times \Delta x / v_{\max}$.
Typical: \textbf{0.8}.  Lower values increase stability at the cost
of more time steps.
\end{quote}

% ------------------------------------------------------------------
\paramheader{scheme}{character}{`muscl'}
\label{param:scheme}

\begin{quote}
Hydrodynamics integration scheme.
\begin{description}[style=nextline,leftmargin=2.5cm,font=\ttfamily]
  \item[`muscl'] MUSCL--Hancock (Monotonic Upstream-centred Scheme
    for Conservation Laws), second-order in space and time.  This is
    the only production scheme in RAMSES.
\end{description}
\end{quote}

% ------------------------------------------------------------------
\paramheader{slope\_type}{integer}{1}
\label{param:slope_type}

\begin{quote}
Slope limiter for MUSCL piecewise-linear reconstruction.
\begin{description}[style=nextline,leftmargin=1.5cm,font=\ttfamily]
  \item[1] MinMod -- most robust, more diffusive.
  \item[2] MonCen -- monotonised central, less diffusive.
        \textbf{Recommended} for production runs.
  \item[3] Unlimited -- no limiting (unstable; testing only).
\end{description}
\end{quote}

% ------------------------------------------------------------------
\paramheader{riemann}{character}{`llf'}
\label{param:riemann}

\begin{quote}
Approximate Riemann solver for inter-cell flux computation.
\begin{description}[style=nextline,leftmargin=2.5cm,font=\ttfamily]
  \item[`llf'] Local Lax--Friedrichs (Rusanov).  Most diffusive but
    unconditionally stable.  Good default.
  \item[`hll'] Harten--Lax--van Leer.  Two-wave solver.
  \item[`hllc'] HLL with Contact restoration.  Three-wave solver,
    most accurate for contact discontinuities.
    \textbf{Recommended for cosmological simulations}.
  \item[`exact'] Exact Riemann solver (expensive; primarily for
    validation).
\end{description}
\end{quote}

% ------------------------------------------------------------------
\paramheader{pressure\_fix}{logical}{.false.}
\label{param:pressure_fix}

\begin{quote}
Enable pressure floor to prevent negative pressures in strong shocks
or highly supersonic flows.  When the internal energy becomes negative,
RAMSES falls back to a pressure estimate from the total energy.

\textbf{Recommended:} \texttt{.true.}\ for cosmological simulations.
See also \paramref{beta_fix}.
\end{quote}

% ------------------------------------------------------------------
\paramheader{beta\_fix}{real(dp)}{0.0}
\label{param:beta_fix}

\begin{quote}
Pressure fix parameter.  Controls the magnitude of the pressure floor:
$P_{\text{floor}} = \texttt{beta\_fix} \times \rho v^2 / 2$.
Typical value: \textbf{0.5} when
\paramref{pressure_fix}\texttt{=.true.}
\end{quote}

% ------------------------------------------------------------------
\paramheader{isothermal}{logical}{.false.}
\label{param:isothermal}

\begin{quote}
Isothermal mode.  When \texttt{.true.}, the energy equation is not
solved and the gas temperature remains constant.  Reduces the number
of hydro variables by one.
\end{quote}


%======================================================================
\section{\texttt{\&POISSON\_PARAMS} -- Gravity Solver}
\label{sec:poisson_params}
%======================================================================

Controls the multigrid Poisson solver for self-gravity.

% ------------------------------------------------------------------
\paramheader{epsilon}{real(dp)}{$10^{-4}$}
\label{param:epsilon}

\begin{quote}
Multigrid convergence criterion.  The V-cycle iteration at each level
stops when the residual norm satisfies
$\|r\| / \|r_0\| < \texttt{epsilon}$.
Typical value for cosmological runs: $10^{-5}$ to $10^{-4}$.
Tighter values improve force accuracy but increase iteration count.
\end{quote}

% ------------------------------------------------------------------
\paramheader{gravity\_type}{integer}{0}
\label{param:gravity_type}

\begin{quote}
Gravity model selection.
\begin{description}[style=nextline,leftmargin=1.5cm,font=\ttfamily]
  \item[0] Self-gravity (solve Poisson equation on the AMR grid).
  \item[>0] Analytical gravitational potential (e.g.\ for test
    problems with known solutions).  The integer value selects the
    specific analytical profile.
\end{description}
\end{quote}

% ------------------------------------------------------------------
\paramheader{cg\_levelmin}{integer}{999}
\label{param:cg_levelmin}

\begin{quote}
Minimum level at which the conjugate gradient (CG) fallback solver
activates.  When the multigrid solver stalls at high AMR levels, CG
provides guaranteed convergence.  Set to \paramref{levelmax} for
best convergence behaviour.

Typical: \texttt{cg\_levelmin = levelmax}.  The default (999) means
CG is effectively disabled unless \texttt{levelmax} is absurdly
large.
\end{quote}

% ------------------------------------------------------------------
\paramheader{cic\_levelmax}{integer}{0}
\label{param:cic_levelmax}

\begin{quote}
Maximum level for cloud-in-cell (CIC) particle mass deposition.
\begin{itemize}[nosep]
  \item \texttt{0} -- deposit particles at all levels (standard).
  \item $N > 0$ -- deposit particles only up to level $N$; finer
    levels inherit the coarse density by prolongation.
\end{itemize}
Rarely modified.
\end{quote}


%======================================================================
\section{\texttt{\&PHYSICS\_PARAMS} -- Sub-grid Physics}
\label{sec:physics_params}
%======================================================================

Controls cooling, star formation, stellar/AGN feedback, and
cosmological parameters.  This block is optional; omit it entirely
for adiabatic (non-radiative) simulations.

%----------------------------------------------------------------------
\subsection{Cooling and UV Background}
%----------------------------------------------------------------------

\paramheader{cooling}{logical}{.false.}
\label{param:cooling}

\begin{quote}
Enable radiative cooling with a metal-dependent cooling function.
When \texttt{.true.}, RAMSES integrates the cooling/heating rate
at each time step using tabulated cooling curves.  Requires
\paramref{hydro}\texttt{=.true.}
\end{quote}

% ------------------------------------------------------------------
\paramheader{haardt\_madau}{logical}{.false.}
\label{param:haardt_madau}

\begin{quote}
Enable the Haardt \& Madau (2012) ultraviolet background model for
cosmic reionization.  Provides a redshift-dependent photo-heating
and photo-ionization rate.  Used together with \paramref{cooling}.
\end{quote}

% ------------------------------------------------------------------
\paramheader{z\_reion}{real(dp)}{8.5}
\label{param:z_reion}

\begin{quote}
Reionization redshift.  Hydrogen reionization heating is applied
instantaneously at this redshift.  Typical range: 6--10, depending
on the reionization model.
\end{quote}

% ------------------------------------------------------------------
\paramheader{z\_ave}{real(dp)}{0.0}
\label{param:z_ave}

\begin{quote}
Initial mean metallicity of the gas in solar units ($Z_\odot$).
Applied uniformly at initialization.  Use \texttt{0.0} for
primordial composition.
\end{quote}

% ------------------------------------------------------------------
\paramheader{delayed\_cooling}{logical}{.false.}
\label{param:delayed_cooling}

\begin{quote}
Delay radiative cooling in supernova-heated gas to prevent
overcooling.  When a cell receives SN energy, cooling is suppressed
for a duration related to the Sedov--Taylor phase.  Improves the
effectiveness of stellar feedback in regulating star formation.
\end{quote}

% ------------------------------------------------------------------
\paramheader{tol}{real(dp)}{$10^{-3}$}
\label{param:tol}

\begin{quote}
Tolerance for the implicit cooling solver.  The Newton--Raphson
iteration converges when the relative temperature change
$|\Delta T / T| < \texttt{tol}$.
\end{quote}

%----------------------------------------------------------------------
\subsection{Star Formation}
%----------------------------------------------------------------------

\paramheader{n\_star}{real(dp)}{0.1}
\label{param:n_star}

\begin{quote}
Star formation hydrogen number density threshold in
$\text{H}\;\text{cm}^{-3}$.  Only gas denser than this value is
eligible for star formation.  Typical range: 0.1--10.
\end{quote}

% ------------------------------------------------------------------
\paramheader{eps\_star}{real(dp)}{0.0}
\label{param:eps_star}

\begin{quote}
Star formation efficiency per free-fall time $\epsilon_{\text{ff}}$.
The star formation rate density is
$\dot{\rho}_\star = \epsilon_{\text{ff}} \, \rho_{\text{gas}} /
t_{\text{ff}}$.
Typical value: \textbf{0.01--0.02} (1--2\%\ per free-fall time).
Set to \texttt{0.0} to disable star formation entirely.
\end{quote}

% ------------------------------------------------------------------
\paramheader{del\_star}{real(dp)}{200}
\label{param:del_star}

\begin{quote}
Overdensity threshold for star formation (in units of the cosmic
mean density).  Gas must exceed $\delta > \texttt{del\_star}$ in
addition to the density threshold \paramref{n_star}.
\end{quote}

% ------------------------------------------------------------------
\paramheader{m\_star}{real(dp)}{$-1$}
\label{param:m_star}

\begin{quote}
Minimum stellar particle mass in code units.  When a star-forming
cell would produce a particle below this mass, the event is
stochastically deferred to the next time step.
\begin{itemize}[nosep]
  \item $< 0$: use the cell gas mass (no minimum).
  \item $> 0$: explicit minimum mass.
\end{itemize}
\end{quote}

% ------------------------------------------------------------------
\paramheader{T2\_star}{real(dp)}{0}
\label{param:T2_star}

\begin{quote}
ISM polytropic equation-of-state temperature floor in Kelvin.  Gas
above the star-formation density threshold \paramref{n_star} follows
a polytropic relation:
\[
  T = T_{2,\star} \left(\frac{n}{n_\star}\right)^{\gamma_\star - 1}
\]
where $\gamma_\star$ is \paramref{g_star}.  This prevents artificial
fragmentation below the resolution limit (Jeans mass floor).
\end{quote}

% ------------------------------------------------------------------
\paramheader{g\_star}{real(dp)}{1.6}
\label{param:g_star}

\begin{quote}
Polytropic index $\gamma_\star$ for the ISM equation of state (see
\paramref{T2_star}).  Typical value: \textbf{1.6} (stiff polytrope)
or \textbf{5/3} (adiabatic floor).
\end{quote}

%----------------------------------------------------------------------
\subsection{Stellar Feedback}
%----------------------------------------------------------------------

\paramheader{f\_w}{real(dp)}{0}
\label{param:f_w}

\begin{quote}
Mass loading factor for supernova-driven winds.  The wind mass flux
is $\dot{M}_w = \texttt{f\_w} \times \dot{M}_\star$.  Set to
\texttt{0} to disable winds.  Typical range: 1--5.
\end{quote}

% ------------------------------------------------------------------
\paramheader{f\_ek}{real(dp)}{1.0}
\label{param:f_ek}

\begin{quote}
Kinetic energy fraction of supernova feedback.  Controls the
partition between kinetic (\texttt{f\_ek}) and thermal
($1 - \texttt{f\_ek}$) energy injection.  \texttt{f\_ek=1} is
purely kinetic feedback; \texttt{f\_ek=0} is purely thermal.
\end{quote}

% ------------------------------------------------------------------
\paramheader{eps\_sn1}{real(dp)}{0}
\label{param:eps_sn1}

\begin{quote}
Type Ia supernova energy per event in units of $10^{51}\,$erg.
Set to \texttt{0} to disable Type Ia SN feedback.
\end{quote}

% ------------------------------------------------------------------
\paramheader{eps\_sn2}{real(dp)}{0}
\label{param:eps_sn2}

\begin{quote}
Type II supernova energy per event in units of $10^{51}\,$erg.
Set to \texttt{0} to disable Type II SN feedback.
\end{quote}

% ------------------------------------------------------------------
\paramheader{yieldtablefilename}{character}{---}
\label{param:yieldtablefilename}

\begin{quote}
Path to the chemical yield table file for metal enrichment
calculations.  Required when metal-dependent cooling or chemical
evolution tracking is enabled.
\end{quote}

%----------------------------------------------------------------------
\subsection{Cosmological Parameters}
%----------------------------------------------------------------------

\paramheader{omega\_b}{real(dp)}{---}
\label{param:omega_b}

\begin{quote}
Baryon density parameter $\Omega_b$.  Overrides the value read from
the IC file header.  Must be consistent with the initial conditions
and other cosmological parameters ($\Omega_m$, $H_0$, etc.\ are
read from the GRAFIC2 header).
\end{quote}

%----------------------------------------------------------------------
\subsection{AGN and Sink Particle Parameters}
%----------------------------------------------------------------------

\paramheader{Mseed}{real(dp)}{---}
\label{param:Mseed}

\begin{quote}
Seed black hole mass in solar masses ($M_\odot$).  When a sink
particle forms, it is initialised with this mass.  Typical range:
$10^4$--$10^6\,M_\odot$ for cosmological simulations.
\end{quote}

% ------------------------------------------------------------------
\paramheader{sink\_AGN}{logical}{.false.}
\label{param:sink_AGN}

\begin{quote}
Enable AGN feedback from sink particles.  When \texttt{.true.}, sink
particles inject thermal and/or kinetic energy into their surroundings
based on their accretion rate.  Requires
\paramref{sink}\texttt{=.true.}
\end{quote}

% ------------------------------------------------------------------
\paramheader{bondi}{logical}{.false.}
\label{param:bondi}

\begin{quote}
Enable Bondi--Hoyle--Lyttleton accretion for sink particles.  The
accretion rate is computed from the local gas density, sound speed,
and relative velocity:
\[
  \dot{M} = \frac{4\pi G^2 M_{\text{BH}}^2 \rho}%
             {(c_s^2 + v_{\text{rel}}^2)^{3/2}}
\]
Can be boosted by \paramref{boost_acc}.
\end{quote}

% ------------------------------------------------------------------
\paramheader{drag}{logical}{.false.}
\label{param:drag}

\begin{quote}
Enable dynamical friction on sink particles.  Applies a drag force
opposing the sink's motion relative to the background gas.  Strength
can be amplified by \paramref{boost_drag}.
\end{quote}

% ------------------------------------------------------------------
\paramheader{rAGN}{real(dp)}{---}
\label{param:rAGN}

\begin{quote}
AGN feedback energy injection radius in units of the cell size at
\paramref{levelmax}.  Feedback energy is distributed over a sphere
of this radius centred on the sink particle.
\end{quote}

% ------------------------------------------------------------------
\paramheader{X\_floor}{real(dp)}{---}
\label{param:X_floor}

\begin{quote}
Hydrogen mass fraction floor.  Prevents the hydrogen fraction from
dropping below this value due to numerical artefacts.  Typical:
\texttt{0.76}.
\end{quote}

% ------------------------------------------------------------------
\paramheader{eAGN\_K}{real(dp)}{---}
\label{param:eAGN_K}

\begin{quote}
AGN kinetic feedback efficiency $\epsilon_K$.  Fraction of the
accreted rest-mass energy deposited as kinetic energy:
$\dot{E}_K = \epsilon_K \, \dot{M} c^2$.
\end{quote}

% ------------------------------------------------------------------
\paramheader{eAGN\_T}{real(dp)}{---}
\label{param:eAGN_T}

\begin{quote}
AGN thermal feedback efficiency $\epsilon_T$.  Fraction of the
accreted rest-mass energy deposited as thermal energy:
$\dot{E}_T = \epsilon_T \, \dot{M} c^2$.
\end{quote}

% ------------------------------------------------------------------
\paramheader{TAGN}{real(dp)}{---}
\label{param:TAGN}

\begin{quote}
AGN heating temperature in Kelvin.  The AGN thermal energy is
deposited by raising gas temperature toward this value within the
feedback region \paramref{rAGN}.
\end{quote}

% ------------------------------------------------------------------
\paramheader{r\_gal}{real(dp)}{---}
\label{param:r_gal}

\begin{quote}
Galaxy definition radius for AGN feedback, in code units.  Used to
compute the local galaxy properties (stellar mass, gas mass) around
a sink particle for AGN mode switching.
\end{quote}

% ------------------------------------------------------------------
\paramheader{T2maxAGN}{real(dp)}{---}
\label{param:T2maxAGN}

\begin{quote}
Maximum AGN heating temperature in Kelvin.  Caps the temperature
increase from a single AGN feedback event to prevent unphysically
hot gas.
\end{quote}

% ------------------------------------------------------------------
\paramheader{boost\_acc}{real(dp)}{---}
\label{param:boost_acc}

\begin{quote}
Bondi accretion boost factor.  Multiplies the Bondi--Hoyle accretion
rate by this factor to compensate for unresolved gas structure near
the black hole.  Typical range: 1--100.  Requires
\paramref{bondi}\texttt{=.true.}
\end{quote}

% ------------------------------------------------------------------
\paramheader{boost\_drag}{real(dp)}{---}
\label{param:boost_drag}

\begin{quote}
Dynamical friction drag boost factor.  Multiplies the drag force by
this factor.  Requires \paramref{drag}\texttt{=.true.}
\end{quote}

% ------------------------------------------------------------------
\paramheader{vrel\_merge}{logical}{---}
\label{param:vrel_merge}

\begin{quote}
Use relative velocity criterion for sink particle merging.  When
\texttt{.true.}, two sinks merge only if their relative velocity is
below the local escape velocity, in addition to the spatial proximity
criterion \paramref{rmerge}.
\end{quote}

% ------------------------------------------------------------------
\paramheader{rmerge}{real(dp)}{---}
\label{param:rmerge}

\begin{quote}
Sink merging radius in units of the cell size at
\paramref{levelmax}.  Two sink particles closer than this distance
are candidates for merging (subject to additional criteria if
\paramref{vrel_merge}\texttt{=.true.}).
\end{quote}

% ------------------------------------------------------------------
\paramheader{spin\_bh}{logical}{---}
\label{param:spin_bh}

\begin{quote}
Track black hole spin evolution.  When \texttt{.true.}, the code
evolves the dimensionless spin parameter $a_\star$ of each sink
particle based on the angular momentum of accreted gas.
\end{quote}

% ------------------------------------------------------------------
\paramheader{mad\_jet}{logical}{---}
\label{param:mad_jet}

\begin{quote}
Enable the magnetically arrested disk (MAD) jet model.  When
\texttt{.true.}, AGN kinetic feedback is launched as a collimated
bipolar jet aligned with the black hole spin axis.  Requires
\paramref{spin_bh}\texttt{=.true.}
\end{quote}


%======================================================================
\section{\texttt{\&LIGHTCONE\_PARAMS} -- Lightcone Output}
\label{sec:lightcone_params}
%======================================================================

\begin{quote}
Parameters for lightcone output mode (activated when
\paramref{lightcone}\texttt{=.true.}).  In this mode, particles
and/or cells that cross the observer's past lightcone during each
time step are written to special output files, enabling the
construction of mock galaxy surveys and weak-lensing maps without
storing full snapshots.

Configuration parameters include the observer position, opening
angle, and selection criteria.  Consult the RAMSES lightcone
documentation for the full parameter list, which varies by
application.
\end{quote}


%======================================================================
\section{\texttt{\&SPHERICAL\_REGION\_PARAMS}}
\label{sec:spherical_region_params}
%======================================================================

\paramheader{spherical\_region}{logical}{.false.}
\label{param:spherical_region}

\begin{quote}
Enable a spherical refinement region.  When \texttt{.true.}, AMR
refinement is restricted to a spherical sub-volume of the simulation
box.  This is useful for re-simulations of specific halos where a
cubic zoom region is not optimal.  Additional parameters define the
centre and radius of the sphere.
\end{quote}


%======================================================================
\section{Complete Example: Cosmological Zoom-In}
\label{sec:example}
%======================================================================

The following namelist illustrates a production cosmological zoom-in
simulation with dark matter particles, baryonic gas, self-gravity,
cooling, star formation, and AGN feedback.

\begin{lstlisting}[language=Fortran,
  caption={Cosmological zoom-in namelist},
  label={lst:zoomin}]
&RUN_PARAMS
cosmo    = .true.
pic      = .true.
poisson  = .true.
hydro    = .true.
sink     = .true.
nrestart = 0
nremap   = 5
nsubcycle = 1, 1, 1, 2, 2, 2, 2
nstepmax = 10000000
ordering = 'ksection'
memory_balance = .true.
/

&AMR_PARAMS
levelmin  = 7
levelmax  = 18
nexpand   = 1, 1, 1, 1, 1, 1, 1, 1, 1, 1, 1, 1
ngridtot  = 400000000
nparttot  = 600000000
/

&OUTPUT_PARAMS
noutput = 10
aout = 0.05, 0.1, 0.15, 0.2, 0.3, 0.4, 0.5, 0.7, 0.85, 1.0
foutput = 500
/

&INIT_PARAMS
filetype = 'grafic'
initfile = '/data/IC/level_07'
         , '/data/IC/level_08'
         , '/data/IC/level_09'
         , '/data/IC/level_10'
         , '/data/IC/level_11'
         , '/data/IC/level_12'
         , '/data/IC/level_13'
/

&REFINE_PARAMS
m_refine = 8., 8., 8., 8., 8., 8., 8., 8., 8., 8., 8., 8.
ivar_refine     = 11
var_cut_refine  = 0.01
mass_cut_refine = 3.63798e-12
interpol_var    = 1
interpol_type   = 1
/

&HYDRO_PARAMS
gamma         = 1.6666667
courant_factor = 0.8
scheme        = 'muscl'
slope_type    = 2
riemann       = 'hllc'
pressure_fix  = .true.
beta_fix      = 0.5
/

&POISSON_PARAMS
epsilon      = 1.0e-5
gravity_type = 0
cg_levelmin  = 18
/

&PHYSICS_PARAMS
cooling        = .true.
haardt_madau   = .true.
z_reion        = 8.5
n_star         = 0.1
eps_star       = 0.02
T2_star        = 1.0e4
g_star         = 1.6
del_star       = 200.0
f_ek           = 1.0
sink_AGN       = .true.
bondi          = .true.
Mseed          = 1.0e5
/
\end{lstlisting}


%======================================================================
\section{Parameter Cross-Reference Index}
\label{sec:crossref}
%======================================================================

Table~\ref{tab:crossref} lists parameters that commonly interact and
should be set consistently.

\begin{longtable}{@{}p{4cm}p{4.5cm}p{6.5cm}@{}}
\caption{Cross-reference of interacting parameters.}
\label{tab:crossref}\\
\toprule
\textbf{Parameter} & \textbf{Related Parameters} & \textbf{Notes} \\
\midrule
\endfirsthead
\toprule
\textbf{Parameter} & \textbf{Related Parameters} & \textbf{Notes} \\
\midrule
\endhead
\bottomrule
\endfoot
\texttt{cosmo}
  & \texttt{aout}, \texttt{omega\_b}
  & Cosmological mode requires scale-factor outputs \\
\addlinespace
\texttt{pic}
  & \texttt{poisson}, \texttt{nparttot}
  & Particles need gravity and memory allocation \\
\addlinespace
\texttt{ordering}
  & \texttt{memory\_balance}
  & Memory balancing requires \texttt{ksection} \\
\addlinespace
\texttt{levelmax}
  & \texttt{m\_refine}, \texttt{cg\_levelmin}
  & Set \texttt{cg\_levelmin = levelmax} \\
\addlinespace
\texttt{ivar\_refine}
  & \texttt{var\_cut\_refine}, \texttt{mass\_cut\_refine}, \texttt{m\_refine}
  & All must be consistent for zoom-in \\
\addlinespace
\texttt{mass\_cut\_refine}
  & \texttt{ivar\_refine}
  & Set to finest-level DM particle mass \\
\addlinespace
\texttt{pressure\_fix}
  & \texttt{beta\_fix}
  & \texttt{beta\_fix} only effective when fix is on \\
\addlinespace
\texttt{T2\_star}
  & \texttt{g\_star}, \texttt{n\_star}
  & Polytropic EOS parameters \\
\addlinespace
\texttt{sink}
  & \texttt{sink\_AGN}, \texttt{bondi}, \texttt{Mseed}
  & AGN feedback requires sink particles \\
\addlinespace
\texttt{sink\_AGN}
  & \texttt{eAGN\_K}, \texttt{eAGN\_T}, \texttt{TAGN}, \texttt{rAGN}
  & AGN feedback parameters \\
\addlinespace
\texttt{bondi}
  & \texttt{boost\_acc}
  & Boost factor for unresolved accretion \\
\addlinespace
\texttt{drag}
  & \texttt{boost\_drag}
  & Drag boost factor \\
\addlinespace
\texttt{spin\_bh}
  & \texttt{mad\_jet}
  & MAD jet requires spin tracking \\
\addlinespace
\texttt{cooling}
  & \texttt{haardt\_madau}, \texttt{z\_reion}
  & UV background for reionization heating \\
\addlinespace
\texttt{ngridtot}
  & \texttt{nparttot}
  & Both determine per-process memory usage \\
\end{longtable}


%======================================================================
\vfill
\begin{center}
\small\sffamily\textcolor{apitype}{%
Generated for cuRAMSES-kjhan --- \today}
\end{center}

\end{document}

%   (when included, comment out the \documentclass .. \begin{document}
%    and \end{document} lines, or wrap them with \ifx\STANDALONE\undefined)
%%%%%%%%%%%%%%%%%%%%%%%%%%%%%%%%%%%%%%%%%%%%%%%%%%%%%%%%%%%%%%%%%%%%%%%%%

\documentclass[11pt]{article}

%----------------------------------------------------------------------
%  Packages
%----------------------------------------------------------------------
\usepackage[top=2.5cm,bottom=2.5cm,left=2.8cm,right=2.8cm,a4paper]{geometry}
\usepackage[utf8]{inputenc}
\usepackage[T1]{fontenc}
\usepackage{lmodern}
\usepackage{microtype}
\usepackage{xcolor}
\usepackage{hyperref}
\usepackage{enumitem}
\usepackage{amsmath,amssymb}
\usepackage{booktabs}
\usepackage{longtable}
\usepackage{fancyhdr}
\usepackage{titlesec}
\usepackage{parskip}
\usepackage{listings}

%----------------------------------------------------------------------
%  Colour palette (CAMB-readthedocs inspired)
%----------------------------------------------------------------------
\definecolor{apibg}{RGB}{248,249,251}       % light background
\definecolor{apiborder}{RGB}{200,210,225}    % border
\definecolor{apiname}{RGB}{0,64,128}         % parameter name
\definecolor{apitype}{RGB}{100,100,100}      % type annotation
\definecolor{apidefault}{RGB}{60,130,60}     % default value
\definecolor{apinote}{RGB}{180,80,40}        % warning / note
\definecolor{sectioncolor}{RGB}{0,80,160}    % section headings
\definecolor{codebg}{RGB}{243,245,248}
\definecolor{codeframe}{RGB}{190,200,215}

%----------------------------------------------------------------------
%  Hyperref
%----------------------------------------------------------------------
\hypersetup{
  colorlinks=true,
  linkcolor=sectioncolor,
  urlcolor=sectioncolor!80!black,
  citecolor=sectioncolor,
  pdfauthor={Juhan Kim},
  pdftitle={RAMSES Namelist Parameter Reference},
}

%----------------------------------------------------------------------
%  Section formatting
%----------------------------------------------------------------------
\titleformat{\section}
  {\Large\bfseries\sffamily\color{sectioncolor}}
  {\thesection}{1em}{}
  [\vspace{-0.6em}\textcolor{sectioncolor!40}{\rule{\linewidth}{0.8pt}}]

\titleformat{\subsection}
  {\large\bfseries\sffamily\color{sectioncolor!80!black}}
  {\thesubsection}{0.8em}{}

%----------------------------------------------------------------------
%  Header / footer
%----------------------------------------------------------------------
\pagestyle{fancy}
\fancyhf{}
\fancyhead[L]{\small\sffamily\textcolor{apitype}{RAMSES Namelist Reference}}
\fancyhead[R]{\small\sffamily\textcolor{apitype}{\nouppercase{\leftmark}}}
\fancyfoot[C]{\small\sffamily\thepage}
\renewcommand{\headrulewidth}{0.4pt}

%----------------------------------------------------------------------
%  Code listings
%----------------------------------------------------------------------
\lstset{
  basicstyle=\ttfamily\small,
  backgroundcolor=\color{codebg},
  frame=single,
  rulecolor=\color{codeframe},
  framesep=4pt,
  xleftmargin=6pt,
  xrightmargin=6pt,
  breaklines=true,
  columns=fullflexible,
  keepspaces=true,
  showstringspaces=false,
}

%----------------------------------------------------------------------
%  Custom environments for parameter documentation.
%
%  We use a "paramblock" environment instead of a macro so that
%  verbatim-like content (lstlisting) can appear inside parameter
%  descriptions.
%----------------------------------------------------------------------

% --- parameter header bar (name + type + default) ---
\newcommand{\paramheader}[3]{%
  \vspace{0.7em}%
  \noindent
  \colorbox{apibg}{%
    \parbox{\dimexpr\linewidth-2\fboxsep}{%
      \vspace{4pt}%
      \hspace{2pt}%
      {\large\textcolor{apiname}{\texttt{\textbf{#1}}}}%
      \hfill
      {\small\textcolor{apitype}{\textsf{#2}}}%
      \quad
      {\small\textcolor{apidefault}{\textsf{default:~\texttt{#3}}}}%
      \vspace{4pt}%
    }%
  }%
  \nopagebreak[4]\par\nopagebreak[4]\vspace{0.25em}%
}

% --- required parameter header bar (no default) ---
\newcommand{\paramheaderreq}[2]{%
  \vspace{0.7em}%
  \noindent
  \colorbox{apibg}{%
    \parbox{\dimexpr\linewidth-2\fboxsep}{%
      \vspace{4pt}%
      \hspace{2pt}%
      {\large\textcolor{apiname}{\texttt{\textbf{#1}}}}%
      \hfill
      {\small\textcolor{apitype}{\textsf{#2}}}%
      \quad
      {\small\textcolor{apinote}{\textsf{required}}}%
      \vspace{4pt}%
    }%
  }%
  \nopagebreak[4]\par\nopagebreak[4]\vspace{0.25em}%
}

%----------------------------------------------------------------------
%  Note / Warning / Example helpers (usable at top level)
%----------------------------------------------------------------------
\newcommand{\apinote}[1]{%
  \par\vspace{0.3em}%
  \noindent\textcolor{sectioncolor}{\textsf{\textbf{Note:}}} #1%
  \par\vspace{0.2em}%
}

\newcommand{\apiwarning}[1]{%
  \par\vspace{0.3em}%
  \noindent\textcolor{apinote}{\textsf{\textbf{Warning:}}} #1%
  \par\vspace{0.2em}%
}

%----------------------------------------------------------------------
%  Cross-reference helper
%  Usage: \paramref{cosmo}  or  \paramref{memory_balance}
%  The label name uses the raw form (with literal underscores);
%  the displayed text uses \texttt with escaped underscores.
%----------------------------------------------------------------------
\makeatletter
\newcommand{\paramref}[1]{%
  \begingroup
  \def\tmp{#1}%
  \hyperref[param:\tmp]{\texttt{\detokenize{#1}}}%
  \endgroup
}
\makeatother

%======================================================================
\begin{document}

%----------------------------------------------------------------------
%  Title
%----------------------------------------------------------------------
\begin{center}
{\fontsize{28}{34}\selectfont\sffamily\bfseries
  \textcolor{sectioncolor}{RAMSES Namelist Parameter Reference}}\\[0.8cm]
{\large\sffamily cuRAMSES-kjhan -- February 2026}\\[0.3cm]
{\normalsize\sffamily Juhan Kim}\\[0.2cm]
{\small\sffamily\textcolor{apitype}{%
  Based on RAMSES by Romain Teyssier}}
\end{center}

\vspace{0.5cm}
\noindent\textcolor{sectioncolor!40}{\rule{\linewidth}{1pt}}

\vspace{0.6cm}
\noindent
This document provides a complete reference for every namelist parameter
accepted by RAMSES and the cuRAMSES-kjhan extensions.  Parameters are
grouped by their Fortran namelist block (\texttt{\&RUN\_PARAMS},
\texttt{\&AMR\_PARAMS}, etc.).  Each entry specifies the parameter name,
Fortran type, default value, and a detailed description including valid
ranges and interactions with other parameters.

The namelist file uses standard Fortran namelist syntax.  Each block
begins with \texttt{\&BLOCK\_NAME} and ends with a single~\texttt{/}.

\vspace{0.4cm}
\tableofcontents
\newpage

%======================================================================
\section{\texttt{\&RUN\_PARAMS} -- Global Run Control}
\label{sec:run_params}
%======================================================================

This mandatory block controls the physics modules to activate, restart
behaviour, domain decomposition strategy, and general simulation
parameters.

% ------------------------------------------------------------------
\paramheader{cosmo}{logical}{.false.}
\label{param:cosmo}

\begin{quote}
Enable cosmological mode.  When \texttt{.true.}, RAMSES uses comoving
(super-comoving) coordinates with the expansion factor $a(t)$ as the
time variable.  The box length is interpreted in $h^{-1}\,$Mpc.
Friedmann equations are integrated internally.

Enabling this flag also activates expansion-factor--based output
scheduling (see \paramref{aout} in
Section~\ref{sec:output_params}).  Cosmological initial conditions
(GRAFIC2 format) must be provided via \paramref{initfile}.
\end{quote}

% ------------------------------------------------------------------
\paramheader{pic}{logical}{.false.}
\label{param:pic}

\begin{quote}
Enable the Particle-In-Cell (PIC) method for collisionless $N$-body
dynamics (dark matter, stars).  Particles are deposited onto the AMR
grid using cloud-in-cell (CIC) interpolation, and forces are
interpolated back to particle positions.

Required for any simulation containing dark matter particles.  Usually
combined with \paramref{poisson}\texttt{=.true.}
\end{quote}

% ------------------------------------------------------------------
\paramheader{poisson}{logical}{.false.}
\label{param:poisson}

\begin{quote}
Enable the self-gravity Poisson solver.  RAMSES uses an adaptive
multigrid (MG) method on the AMR hierarchy with V-cycles and
red--black Gauss--Seidel smoothing.  Convergence is controlled by
\paramref{epsilon} in \texttt{\&POISSON\_PARAMS}.

Must be \texttt{.true.}\ whenever \paramref{pic}\texttt{=.true.}\ or
whenever gas self-gravity is desired.
\end{quote}

% ------------------------------------------------------------------
\paramheader{hydro}{logical}{.false.}
\label{param:hydro}

\begin{quote}
Enable the hydrodynamics (or MHD) solver.  RAMSES employs a
second-order MUSCL--Hancock scheme with approximate Riemann solvers
(see \paramref{scheme}, \paramref{riemann} in
Section~\ref{sec:hydro_params}).

Set to \texttt{.true.}\ for any simulation involving baryonic gas.
\end{quote}

% ------------------------------------------------------------------
\paramheader{nrestart}{integer}{0}
\label{param:nrestart}

\begin{quote}
Restart from checkpoint (output snapshot) number \texttt{nrestart}.
\begin{itemize}[nosep]
  \item \texttt{nrestart=0} -- fresh start from initial conditions.
  \item \texttt{nrestart=}$N$ -- load \texttt{output\_}$N$\texttt{/} and resume.
\end{itemize}
The number of MPI processes must match the run that produced the
checkpoint.  RAMSES reads all AMR, hydro, particle, and gravity data
from the snapshot directory.
\end{quote}

% ------------------------------------------------------------------
\paramheader{nremap}{integer}{5}
\label{param:nremap}

\begin{quote}
Load-balancing frequency: perform domain decomposition every
\texttt{nremap} coarse time steps.  Recommended value: \textbf{5}
(balances redistribution overhead against growing load imbalance).
Set to \texttt{0} to disable load balancing entirely.

\apinote{Benchmarks (200\,M particles, 12~ranks, 10~steps) show that
\texttt{nremap=5} reduces total runtime by 18\%\ compared to
\texttt{nremap=1}, with load-balance overhead at 6.3\%\ of wall time.
Larger values (e.g.\ 10) save overhead but allow imbalance to grow.}
\end{quote}

% ------------------------------------------------------------------
\paramheader{nsubcycle}{integer array}{1,1,2}
\label{param:nsubcycle}

\begin{quote}
Time sub-cycling factors per AMR level.  The $i$-th entry gives the
number of fine time steps per coarse step at level
$\texttt{levelmin}+i-1$.  Typical usage:
\begin{itemize}[nosep]
  \item \texttt{1} for coarse levels (no sub-cycling).
  \item \texttt{2} for fine levels (halve the time step at each
        finer level).
\end{itemize}
The array has up to \texttt{levelmax\,--\,levelmin+1} entries.  Any
unspecified trailing entries default to~\texttt{1}.
\end{quote}

\noindent\textcolor{apidefault}{\textsf{\textbf{Example:}}}
\begin{lstlisting}
nsubcycle = 1, 1, 1, 2, 2, 2, 2
\end{lstlisting}

% ------------------------------------------------------------------
\paramheader{ncontrol}{integer}{1}
\label{param:ncontrol}

\begin{quote}
Print control output (energy diagnostics, timing) every
\texttt{ncontrol} coarse time steps to standard output.
\end{quote}

% ------------------------------------------------------------------
\paramheader{nstepmax}{integer}{1000000}
\label{param:nstepmax}

\begin{quote}
Maximum number of coarse time steps.  The simulation stops when
\texttt{nstep} reaches this value, even if the final output time has
not been reached.  Useful for short test runs.
\end{quote}

% ------------------------------------------------------------------
\paramheader{ordering}{character}{`hilbert'}
\label{param:ordering}

\begin{quote}
Domain decomposition ordering strategy.
\begin{description}[style=nextline,leftmargin=2.5cm,font=\ttfamily]
  \item[`hilbert'] Hilbert space-filling curve.  Standard choice for
    moderate core counts ($\lesssim 1000$).
  \item[`ksection'] K-section tree-based decomposition.  Provides
    $O(k)$~message scaling (where $k$ is the branching factor) for
    large core counts.  Enables hierarchical MPI exchanges and
    memory-based load balancing (see \paramref{memory_balance}).
\end{description}
When \texttt{ordering='ksection'}, the communication pattern in ghost
zone exchanges, multigrid solvers, and \texttt{build\_comm} all switch
to ksection tree routing automatically.
\end{quote}

% ------------------------------------------------------------------
\paramheader{memory\_balance}{logical}{.false.}
\label{param:memory_balance}

\begin{quote}
Enable memory-based load balancing.  When \texttt{.true.}, the
bisection histogram weights each cell by its memory footprint (grid
metadata + attached particles) instead of uniform cell count.

\textbf{Requires} \paramref{ordering}\texttt{='ksection'}.

The cell cost function is:
\[
  C_{\text{cell}} = \frac{\texttt{mem\_weight\_grid}}{\texttt{twotondim}}
                + n_{\text{part}} \times
                  \frac{\texttt{mem\_weight\_part}}{\texttt{twotondim}}
\]
where $n_{\text{part}}$ is the number of particles attached to the parent
grid.  The weight parameters \texttt{mem\_weight\_grid} (default~270) and
\texttt{mem\_weight\_part} (default~12) are set in the same namelist block.

\apinote{All histogram variables (\texttt{bisec\_hist},
\texttt{bisec\_cpu\_load}, \texttt{cell\_cost}) use 64-bit integers
(\texttt{integer(i8b)}) and \texttt{MPI\_INTEGER8} to avoid overflow at
high particle counts.}
\end{quote}

% ------------------------------------------------------------------
\paramheader{sink}{logical}{.false.}
\label{param:sink}

\begin{quote}
Enable sink particle formation and evolution.  Sink particles represent
compact objects (e.g.\ black holes, protostars) that accrete gas from
their surroundings.  When active, cells exceeding a density threshold
at \texttt{levelmax} can spawn sink particles.

See also \paramref{sink_AGN}, \paramref{bondi}, \paramref{Mseed} in
Section~\ref{sec:physics_params}.
\end{quote}

% ------------------------------------------------------------------
\paramheader{sinkprops}{logical}{.false.}
\label{param:sinkprops}

\begin{quote}
Output detailed sink particle properties (mass, position, velocity,
accretion rate, spin) to dedicated files at each snapshot.
\end{quote}

% ------------------------------------------------------------------
\paramheader{lightcone}{logical}{.false.}
\label{param:lightcone}

\begin{quote}
Enable lightcone output mode.  When \texttt{.true.}, particles and/or
cells crossing the observer's past lightcone are written to special
output files during the simulation.  See
Section~\ref{sec:lightcone_params} for additional parameters.
\end{quote}

% ------------------------------------------------------------------
\paramheader{verbose}{logical}{.false.}
\label{param:verbose}

\begin{quote}
Enable verbose output during initialization and evolution.  Prints
additional diagnostics (grid counts, memory usage, load balance
statistics) to standard output at each coarse step.
\end{quote}


%======================================================================
\section{\texttt{\&AMR\_PARAMS} -- Adaptive Mesh Refinement}
\label{sec:amr_params}
%======================================================================

This mandatory block controls the AMR grid hierarchy, memory allocation
sizes, and the simulation box geometry.

% ------------------------------------------------------------------
\paramheaderreq{levelmin}{integer}
\label{param:levelmin}

\begin{quote}
Minimum (base) AMR level.  The base grid has $2^{\texttt{levelmin}}$
cells per dimension.

\noindent\textcolor{apidefault}{\textsf{\textbf{Example:}}}
\texttt{levelmin=7} produces a $128^3$ base grid.
\texttt{levelmin=9} produces a $512^3$ base grid.

This level is fully covered -- every cell at \texttt{levelmin} exists
on exactly one MPI process.
\end{quote}

% ------------------------------------------------------------------
\paramheaderreq{levelmax}{integer}
\label{param:levelmax}

\begin{quote}
Maximum AMR level.  Determines the finest attainable resolution:
\[
  \Delta x_{\min} = \frac{L_{\text{box}}}{2^{\texttt{levelmax}}}
\]
For cosmological zoom-in simulations, this controls the physical
resolution at $z=0$.  The number of refinement levels beyond
\texttt{levelmin} is \texttt{levelmax\,--\,levelmin}.

Refinement criteria (\paramref{m_refine}, \paramref{ivar_refine})
determine which cells actually refine up to this level.
\end{quote}

% ------------------------------------------------------------------
\paramheader{nexpand}{integer array}{1}
\label{param:nexpand}

\begin{quote}
Number of buffer (guard) cell layers per level to ensure smooth
transitions between refinement levels.  The $i$-th entry applies to
level $\texttt{levelmin}+i-1$.  Typical value: \texttt{1} for all
levels.

Larger values produce wider buffer zones around refined patches,
improving solution quality at the cost of more cells.
\end{quote}

% ------------------------------------------------------------------
\paramheaderreq{ngridtot}{integer(i8b)}
\label{param:ngridtot}

\begin{quote}
Total number of AMR grids (octs) allocated across all MPI processes.
Each process receives $\texttt{ngridmax} = \texttt{ngridtot} /
\texttt{ncpu}$ grids.  Each grid (oct) contains $2^{\texttt{ndim}}$
cells (8~cells in 3D).

\apiwarning{RAMSES allocates full arrays at startup based on
\texttt{ngridmax}.  The virtual memory footprint is approximately
$\texttt{ngridmax} \times 20\;\text{bytes} \times \texttt{nvar}$.
This must not exceed the system's \texttt{CommitLimit} (typically
RAM~$\times$~\texttt{overcommit\_ratio}/100).}

\textbf{Rule of thumb:} For $N$~processes on a node with $M$\,GB of
RAM,
\[
  \texttt{ngridtot} < \frac{M \times 0.5}{20 \times \texttt{nvar}}
  \times N
\]
\end{quote}

% ------------------------------------------------------------------
\paramheaderreq{nparttot}{integer(i8b)}
\label{param:nparttot}

\begin{quote}
Total particle allocation across all MPI processes.  Each process gets
$\texttt{npartmax} = \texttt{nparttot} / \texttt{ncpu}$.  Should be
at least $2\times$ the total number of DM + star particles expected
during the simulation (to accommodate load imbalance and new star
particle creation).
\end{quote}

\noindent\textcolor{apidefault}{\textsf{\textbf{Example:}}}
\begin{lstlisting}
! 100M DM particles, allow for stars
nparttot = 300000000
\end{lstlisting}

% ------------------------------------------------------------------
\paramheader{boxlen}{real(dp)}{1.0}
\label{param:boxlen}

\begin{quote}
Box length in code units.  For cosmological runs, the box size is
typically read from the IC header (in $h^{-1}\,$Mpc) and this
parameter is overridden.  For non-cosmological (idealised) setups,
\texttt{boxlen} defines the physical domain extent.
\end{quote}


%======================================================================
\section{\texttt{\&OUTPUT\_PARAMS} -- Snapshot Output}
\label{sec:output_params}
%======================================================================

Controls when and how simulation snapshots are written to disk.

% ------------------------------------------------------------------
\paramheader{noutput}{integer}{1}
\label{param:noutput}

\begin{quote}
Number of output snapshots requested.  The corresponding times or
expansion factors must be listed in \paramref{tout} (non-cosmological)
or \paramref{aout} (cosmological).
\end{quote}

% ------------------------------------------------------------------
\paramheader{aout}{real(dp) array}{---}
\label{param:aout}

\begin{quote}
Scale factors at which to write output snapshots (cosmological mode
only, i.e.\ when \paramref{cosmo}\texttt{=.true.}).  The array must
contain \paramref{noutput} entries, in ascending order.
\end{quote}

\noindent\textcolor{apidefault}{\textsf{\textbf{Example:}}}
\begin{lstlisting}
noutput = 4
aout    = 0.1, 0.2, 0.5, 1.0
! Outputs at z = 9, 4, 1, 0
\end{lstlisting}

% ------------------------------------------------------------------
\paramheader{tout}{real(dp) array}{---}
\label{param:tout}

\begin{quote}
Output times in code units (non-cosmological mode).  The array must
contain \paramref{noutput} entries, in ascending order.
\end{quote}

% ------------------------------------------------------------------
\paramheader{foutput}{integer}{1000000}
\label{param:foutput}

\begin{quote}
Write an output snapshot every \texttt{foutput} coarse time steps,
regardless of the \paramref{aout}/\paramref{tout} schedule.  Useful
for periodic checkpointing in long runs.  Set to a very large number
to effectively disable.
\end{quote}

% ------------------------------------------------------------------
\paramheader{outformat}{character}{`original'}
\label{param:outformat}

\begin{quote}
Output file format for snapshots.
\begin{description}[style=nextline,leftmargin=2.5cm,font=\ttfamily]
  \item[`original'] Standard RAMSES per-CPU binary format.  Each MPI
    process writes separate files (\texttt{amr\_NNNNN.outNNNNN},
    \texttt{hydro\_NNNNN.outNNNNN}, etc.).
  \item[`hdf5'] Single HDF5 file per snapshot (\texttt{data\_NNNNN.h5}).
    Uses MPI parallel I/O for collective writes.  The HDF5 file stores
    all AMR, hydro, gravity, particle, and sink data in a hierarchical
    group structure.  \textbf{Requires compilation with}
    \texttt{make~HDF5=1}.
\end{description}

\apinote{The standard auxiliary files (\texttt{info\_NNNNN.txt},
\texttt{header\_NNNNN.txt}, \texttt{compilation.txt},
\texttt{makefile.txt}, \texttt{namelist.txt}) are always written
regardless of \texttt{outformat}.}
\end{quote}

% ------------------------------------------------------------------
\paramheader{informat}{character}{`original'}
\label{param:informat}

\begin{quote}
Input (restart) file format.
\begin{description}[style=nextline,leftmargin=2.5cm,font=\ttfamily]
  \item[`original'] Read from standard per-CPU binary files.  The
    number of MPI processes must match the run that produced the
    checkpoint.
  \item[`hdf5'] Read from the single HDF5 file
    (\texttt{data\_NNNNN.h5}).  Currently requires the same number of
    MPI processes as the original run.  \textbf{Requires compilation
    with} \texttt{make~HDF5=1}.
\end{description}

\apinote{\texttt{informat} and \texttt{outformat} can be set
independently, allowing cross-format conversion (e.g.\ restart from
binary and output to HDF5, or vice versa).}
\end{quote}


%======================================================================
\section{\texttt{\&INIT\_PARAMS} -- Initial Conditions}
\label{sec:init_params}
%======================================================================

Specifies the format and location of initial condition files.

% ------------------------------------------------------------------
\paramheader{filetype}{character}{`grafic'}
\label{param:filetype}

\begin{quote}
Initial condition file format.
\begin{description}[style=nextline,leftmargin=2.5cm,font=\ttfamily]
  \item[`grafic'] GRAFIC2 binary format (Bertschinger 2001).  Each
    level's IC directory contains binary files for density
    perturbations, velocities, and (optionally) particle
    displacements.
  \item[`ascii'] Text-based initial conditions (for simple test
    problems).
\end{description}
\end{quote}

% ------------------------------------------------------------------
\paramheader{initfile}{character array}{---}
\label{param:initfile}

\begin{quote}
Paths to IC directories, one per AMR level.
\texttt{initfile(1)} corresponds to \paramref{levelmin},
\texttt{initfile(2)} to $\texttt{levelmin}+1$, and so on.

Each directory must contain the following binary files:
\begin{itemize}[nosep]
  \item \texttt{ic\_deltab} -- baryon density perturbation field
  \item \texttt{ic\_velbx}, \texttt{ic\_velby}, \texttt{ic\_velbz}
        -- baryon velocity fields
  \item \texttt{ic\_velcx}, \texttt{ic\_velcy}, \texttt{ic\_velcz}
        -- dark matter (CDM) velocity fields
  \item \texttt{ic\_poscx}, \texttt{ic\_poscy}, \texttt{ic\_poscz}
        -- dark matter displacement fields (optional, for multi-level
        zoom-in)
  \item \texttt{ic\_tempb} -- baryon temperature perturbation
        (optional)
  \item \texttt{ic\_pvar\_00001}, \ldots\ -- passive scalar fields
        (optional, for zoom-in refinement tagging;
        see \paramref{ivar_refine})
  \item \texttt{ic\_refmap} -- refinement map (optional)
\end{itemize}
\end{quote}

\noindent\textcolor{apidefault}{\textsf{\textbf{Example:}}}
\begin{lstlisting}
initfile = '/data/IC/level_07'
         , '/data/IC/level_08'
         , '/data/IC/level_09'
\end{lstlisting}


%======================================================================
\section{\texttt{\&REFINE\_PARAMS} -- Refinement Criteria}
\label{sec:refine_params}
%======================================================================

Controls which cells are refined in the AMR hierarchy.  These
parameters are \textbf{critical for zoom-in simulations}, where
background regions must remain coarse while the zoom region refines
to high resolution.

% ------------------------------------------------------------------
\paramheader{m\_refine}{real(dp) array}{$-1$}
\label{param:m_refine}

\begin{quote}
Quasi-Lagrangian mass threshold per level.  The $i$-th entry applies
to level $\texttt{levelmin}+i-1$.  A cell is flagged for refinement
when the effective mass indicator $\phi \geq \texttt{m\_refine}(i)$.

Typical value: \textbf{8.0} for all levels (refine when the
equivalent of $\geq 8$ particles occupies a cell).  Provide one
entry for each level from \texttt{levelmin} to \texttt{levelmax}.

Interacts with \paramref{ivar_refine} and
\paramref{mass_cut_refine} to determine which particles contribute
to the density used for the refinement decision.
\end{quote}

\noindent\textcolor{apidefault}{\textsf{\textbf{Example:}}}
\begin{lstlisting}
! 6 levels of refinement (levelmin=7, levelmax=13)
m_refine = 8., 8., 8., 8., 8., 8., 8.
\end{lstlisting}

% ------------------------------------------------------------------
\paramheader{ivar\_refine}{integer}{$-1$}
\label{param:ivar_refine}

\begin{quote}
Variable index controlling the refinement criterion in
\texttt{poisson\_refine}.  This parameter fundamentally changes how
refinement regions are selected:
\begin{description}[style=nextline,leftmargin=1em]
  \item[\texttt{ivar\_refine = 0}:]
    Use \texttt{cpu\_map2} for refinement control.  During
    initialization, \texttt{cpu\_map2} is set by
    \texttt{init\_refmap} from \texttt{ic\_refmap} (if present);
    during evolution, it is updated by \texttt{rho\_fine} based on
    the local density field.  This is the standard quasi-Lagrangian
    approach.

    \apiwarning{In zoom-in simulations, this can cause uncontrolled
    AMR expansion into background regions if \texttt{cpu\_map2} is
    not properly restricted by \paramref{mass_cut_refine}.}

  \item[\texttt{ivar\_refine > 0} (e.g.\ 11):]
    During initialization, use passive scalar criterion:
    \texttt{uold(cell, ivar\_refine) / uold(cell, 1)} $>$
    \paramref{var_cut_refine}.

    \textbf{Recommended for zoom-in:} set
    \texttt{ivar\_refine=NVAR} (the last hydro variable), and create
    \texttt{ic\_pvar\_NNNNN} files with value 1.0 inside the zoom
    region and 0.0 in the background.  After initialization,
    \texttt{cpu\_map2} (set by \texttt{rho\_fine} with
    \paramref{mass_cut_refine} filtering) takes over.

  \item[\texttt{ivar\_refine < 0} (default):]
    Pure density-based refinement at both initialization and runtime.
    A cell is refined when
    \texttt{uold(cell, 1)} $\geq$ \texttt{m\_refine}
    $\times\; m_{\text{sph}} / V_{\text{cell}}$.
\end{description}
\end{quote}

% ------------------------------------------------------------------
\paramheader{var\_cut\_refine}{real(dp)}{$-1$}
\label{param:var_cut_refine}

\begin{quote}
Threshold for passive-scalar-based refinement when
\paramref{ivar_refine} $> 0$.  A cell is refined only if
\[
  \frac{\texttt{uold}(\text{cell},\;\texttt{ivar\_refine})}%
       {\texttt{uold}(\text{cell},\;1)}
  > \texttt{var\_cut\_refine}
\]
Typical value: \textbf{0.01} for zoom geometry tagging (the passive
scalar is 1.0 inside the zoom region, 0.0 outside).
\end{quote}

% ------------------------------------------------------------------
\paramheader{mass\_cut\_refine}{real(dp)}{$-1$}
\label{param:mass_cut_refine}

\begin{quote}
Particle mass threshold for quasi-Lagrangian refinement.  In
\texttt{rho\_fine}, dark matter particles with mass
$\geq \texttt{mass\_cut\_refine}$ are \emph{excluded} from the
density computation that drives cell refinement.  This prevents
heavy (coarse-level) background particles from triggering spurious
refinement.

Set this to the DM particle mass at the finest IC level.  Reference
values for a $100\,h^{-1}\,$Mpc box
($\Omega_m = 0.3$, $h = 0.68$):

\begin{center}
\small
\begin{tabular}{@{}ll@{}}
\toprule
\textbf{IC finest level} & \textbf{mass\_cut\_refine} \\
\midrule
8  & \texttt{1.19209e-07} \\
9  & \texttt{1.49012e-08} \\
10 & \texttt{1.86265e-09} \\
11 & \texttt{2.32831e-10} \\
12 & \texttt{2.91038e-11} \\
13 & \texttt{3.63798e-12} \\
\bottomrule
\end{tabular}
\end{center}

\apinote{This parameter interacts with \paramref{ivar_refine} and
\paramref{m_refine}.  All three should be set consistently for
zoom-in simulations.}
\end{quote}

% ------------------------------------------------------------------
\paramheader{interpol\_var}{integer}{0}
\label{param:interpol_var}

\begin{quote}
Interpolation variable type used when prolongating (interpolating)
data from coarse to fine grids.
\begin{description}[style=nextline,leftmargin=1.5cm,font=\ttfamily]
  \item[0] Conservative variables ($\rho$, $\rho v$, $E$).
  \item[1] Primitive variables ($\rho$, $v$, $P$).
        \textbf{Recommended} for cosmological simulations to avoid
        interpolation artefacts in low-density regions.
\end{description}
\end{quote}

% ------------------------------------------------------------------
\paramheader{interpol\_type}{integer}{1}
\label{param:interpol_type}

\begin{quote}
Interpolation slope limiter for prolongation.
\begin{description}[style=nextline,leftmargin=1.5cm,font=\ttfamily]
  \item[0] MinMod limiter -- more diffusive, more robust.
  \item[1] MonCen (monotonised central) limiter -- less diffusive.
        \textbf{Recommended}.
\end{description}
\end{quote}

% ------------------------------------------------------------------
\paramheader{sink\_refine}{logical}{.false.}
\label{param:sink_refine}

\begin{quote}
Force maximum refinement around sink particles.  When \texttt{.true.},
a contribution equal to \paramref{m_refine} is added to the
refinement indicator $\phi$ for every cell containing a sink particle,
ensuring refinement up to \paramref{levelmax}.
\end{quote}

% ------------------------------------------------------------------
\paramheader{jeans\_ncells}{real(dp)}{$-1$}
\label{param:jeans_ncells}

\begin{quote}
Jeans refinement criterion.  If $> 0$, cells are refined to resolve
the local Jeans length by at least this many cells:
\[
  \Delta x < \frac{\lambda_J}{\texttt{jeans\_ncells}}
\]
Enabling this also activates a polytropic equation-of-state floor to
prevent artificial fragmentation (Truelove criterion).  Typical
value: \textbf{4} (minimum of 4 cells per Jeans length).
\end{quote}


%======================================================================
\section{\texttt{\&HYDRO\_PARAMS} -- Hydrodynamics Solver}
\label{sec:hydro_params}
%======================================================================

Controls the gas dynamics solver configuration.

% ------------------------------------------------------------------
\paramheader{gamma}{real(dp)}{$5/3$}
\label{param:gamma}

\begin{quote}
Adiabatic index $\gamma$ of the ideal gas equation of state,
$P = (\gamma - 1) \rho e$.
Standard value: $5/3$ for a monatomic ideal gas.  Use $7/5$ for
diatomic gas or $4/3$ for radiation-dominated flow.
\end{quote}

% ------------------------------------------------------------------
\paramheader{courant\_factor}{real(dp)}{0.8}
\label{param:courant_factor}

\begin{quote}
Courant--Friedrichs--Lewy (CFL) number for time step control.  The
time step at each level is
$\Delta t = \texttt{courant\_factor} \times \Delta x / v_{\max}$.
Typical: \textbf{0.8}.  Lower values increase stability at the cost
of more time steps.
\end{quote}

% ------------------------------------------------------------------
\paramheader{scheme}{character}{`muscl'}
\label{param:scheme}

\begin{quote}
Hydrodynamics integration scheme.
\begin{description}[style=nextline,leftmargin=2.5cm,font=\ttfamily]
  \item[`muscl'] MUSCL--Hancock (Monotonic Upstream-centred Scheme
    for Conservation Laws), second-order in space and time.  This is
    the only production scheme in RAMSES.
\end{description}
\end{quote}

% ------------------------------------------------------------------
\paramheader{slope\_type}{integer}{1}
\label{param:slope_type}

\begin{quote}
Slope limiter for MUSCL piecewise-linear reconstruction.
\begin{description}[style=nextline,leftmargin=1.5cm,font=\ttfamily]
  \item[1] MinMod -- most robust, more diffusive.
  \item[2] MonCen -- monotonised central, less diffusive.
        \textbf{Recommended} for production runs.
  \item[3] Unlimited -- no limiting (unstable; testing only).
\end{description}
\end{quote}

% ------------------------------------------------------------------
\paramheader{riemann}{character}{`llf'}
\label{param:riemann}

\begin{quote}
Approximate Riemann solver for inter-cell flux computation.
\begin{description}[style=nextline,leftmargin=2.5cm,font=\ttfamily]
  \item[`llf'] Local Lax--Friedrichs (Rusanov).  Most diffusive but
    unconditionally stable.  Good default.
  \item[`hll'] Harten--Lax--van Leer.  Two-wave solver.
  \item[`hllc'] HLL with Contact restoration.  Three-wave solver,
    most accurate for contact discontinuities.
    \textbf{Recommended for cosmological simulations}.
  \item[`exact'] Exact Riemann solver (expensive; primarily for
    validation).
\end{description}
\end{quote}

% ------------------------------------------------------------------
\paramheader{pressure\_fix}{logical}{.false.}
\label{param:pressure_fix}

\begin{quote}
Enable pressure floor to prevent negative pressures in strong shocks
or highly supersonic flows.  When the internal energy becomes negative,
RAMSES falls back to a pressure estimate from the total energy.

\textbf{Recommended:} \texttt{.true.}\ for cosmological simulations.
See also \paramref{beta_fix}.
\end{quote}

% ------------------------------------------------------------------
\paramheader{beta\_fix}{real(dp)}{0.0}
\label{param:beta_fix}

\begin{quote}
Pressure fix parameter.  Controls the magnitude of the pressure floor:
$P_{\text{floor}} = \texttt{beta\_fix} \times \rho v^2 / 2$.
Typical value: \textbf{0.5} when
\paramref{pressure_fix}\texttt{=.true.}
\end{quote}

% ------------------------------------------------------------------
\paramheader{isothermal}{logical}{.false.}
\label{param:isothermal}

\begin{quote}
Isothermal mode.  When \texttt{.true.}, the energy equation is not
solved and the gas temperature remains constant.  Reduces the number
of hydro variables by one.
\end{quote}


%======================================================================
\section{\texttt{\&POISSON\_PARAMS} -- Gravity Solver}
\label{sec:poisson_params}
%======================================================================

Controls the multigrid Poisson solver for self-gravity.

% ------------------------------------------------------------------
\paramheader{epsilon}{real(dp)}{$10^{-4}$}
\label{param:epsilon}

\begin{quote}
Multigrid convergence criterion.  The V-cycle iteration at each level
stops when the residual norm satisfies
$\|r\| / \|r_0\| < \texttt{epsilon}$.
Typical value for cosmological runs: $10^{-5}$ to $10^{-4}$.
Tighter values improve force accuracy but increase iteration count.
\end{quote}

% ------------------------------------------------------------------
\paramheader{gravity\_type}{integer}{0}
\label{param:gravity_type}

\begin{quote}
Gravity model selection.
\begin{description}[style=nextline,leftmargin=1.5cm,font=\ttfamily]
  \item[0] Self-gravity (solve Poisson equation on the AMR grid).
  \item[>0] Analytical gravitational potential (e.g.\ for test
    problems with known solutions).  The integer value selects the
    specific analytical profile.
\end{description}
\end{quote}

% ------------------------------------------------------------------
\paramheader{cg\_levelmin}{integer}{999}
\label{param:cg_levelmin}

\begin{quote}
Minimum level at which the conjugate gradient (CG) fallback solver
activates.  When the multigrid solver stalls at high AMR levels, CG
provides guaranteed convergence.  Set to \paramref{levelmax} for
best convergence behaviour.

Typical: \texttt{cg\_levelmin = levelmax}.  The default (999) means
CG is effectively disabled unless \texttt{levelmax} is absurdly
large.
\end{quote}

% ------------------------------------------------------------------
\paramheader{cic\_levelmax}{integer}{0}
\label{param:cic_levelmax}

\begin{quote}
Maximum level for cloud-in-cell (CIC) particle mass deposition.
\begin{itemize}[nosep]
  \item \texttt{0} -- deposit particles at all levels (standard).
  \item $N > 0$ -- deposit particles only up to level $N$; finer
    levels inherit the coarse density by prolongation.
\end{itemize}
Rarely modified.
\end{quote}


%======================================================================
\section{\texttt{\&PHYSICS\_PARAMS} -- Sub-grid Physics}
\label{sec:physics_params}
%======================================================================

Controls cooling, star formation, stellar/AGN feedback, and
cosmological parameters.  This block is optional; omit it entirely
for adiabatic (non-radiative) simulations.

%----------------------------------------------------------------------
\subsection{Cooling and UV Background}
%----------------------------------------------------------------------

\paramheader{cooling}{logical}{.false.}
\label{param:cooling}

\begin{quote}
Enable radiative cooling with a metal-dependent cooling function.
When \texttt{.true.}, RAMSES integrates the cooling/heating rate
at each time step using tabulated cooling curves.  Requires
\paramref{hydro}\texttt{=.true.}
\end{quote}

% ------------------------------------------------------------------
\paramheader{haardt\_madau}{logical}{.false.}
\label{param:haardt_madau}

\begin{quote}
Enable the Haardt \& Madau (2012) ultraviolet background model for
cosmic reionization.  Provides a redshift-dependent photo-heating
and photo-ionization rate.  Used together with \paramref{cooling}.
\end{quote}

% ------------------------------------------------------------------
\paramheader{z\_reion}{real(dp)}{8.5}
\label{param:z_reion}

\begin{quote}
Reionization redshift.  Hydrogen reionization heating is applied
instantaneously at this redshift.  Typical range: 6--10, depending
on the reionization model.
\end{quote}

% ------------------------------------------------------------------
\paramheader{z\_ave}{real(dp)}{0.0}
\label{param:z_ave}

\begin{quote}
Initial mean metallicity of the gas in solar units ($Z_\odot$).
Applied uniformly at initialization.  Use \texttt{0.0} for
primordial composition.
\end{quote}

% ------------------------------------------------------------------
\paramheader{delayed\_cooling}{logical}{.false.}
\label{param:delayed_cooling}

\begin{quote}
Delay radiative cooling in supernova-heated gas to prevent
overcooling.  When a cell receives SN energy, cooling is suppressed
for a duration related to the Sedov--Taylor phase.  Improves the
effectiveness of stellar feedback in regulating star formation.
\end{quote}

% ------------------------------------------------------------------
\paramheader{tol}{real(dp)}{$10^{-3}$}
\label{param:tol}

\begin{quote}
Tolerance for the implicit cooling solver.  The Newton--Raphson
iteration converges when the relative temperature change
$|\Delta T / T| < \texttt{tol}$.
\end{quote}

%----------------------------------------------------------------------
\subsection{Star Formation}
%----------------------------------------------------------------------

\paramheader{n\_star}{real(dp)}{0.1}
\label{param:n_star}

\begin{quote}
Star formation hydrogen number density threshold in
$\text{H}\;\text{cm}^{-3}$.  Only gas denser than this value is
eligible for star formation.  Typical range: 0.1--10.
\end{quote}

% ------------------------------------------------------------------
\paramheader{eps\_star}{real(dp)}{0.0}
\label{param:eps_star}

\begin{quote}
Star formation efficiency per free-fall time $\epsilon_{\text{ff}}$.
The star formation rate density is
$\dot{\rho}_\star = \epsilon_{\text{ff}} \, \rho_{\text{gas}} /
t_{\text{ff}}$.
Typical value: \textbf{0.01--0.02} (1--2\%\ per free-fall time).
Set to \texttt{0.0} to disable star formation entirely.
\end{quote}

% ------------------------------------------------------------------
\paramheader{del\_star}{real(dp)}{200}
\label{param:del_star}

\begin{quote}
Overdensity threshold for star formation (in units of the cosmic
mean density).  Gas must exceed $\delta > \texttt{del\_star}$ in
addition to the density threshold \paramref{n_star}.
\end{quote}

% ------------------------------------------------------------------
\paramheader{m\_star}{real(dp)}{$-1$}
\label{param:m_star}

\begin{quote}
Minimum stellar particle mass in code units.  When a star-forming
cell would produce a particle below this mass, the event is
stochastically deferred to the next time step.
\begin{itemize}[nosep]
  \item $< 0$: use the cell gas mass (no minimum).
  \item $> 0$: explicit minimum mass.
\end{itemize}
\end{quote}

% ------------------------------------------------------------------
\paramheader{T2\_star}{real(dp)}{0}
\label{param:T2_star}

\begin{quote}
ISM polytropic equation-of-state temperature floor in Kelvin.  Gas
above the star-formation density threshold \paramref{n_star} follows
a polytropic relation:
\[
  T = T_{2,\star} \left(\frac{n}{n_\star}\right)^{\gamma_\star - 1}
\]
where $\gamma_\star$ is \paramref{g_star}.  This prevents artificial
fragmentation below the resolution limit (Jeans mass floor).
\end{quote}

% ------------------------------------------------------------------
\paramheader{g\_star}{real(dp)}{1.6}
\label{param:g_star}

\begin{quote}
Polytropic index $\gamma_\star$ for the ISM equation of state (see
\paramref{T2_star}).  Typical value: \textbf{1.6} (stiff polytrope)
or \textbf{5/3} (adiabatic floor).
\end{quote}

%----------------------------------------------------------------------
\subsection{Stellar Feedback}
%----------------------------------------------------------------------

\paramheader{f\_w}{real(dp)}{0}
\label{param:f_w}

\begin{quote}
Mass loading factor for supernova-driven winds.  The wind mass flux
is $\dot{M}_w = \texttt{f\_w} \times \dot{M}_\star$.  Set to
\texttt{0} to disable winds.  Typical range: 1--5.
\end{quote}

% ------------------------------------------------------------------
\paramheader{f\_ek}{real(dp)}{1.0}
\label{param:f_ek}

\begin{quote}
Kinetic energy fraction of supernova feedback.  Controls the
partition between kinetic (\texttt{f\_ek}) and thermal
($1 - \texttt{f\_ek}$) energy injection.  \texttt{f\_ek=1} is
purely kinetic feedback; \texttt{f\_ek=0} is purely thermal.
\end{quote}

% ------------------------------------------------------------------
\paramheader{eps\_sn1}{real(dp)}{0}
\label{param:eps_sn1}

\begin{quote}
Type Ia supernova energy per event in units of $10^{51}\,$erg.
Set to \texttt{0} to disable Type Ia SN feedback.
\end{quote}

% ------------------------------------------------------------------
\paramheader{eps\_sn2}{real(dp)}{0}
\label{param:eps_sn2}

\begin{quote}
Type II supernova energy per event in units of $10^{51}\,$erg.
Set to \texttt{0} to disable Type II SN feedback.
\end{quote}

% ------------------------------------------------------------------
\paramheader{yieldtablefilename}{character}{---}
\label{param:yieldtablefilename}

\begin{quote}
Path to the chemical yield table file for metal enrichment
calculations.  Required when metal-dependent cooling or chemical
evolution tracking is enabled.
\end{quote}

%----------------------------------------------------------------------
\subsection{Cosmological Parameters}
%----------------------------------------------------------------------

\paramheader{omega\_b}{real(dp)}{---}
\label{param:omega_b}

\begin{quote}
Baryon density parameter $\Omega_b$.  Overrides the value read from
the IC file header.  Must be consistent with the initial conditions
and other cosmological parameters ($\Omega_m$, $H_0$, etc.\ are
read from the GRAFIC2 header).
\end{quote}

%----------------------------------------------------------------------
\subsection{AGN and Sink Particle Parameters}
%----------------------------------------------------------------------

\paramheader{Mseed}{real(dp)}{---}
\label{param:Mseed}

\begin{quote}
Seed black hole mass in solar masses ($M_\odot$).  When a sink
particle forms, it is initialised with this mass.  Typical range:
$10^4$--$10^6\,M_\odot$ for cosmological simulations.
\end{quote}

% ------------------------------------------------------------------
\paramheader{sink\_AGN}{logical}{.false.}
\label{param:sink_AGN}

\begin{quote}
Enable AGN feedback from sink particles.  When \texttt{.true.}, sink
particles inject thermal and/or kinetic energy into their surroundings
based on their accretion rate.  Requires
\paramref{sink}\texttt{=.true.}
\end{quote}

% ------------------------------------------------------------------
\paramheader{bondi}{logical}{.false.}
\label{param:bondi}

\begin{quote}
Enable Bondi--Hoyle--Lyttleton accretion for sink particles.  The
accretion rate is computed from the local gas density, sound speed,
and relative velocity:
\[
  \dot{M} = \frac{4\pi G^2 M_{\text{BH}}^2 \rho}%
             {(c_s^2 + v_{\text{rel}}^2)^{3/2}}
\]
Can be boosted by \paramref{boost_acc}.
\end{quote}

% ------------------------------------------------------------------
\paramheader{drag}{logical}{.false.}
\label{param:drag}

\begin{quote}
Enable dynamical friction on sink particles.  Applies a drag force
opposing the sink's motion relative to the background gas.  Strength
can be amplified by \paramref{boost_drag}.
\end{quote}

% ------------------------------------------------------------------
\paramheader{rAGN}{real(dp)}{---}
\label{param:rAGN}

\begin{quote}
AGN feedback energy injection radius in units of the cell size at
\paramref{levelmax}.  Feedback energy is distributed over a sphere
of this radius centred on the sink particle.
\end{quote}

% ------------------------------------------------------------------
\paramheader{X\_floor}{real(dp)}{---}
\label{param:X_floor}

\begin{quote}
Hydrogen mass fraction floor.  Prevents the hydrogen fraction from
dropping below this value due to numerical artefacts.  Typical:
\texttt{0.76}.
\end{quote}

% ------------------------------------------------------------------
\paramheader{eAGN\_K}{real(dp)}{---}
\label{param:eAGN_K}

\begin{quote}
AGN kinetic feedback efficiency $\epsilon_K$.  Fraction of the
accreted rest-mass energy deposited as kinetic energy:
$\dot{E}_K = \epsilon_K \, \dot{M} c^2$.
\end{quote}

% ------------------------------------------------------------------
\paramheader{eAGN\_T}{real(dp)}{---}
\label{param:eAGN_T}

\begin{quote}
AGN thermal feedback efficiency $\epsilon_T$.  Fraction of the
accreted rest-mass energy deposited as thermal energy:
$\dot{E}_T = \epsilon_T \, \dot{M} c^2$.
\end{quote}

% ------------------------------------------------------------------
\paramheader{TAGN}{real(dp)}{---}
\label{param:TAGN}

\begin{quote}
AGN heating temperature in Kelvin.  The AGN thermal energy is
deposited by raising gas temperature toward this value within the
feedback region \paramref{rAGN}.
\end{quote}

% ------------------------------------------------------------------
\paramheader{r\_gal}{real(dp)}{---}
\label{param:r_gal}

\begin{quote}
Galaxy definition radius for AGN feedback, in code units.  Used to
compute the local galaxy properties (stellar mass, gas mass) around
a sink particle for AGN mode switching.
\end{quote}

% ------------------------------------------------------------------
\paramheader{T2maxAGN}{real(dp)}{---}
\label{param:T2maxAGN}

\begin{quote}
Maximum AGN heating temperature in Kelvin.  Caps the temperature
increase from a single AGN feedback event to prevent unphysically
hot gas.
\end{quote}

% ------------------------------------------------------------------
\paramheader{boost\_acc}{real(dp)}{---}
\label{param:boost_acc}

\begin{quote}
Bondi accretion boost factor.  Multiplies the Bondi--Hoyle accretion
rate by this factor to compensate for unresolved gas structure near
the black hole.  Typical range: 1--100.  Requires
\paramref{bondi}\texttt{=.true.}
\end{quote}

% ------------------------------------------------------------------
\paramheader{boost\_drag}{real(dp)}{---}
\label{param:boost_drag}

\begin{quote}
Dynamical friction drag boost factor.  Multiplies the drag force by
this factor.  Requires \paramref{drag}\texttt{=.true.}
\end{quote}

% ------------------------------------------------------------------
\paramheader{vrel\_merge}{logical}{---}
\label{param:vrel_merge}

\begin{quote}
Use relative velocity criterion for sink particle merging.  When
\texttt{.true.}, two sinks merge only if their relative velocity is
below the local escape velocity, in addition to the spatial proximity
criterion \paramref{rmerge}.
\end{quote}

% ------------------------------------------------------------------
\paramheader{rmerge}{real(dp)}{---}
\label{param:rmerge}

\begin{quote}
Sink merging radius in units of the cell size at
\paramref{levelmax}.  Two sink particles closer than this distance
are candidates for merging (subject to additional criteria if
\paramref{vrel_merge}\texttt{=.true.}).
\end{quote}

% ------------------------------------------------------------------
\paramheader{spin\_bh}{logical}{---}
\label{param:spin_bh}

\begin{quote}
Track black hole spin evolution.  When \texttt{.true.}, the code
evolves the dimensionless spin parameter $a_\star$ of each sink
particle based on the angular momentum of accreted gas.
\end{quote}

% ------------------------------------------------------------------
\paramheader{mad\_jet}{logical}{---}
\label{param:mad_jet}

\begin{quote}
Enable the magnetically arrested disk (MAD) jet model.  When
\texttt{.true.}, AGN kinetic feedback is launched as a collimated
bipolar jet aligned with the black hole spin axis.  Requires
\paramref{spin_bh}\texttt{=.true.}
\end{quote}


%======================================================================
\section{\texttt{\&LIGHTCONE\_PARAMS} -- Lightcone Output}
\label{sec:lightcone_params}
%======================================================================

\begin{quote}
Parameters for lightcone output mode (activated when
\paramref{lightcone}\texttt{=.true.}).  In this mode, particles
and/or cells that cross the observer's past lightcone during each
time step are written to special output files, enabling the
construction of mock galaxy surveys and weak-lensing maps without
storing full snapshots.

Configuration parameters include the observer position, opening
angle, and selection criteria.  Consult the RAMSES lightcone
documentation for the full parameter list, which varies by
application.
\end{quote}


%======================================================================
\section{\texttt{\&SPHERICAL\_REGION\_PARAMS}}
\label{sec:spherical_region_params}
%======================================================================

\paramheader{spherical\_region}{logical}{.false.}
\label{param:spherical_region}

\begin{quote}
Enable a spherical refinement region.  When \texttt{.true.}, AMR
refinement is restricted to a spherical sub-volume of the simulation
box.  This is useful for re-simulations of specific halos where a
cubic zoom region is not optimal.  Additional parameters define the
centre and radius of the sphere.
\end{quote}


%======================================================================
\section{Complete Example: Cosmological Zoom-In}
\label{sec:example}
%======================================================================

The following namelist illustrates a production cosmological zoom-in
simulation with dark matter particles, baryonic gas, self-gravity,
cooling, star formation, and AGN feedback.

\begin{lstlisting}[language=Fortran,
  caption={Cosmological zoom-in namelist},
  label={lst:zoomin}]
&RUN_PARAMS
cosmo    = .true.
pic      = .true.
poisson  = .true.
hydro    = .true.
sink     = .true.
nrestart = 0
nremap   = 5
nsubcycle = 1, 1, 1, 2, 2, 2, 2
nstepmax = 10000000
ordering = 'ksection'
memory_balance = .true.
/

&AMR_PARAMS
levelmin  = 7
levelmax  = 18
nexpand   = 1, 1, 1, 1, 1, 1, 1, 1, 1, 1, 1, 1
ngridtot  = 400000000
nparttot  = 600000000
/

&OUTPUT_PARAMS
noutput = 10
aout = 0.05, 0.1, 0.15, 0.2, 0.3, 0.4, 0.5, 0.7, 0.85, 1.0
foutput = 500
/

&INIT_PARAMS
filetype = 'grafic'
initfile = '/data/IC/level_07'
         , '/data/IC/level_08'
         , '/data/IC/level_09'
         , '/data/IC/level_10'
         , '/data/IC/level_11'
         , '/data/IC/level_12'
         , '/data/IC/level_13'
/

&REFINE_PARAMS
m_refine = 8., 8., 8., 8., 8., 8., 8., 8., 8., 8., 8., 8.
ivar_refine     = 11
var_cut_refine  = 0.01
mass_cut_refine = 3.63798e-12
interpol_var    = 1
interpol_type   = 1
/

&HYDRO_PARAMS
gamma         = 1.6666667
courant_factor = 0.8
scheme        = 'muscl'
slope_type    = 2
riemann       = 'hllc'
pressure_fix  = .true.
beta_fix      = 0.5
/

&POISSON_PARAMS
epsilon      = 1.0e-5
gravity_type = 0
cg_levelmin  = 18
/

&PHYSICS_PARAMS
cooling        = .true.
haardt_madau   = .true.
z_reion        = 8.5
n_star         = 0.1
eps_star       = 0.02
T2_star        = 1.0e4
g_star         = 1.6
del_star       = 200.0
f_ek           = 1.0
sink_AGN       = .true.
bondi          = .true.
Mseed          = 1.0e5
/
\end{lstlisting}


%======================================================================
\section{Parameter Cross-Reference Index}
\label{sec:crossref}
%======================================================================

Table~\ref{tab:crossref} lists parameters that commonly interact and
should be set consistently.

\begin{longtable}{@{}p{4cm}p{4.5cm}p{6.5cm}@{}}
\caption{Cross-reference of interacting parameters.}
\label{tab:crossref}\\
\toprule
\textbf{Parameter} & \textbf{Related Parameters} & \textbf{Notes} \\
\midrule
\endfirsthead
\toprule
\textbf{Parameter} & \textbf{Related Parameters} & \textbf{Notes} \\
\midrule
\endhead
\bottomrule
\endfoot
\texttt{cosmo}
  & \texttt{aout}, \texttt{omega\_b}
  & Cosmological mode requires scale-factor outputs \\
\addlinespace
\texttt{pic}
  & \texttt{poisson}, \texttt{nparttot}
  & Particles need gravity and memory allocation \\
\addlinespace
\texttt{ordering}
  & \texttt{memory\_balance}
  & Memory balancing requires \texttt{ksection} \\
\addlinespace
\texttt{levelmax}
  & \texttt{m\_refine}, \texttt{cg\_levelmin}
  & Set \texttt{cg\_levelmin = levelmax} \\
\addlinespace
\texttt{ivar\_refine}
  & \texttt{var\_cut\_refine}, \texttt{mass\_cut\_refine}, \texttt{m\_refine}
  & All must be consistent for zoom-in \\
\addlinespace
\texttt{mass\_cut\_refine}
  & \texttt{ivar\_refine}
  & Set to finest-level DM particle mass \\
\addlinespace
\texttt{pressure\_fix}
  & \texttt{beta\_fix}
  & \texttt{beta\_fix} only effective when fix is on \\
\addlinespace
\texttt{T2\_star}
  & \texttt{g\_star}, \texttt{n\_star}
  & Polytropic EOS parameters \\
\addlinespace
\texttt{sink}
  & \texttt{sink\_AGN}, \texttt{bondi}, \texttt{Mseed}
  & AGN feedback requires sink particles \\
\addlinespace
\texttt{sink\_AGN}
  & \texttt{eAGN\_K}, \texttt{eAGN\_T}, \texttt{TAGN}, \texttt{rAGN}
  & AGN feedback parameters \\
\addlinespace
\texttt{bondi}
  & \texttt{boost\_acc}
  & Boost factor for unresolved accretion \\
\addlinespace
\texttt{drag}
  & \texttt{boost\_drag}
  & Drag boost factor \\
\addlinespace
\texttt{spin\_bh}
  & \texttt{mad\_jet}
  & MAD jet requires spin tracking \\
\addlinespace
\texttt{cooling}
  & \texttt{haardt\_madau}, \texttt{z\_reion}
  & UV background for reionization heating \\
\addlinespace
\texttt{ngridtot}
  & \texttt{nparttot}
  & Both determine per-process memory usage \\
\end{longtable}


%======================================================================
\vfill
\begin{center}
\small\sffamily\textcolor{apitype}{%
Generated for cuRAMSES-kjhan --- \today}
\end{center}

\end{document}

%   (when included, comment out the \documentclass .. \begin{document}
%    and \end{document} lines, or wrap them with \ifx\STANDALONE\undefined)
%%%%%%%%%%%%%%%%%%%%%%%%%%%%%%%%%%%%%%%%%%%%%%%%%%%%%%%%%%%%%%%%%%%%%%%%%

\documentclass[11pt]{article}

%----------------------------------------------------------------------
%  Packages
%----------------------------------------------------------------------
\usepackage[top=2.5cm,bottom=2.5cm,left=2.8cm,right=2.8cm,a4paper]{geometry}
\usepackage[utf8]{inputenc}
\usepackage[T1]{fontenc}
\usepackage{lmodern}
\usepackage{microtype}
\usepackage{xcolor}
\usepackage{hyperref}
\usepackage{enumitem}
\usepackage{amsmath,amssymb}
\usepackage{booktabs}
\usepackage{longtable}
\usepackage{fancyhdr}
\usepackage{titlesec}
\usepackage{parskip}
\usepackage{listings}

%----------------------------------------------------------------------
%  Colour palette (CAMB-readthedocs inspired)
%----------------------------------------------------------------------
\definecolor{apibg}{RGB}{248,249,251}       % light background
\definecolor{apiborder}{RGB}{200,210,225}    % border
\definecolor{apiname}{RGB}{0,64,128}         % parameter name
\definecolor{apitype}{RGB}{100,100,100}      % type annotation
\definecolor{apidefault}{RGB}{60,130,60}     % default value
\definecolor{apinote}{RGB}{180,80,40}        % warning / note
\definecolor{sectioncolor}{RGB}{0,80,160}    % section headings
\definecolor{codebg}{RGB}{243,245,248}
\definecolor{codeframe}{RGB}{190,200,215}

%----------------------------------------------------------------------
%  Hyperref
%----------------------------------------------------------------------
\hypersetup{
  colorlinks=true,
  linkcolor=sectioncolor,
  urlcolor=sectioncolor!80!black,
  citecolor=sectioncolor,
  pdfauthor={Juhan Kim},
  pdftitle={RAMSES Namelist Parameter Reference},
}

%----------------------------------------------------------------------
%  Section formatting
%----------------------------------------------------------------------
\titleformat{\section}
  {\Large\bfseries\sffamily\color{sectioncolor}}
  {\thesection}{1em}{}
  [\vspace{-0.6em}\textcolor{sectioncolor!40}{\rule{\linewidth}{0.8pt}}]

\titleformat{\subsection}
  {\large\bfseries\sffamily\color{sectioncolor!80!black}}
  {\thesubsection}{0.8em}{}

%----------------------------------------------------------------------
%  Header / footer
%----------------------------------------------------------------------
\pagestyle{fancy}
\fancyhf{}
\fancyhead[L]{\small\sffamily\textcolor{apitype}{RAMSES Namelist Reference}}
\fancyhead[R]{\small\sffamily\textcolor{apitype}{\nouppercase{\leftmark}}}
\fancyfoot[C]{\small\sffamily\thepage}
\renewcommand{\headrulewidth}{0.4pt}

%----------------------------------------------------------------------
%  Code listings
%----------------------------------------------------------------------
\lstset{
  basicstyle=\ttfamily\small,
  backgroundcolor=\color{codebg},
  frame=single,
  rulecolor=\color{codeframe},
  framesep=4pt,
  xleftmargin=6pt,
  xrightmargin=6pt,
  breaklines=true,
  columns=fullflexible,
  keepspaces=true,
  showstringspaces=false,
}

%----------------------------------------------------------------------
%  Custom environments for parameter documentation.
%
%  We use a "paramblock" environment instead of a macro so that
%  verbatim-like content (lstlisting) can appear inside parameter
%  descriptions.
%----------------------------------------------------------------------

% --- parameter header bar (name + type + default) ---
\newcommand{\paramheader}[3]{%
  \vspace{0.7em}%
  \noindent
  \colorbox{apibg}{%
    \parbox{\dimexpr\linewidth-2\fboxsep}{%
      \vspace{4pt}%
      \hspace{2pt}%
      {\large\textcolor{apiname}{\texttt{\textbf{#1}}}}%
      \hfill
      {\small\textcolor{apitype}{\textsf{#2}}}%
      \quad
      {\small\textcolor{apidefault}{\textsf{default:~\texttt{#3}}}}%
      \vspace{4pt}%
    }%
  }%
  \nopagebreak[4]\par\nopagebreak[4]\vspace{0.25em}%
}

% --- required parameter header bar (no default) ---
\newcommand{\paramheaderreq}[2]{%
  \vspace{0.7em}%
  \noindent
  \colorbox{apibg}{%
    \parbox{\dimexpr\linewidth-2\fboxsep}{%
      \vspace{4pt}%
      \hspace{2pt}%
      {\large\textcolor{apiname}{\texttt{\textbf{#1}}}}%
      \hfill
      {\small\textcolor{apitype}{\textsf{#2}}}%
      \quad
      {\small\textcolor{apinote}{\textsf{required}}}%
      \vspace{4pt}%
    }%
  }%
  \nopagebreak[4]\par\nopagebreak[4]\vspace{0.25em}%
}

%----------------------------------------------------------------------
%  Note / Warning / Example helpers (usable at top level)
%----------------------------------------------------------------------
\newcommand{\apinote}[1]{%
  \par\vspace{0.3em}%
  \noindent\textcolor{sectioncolor}{\textsf{\textbf{Note:}}} #1%
  \par\vspace{0.2em}%
}

\newcommand{\apiwarning}[1]{%
  \par\vspace{0.3em}%
  \noindent\textcolor{apinote}{\textsf{\textbf{Warning:}}} #1%
  \par\vspace{0.2em}%
}

%----------------------------------------------------------------------
%  Cross-reference helper
%  Usage: \paramref{cosmo}  or  \paramref{memory_balance}
%  The label name uses the raw form (with literal underscores);
%  the displayed text uses \texttt with escaped underscores.
%----------------------------------------------------------------------
\makeatletter
\newcommand{\paramref}[1]{%
  \begingroup
  \def\tmp{#1}%
  \hyperref[param:\tmp]{\texttt{\detokenize{#1}}}%
  \endgroup
}
\makeatother

%======================================================================
\begin{document}

%----------------------------------------------------------------------
%  Title
%----------------------------------------------------------------------
\begin{center}
{\fontsize{28}{34}\selectfont\sffamily\bfseries
  \textcolor{sectioncolor}{RAMSES Namelist Parameter Reference}}\\[0.8cm]
{\large\sffamily cuRAMSES-kjhan -- February 2026}\\[0.3cm]
{\normalsize\sffamily Juhan Kim}\\[0.2cm]
{\small\sffamily\textcolor{apitype}{%
  Based on RAMSES by Romain Teyssier}}
\end{center}

\vspace{0.5cm}
\noindent\textcolor{sectioncolor!40}{\rule{\linewidth}{1pt}}

\vspace{0.6cm}
\noindent
This document provides a complete reference for every namelist parameter
accepted by RAMSES and the cuRAMSES-kjhan extensions.  Parameters are
grouped by their Fortran namelist block (\texttt{\&RUN\_PARAMS},
\texttt{\&AMR\_PARAMS}, etc.).  Each entry specifies the parameter name,
Fortran type, default value, and a detailed description including valid
ranges and interactions with other parameters.

The namelist file uses standard Fortran namelist syntax.  Each block
begins with \texttt{\&BLOCK\_NAME} and ends with a single~\texttt{/}.

\vspace{0.4cm}
\tableofcontents
\newpage

%======================================================================
\section{\texttt{\&RUN\_PARAMS} -- Global Run Control}
\label{sec:run_params}
%======================================================================

This mandatory block controls the physics modules to activate, restart
behaviour, domain decomposition strategy, and general simulation
parameters.

% ------------------------------------------------------------------
\paramheader{cosmo}{logical}{.false.}
\label{param:cosmo}

\begin{quote}
Enable cosmological mode.  When \texttt{.true.}, RAMSES uses comoving
(super-comoving) coordinates with the expansion factor $a(t)$ as the
time variable.  The box length is interpreted in $h^{-1}\,$Mpc.
Friedmann equations are integrated internally.

Enabling this flag also activates expansion-factor--based output
scheduling (see \paramref{aout} in
Section~\ref{sec:output_params}).  Cosmological initial conditions
(GRAFIC2 format) must be provided via \paramref{initfile}.
\end{quote}

% ------------------------------------------------------------------
\paramheader{pic}{logical}{.false.}
\label{param:pic}

\begin{quote}
Enable the Particle-In-Cell (PIC) method for collisionless $N$-body
dynamics (dark matter, stars).  Particles are deposited onto the AMR
grid using cloud-in-cell (CIC) interpolation, and forces are
interpolated back to particle positions.

Required for any simulation containing dark matter particles.  Usually
combined with \paramref{poisson}\texttt{=.true.}
\end{quote}

% ------------------------------------------------------------------
\paramheader{poisson}{logical}{.false.}
\label{param:poisson}

\begin{quote}
Enable the self-gravity Poisson solver.  RAMSES uses an adaptive
multigrid (MG) method on the AMR hierarchy with V-cycles and
red--black Gauss--Seidel smoothing.  Convergence is controlled by
\paramref{epsilon} in \texttt{\&POISSON\_PARAMS}.

Must be \texttt{.true.}\ whenever \paramref{pic}\texttt{=.true.}\ or
whenever gas self-gravity is desired.
\end{quote}

% ------------------------------------------------------------------
\paramheader{hydro}{logical}{.false.}
\label{param:hydro}

\begin{quote}
Enable the hydrodynamics (or MHD) solver.  RAMSES employs a
second-order MUSCL--Hancock scheme with approximate Riemann solvers
(see \paramref{scheme}, \paramref{riemann} in
Section~\ref{sec:hydro_params}).

Set to \texttt{.true.}\ for any simulation involving baryonic gas.
\end{quote}

% ------------------------------------------------------------------
\paramheader{nrestart}{integer}{0}
\label{param:nrestart}

\begin{quote}
Restart from checkpoint (output snapshot) number \texttt{nrestart}.
\begin{itemize}[nosep]
  \item \texttt{nrestart=0} -- fresh start from initial conditions.
  \item \texttt{nrestart=}$N$ -- load \texttt{output\_}$N$\texttt{/} and resume.
\end{itemize}
The number of MPI processes must match the run that produced the
checkpoint.  RAMSES reads all AMR, hydro, particle, and gravity data
from the snapshot directory.
\end{quote}

% ------------------------------------------------------------------
\paramheader{nremap}{integer}{5}
\label{param:nremap}

\begin{quote}
Load-balancing frequency: perform domain decomposition every
\texttt{nremap} coarse time steps.  Recommended value: \textbf{5}
(balances redistribution overhead against growing load imbalance).
Set to \texttt{0} to disable load balancing entirely.

\apinote{Benchmarks (200\,M particles, 12~ranks, 10~steps) show that
\texttt{nremap=5} reduces total runtime by 18\%\ compared to
\texttt{nremap=1}, with load-balance overhead at 6.3\%\ of wall time.
Larger values (e.g.\ 10) save overhead but allow imbalance to grow.}
\end{quote}

% ------------------------------------------------------------------
\paramheader{nsubcycle}{integer array}{1,1,2}
\label{param:nsubcycle}

\begin{quote}
Time sub-cycling factors per AMR level.  The $i$-th entry gives the
number of fine time steps per coarse step at level
$\texttt{levelmin}+i-1$.  Typical usage:
\begin{itemize}[nosep]
  \item \texttt{1} for coarse levels (no sub-cycling).
  \item \texttt{2} for fine levels (halve the time step at each
        finer level).
\end{itemize}
The array has up to \texttt{levelmax\,--\,levelmin+1} entries.  Any
unspecified trailing entries default to~\texttt{1}.
\end{quote}

\noindent\textcolor{apidefault}{\textsf{\textbf{Example:}}}
\begin{lstlisting}
nsubcycle = 1, 1, 1, 2, 2, 2, 2
\end{lstlisting}

% ------------------------------------------------------------------
\paramheader{ncontrol}{integer}{1}
\label{param:ncontrol}

\begin{quote}
Print control output (energy diagnostics, timing) every
\texttt{ncontrol} coarse time steps to standard output.
\end{quote}

% ------------------------------------------------------------------
\paramheader{nstepmax}{integer}{1000000}
\label{param:nstepmax}

\begin{quote}
Maximum number of coarse time steps.  The simulation stops when
\texttt{nstep} reaches this value, even if the final output time has
not been reached.  Useful for short test runs.
\end{quote}

% ------------------------------------------------------------------
\paramheader{ordering}{character}{`hilbert'}
\label{param:ordering}

\begin{quote}
Domain decomposition ordering strategy.
\begin{description}[style=nextline,leftmargin=2.5cm,font=\ttfamily]
  \item[`hilbert'] Hilbert space-filling curve.  Standard choice for
    moderate core counts ($\lesssim 1000$).
  \item[`ksection'] K-section tree-based decomposition.  Provides
    $O(k)$~message scaling (where $k$ is the branching factor) for
    large core counts.  Enables hierarchical MPI exchanges and
    memory-based load balancing (see \paramref{memory_balance}).
\end{description}
When \texttt{ordering='ksection'}, the communication pattern in ghost
zone exchanges, multigrid solvers, and \texttt{build\_comm} all switch
to ksection tree routing automatically.
\end{quote}

% ------------------------------------------------------------------
\paramheader{memory\_balance}{logical}{.false.}
\label{param:memory_balance}

\begin{quote}
Enable memory-based load balancing.  When \texttt{.true.}, the
bisection histogram weights each cell by its memory footprint (grid
metadata + attached particles) instead of uniform cell count.

\textbf{Requires} \paramref{ordering}\texttt{='ksection'}.

The cell cost function is:
\[
  C_{\text{cell}} = \frac{\texttt{mem\_weight\_grid}}{\texttt{twotondim}}
                + n_{\text{part}} \times
                  \frac{\texttt{mem\_weight\_part}}{\texttt{twotondim}}
\]
where $n_{\text{part}}$ is the number of particles attached to the parent
grid.  The weight parameters \texttt{mem\_weight\_grid} (default~270) and
\texttt{mem\_weight\_part} (default~12) are set in the same namelist block.

\apinote{All histogram variables (\texttt{bisec\_hist},
\texttt{bisec\_cpu\_load}, \texttt{cell\_cost}) use 64-bit integers
(\texttt{integer(i8b)}) and \texttt{MPI\_INTEGER8} to avoid overflow at
high particle counts.}
\end{quote}

% ------------------------------------------------------------------
\paramheader{sink}{logical}{.false.}
\label{param:sink}

\begin{quote}
Enable sink particle formation and evolution.  Sink particles represent
compact objects (e.g.\ black holes, protostars) that accrete gas from
their surroundings.  When active, cells exceeding a density threshold
at \texttt{levelmax} can spawn sink particles.

See also \paramref{sink_AGN}, \paramref{bondi}, \paramref{Mseed} in
Section~\ref{sec:physics_params}.
\end{quote}

% ------------------------------------------------------------------
\paramheader{sinkprops}{logical}{.false.}
\label{param:sinkprops}

\begin{quote}
Output detailed sink particle properties (mass, position, velocity,
accretion rate, spin) to dedicated files at each snapshot.
\end{quote}

% ------------------------------------------------------------------
\paramheader{lightcone}{logical}{.false.}
\label{param:lightcone}

\begin{quote}
Enable lightcone output mode.  When \texttt{.true.}, particles and/or
cells crossing the observer's past lightcone are written to special
output files during the simulation.  See
Section~\ref{sec:lightcone_params} for additional parameters.
\end{quote}

% ------------------------------------------------------------------
\paramheader{verbose}{logical}{.false.}
\label{param:verbose}

\begin{quote}
Enable verbose output during initialization and evolution.  Prints
additional diagnostics (grid counts, memory usage, load balance
statistics) to standard output at each coarse step.
\end{quote}


%======================================================================
\section{\texttt{\&AMR\_PARAMS} -- Adaptive Mesh Refinement}
\label{sec:amr_params}
%======================================================================

This mandatory block controls the AMR grid hierarchy, memory allocation
sizes, and the simulation box geometry.

% ------------------------------------------------------------------
\paramheaderreq{levelmin}{integer}
\label{param:levelmin}

\begin{quote}
Minimum (base) AMR level.  The base grid has $2^{\texttt{levelmin}}$
cells per dimension.

\noindent\textcolor{apidefault}{\textsf{\textbf{Example:}}}
\texttt{levelmin=7} produces a $128^3$ base grid.
\texttt{levelmin=9} produces a $512^3$ base grid.

This level is fully covered -- every cell at \texttt{levelmin} exists
on exactly one MPI process.
\end{quote}

% ------------------------------------------------------------------
\paramheaderreq{levelmax}{integer}
\label{param:levelmax}

\begin{quote}
Maximum AMR level.  Determines the finest attainable resolution:
\[
  \Delta x_{\min} = \frac{L_{\text{box}}}{2^{\texttt{levelmax}}}
\]
For cosmological zoom-in simulations, this controls the physical
resolution at $z=0$.  The number of refinement levels beyond
\texttt{levelmin} is \texttt{levelmax\,--\,levelmin}.

Refinement criteria (\paramref{m_refine}, \paramref{ivar_refine})
determine which cells actually refine up to this level.
\end{quote}

% ------------------------------------------------------------------
\paramheader{nexpand}{integer array}{1}
\label{param:nexpand}

\begin{quote}
Number of buffer (guard) cell layers per level to ensure smooth
transitions between refinement levels.  The $i$-th entry applies to
level $\texttt{levelmin}+i-1$.  Typical value: \texttt{1} for all
levels.

Larger values produce wider buffer zones around refined patches,
improving solution quality at the cost of more cells.
\end{quote}

% ------------------------------------------------------------------
\paramheaderreq{ngridtot}{integer(i8b)}
\label{param:ngridtot}

\begin{quote}
Total number of AMR grids (octs) allocated across all MPI processes.
Each process receives $\texttt{ngridmax} = \texttt{ngridtot} /
\texttt{ncpu}$ grids.  Each grid (oct) contains $2^{\texttt{ndim}}$
cells (8~cells in 3D).

\apiwarning{RAMSES allocates full arrays at startup based on
\texttt{ngridmax}.  The virtual memory footprint is approximately
$\texttt{ngridmax} \times 20\;\text{bytes} \times \texttt{nvar}$.
This must not exceed the system's \texttt{CommitLimit} (typically
RAM~$\times$~\texttt{overcommit\_ratio}/100).}

\textbf{Rule of thumb:} For $N$~processes on a node with $M$\,GB of
RAM,
\[
  \texttt{ngridtot} < \frac{M \times 0.5}{20 \times \texttt{nvar}}
  \times N
\]
\end{quote}

% ------------------------------------------------------------------
\paramheaderreq{nparttot}{integer(i8b)}
\label{param:nparttot}

\begin{quote}
Total particle allocation across all MPI processes.  Each process gets
$\texttt{npartmax} = \texttt{nparttot} / \texttt{ncpu}$.  Should be
at least $2\times$ the total number of DM + star particles expected
during the simulation (to accommodate load imbalance and new star
particle creation).
\end{quote}

\noindent\textcolor{apidefault}{\textsf{\textbf{Example:}}}
\begin{lstlisting}
! 100M DM particles, allow for stars
nparttot = 300000000
\end{lstlisting}

% ------------------------------------------------------------------
\paramheader{boxlen}{real(dp)}{1.0}
\label{param:boxlen}

\begin{quote}
Box length in code units.  For cosmological runs, the box size is
typically read from the IC header (in $h^{-1}\,$Mpc) and this
parameter is overridden.  For non-cosmological (idealised) setups,
\texttt{boxlen} defines the physical domain extent.
\end{quote}


%======================================================================
\section{\texttt{\&OUTPUT\_PARAMS} -- Snapshot Output}
\label{sec:output_params}
%======================================================================

Controls when and how simulation snapshots are written to disk.

% ------------------------------------------------------------------
\paramheader{noutput}{integer}{1}
\label{param:noutput}

\begin{quote}
Number of output snapshots requested.  The corresponding times or
expansion factors must be listed in \paramref{tout} (non-cosmological)
or \paramref{aout} (cosmological).
\end{quote}

% ------------------------------------------------------------------
\paramheader{aout}{real(dp) array}{---}
\label{param:aout}

\begin{quote}
Scale factors at which to write output snapshots (cosmological mode
only, i.e.\ when \paramref{cosmo}\texttt{=.true.}).  The array must
contain \paramref{noutput} entries, in ascending order.
\end{quote}

\noindent\textcolor{apidefault}{\textsf{\textbf{Example:}}}
\begin{lstlisting}
noutput = 4
aout    = 0.1, 0.2, 0.5, 1.0
! Outputs at z = 9, 4, 1, 0
\end{lstlisting}

% ------------------------------------------------------------------
\paramheader{tout}{real(dp) array}{---}
\label{param:tout}

\begin{quote}
Output times in code units (non-cosmological mode).  The array must
contain \paramref{noutput} entries, in ascending order.
\end{quote}

% ------------------------------------------------------------------
\paramheader{foutput}{integer}{1000000}
\label{param:foutput}

\begin{quote}
Write an output snapshot every \texttt{foutput} coarse time steps,
regardless of the \paramref{aout}/\paramref{tout} schedule.  Useful
for periodic checkpointing in long runs.  Set to a very large number
to effectively disable.
\end{quote}

% ------------------------------------------------------------------
\paramheader{outformat}{character}{`original'}
\label{param:outformat}

\begin{quote}
Output file format for snapshots.
\begin{description}[style=nextline,leftmargin=2.5cm,font=\ttfamily]
  \item[`original'] Standard RAMSES per-CPU binary format.  Each MPI
    process writes separate files (\texttt{amr\_NNNNN.outNNNNN},
    \texttt{hydro\_NNNNN.outNNNNN}, etc.).
  \item[`hdf5'] Single HDF5 file per snapshot (\texttt{data\_NNNNN.h5}).
    Uses MPI parallel I/O for collective writes.  The HDF5 file stores
    all AMR, hydro, gravity, particle, and sink data in a hierarchical
    group structure.  \textbf{Requires compilation with}
    \texttt{make~HDF5=1}.
\end{description}

\apinote{The standard auxiliary files (\texttt{info\_NNNNN.txt},
\texttt{header\_NNNNN.txt}, \texttt{compilation.txt},
\texttt{makefile.txt}, \texttt{namelist.txt}) are always written
regardless of \texttt{outformat}.}
\end{quote}

% ------------------------------------------------------------------
\paramheader{informat}{character}{`original'}
\label{param:informat}

\begin{quote}
Input (restart) file format.
\begin{description}[style=nextline,leftmargin=2.5cm,font=\ttfamily]
  \item[`original'] Read from standard per-CPU binary files.  The
    number of MPI processes must match the run that produced the
    checkpoint.
  \item[`hdf5'] Read from the single HDF5 file
    (\texttt{data\_NNNNN.h5}).  Currently requires the same number of
    MPI processes as the original run.  \textbf{Requires compilation
    with} \texttt{make~HDF5=1}.
\end{description}

\apinote{\texttt{informat} and \texttt{outformat} can be set
independently, allowing cross-format conversion (e.g.\ restart from
binary and output to HDF5, or vice versa).}
\end{quote}


%======================================================================
\section{\texttt{\&INIT\_PARAMS} -- Initial Conditions}
\label{sec:init_params}
%======================================================================

Specifies the format and location of initial condition files.

% ------------------------------------------------------------------
\paramheader{filetype}{character}{`grafic'}
\label{param:filetype}

\begin{quote}
Initial condition file format.
\begin{description}[style=nextline,leftmargin=2.5cm,font=\ttfamily]
  \item[`grafic'] GRAFIC2 binary format (Bertschinger 2001).  Each
    level's IC directory contains binary files for density
    perturbations, velocities, and (optionally) particle
    displacements.
  \item[`ascii'] Text-based initial conditions (for simple test
    problems).
\end{description}
\end{quote}

% ------------------------------------------------------------------
\paramheader{initfile}{character array}{---}
\label{param:initfile}

\begin{quote}
Paths to IC directories, one per AMR level.
\texttt{initfile(1)} corresponds to \paramref{levelmin},
\texttt{initfile(2)} to $\texttt{levelmin}+1$, and so on.

Each directory must contain the following binary files:
\begin{itemize}[nosep]
  \item \texttt{ic\_deltab} -- baryon density perturbation field
  \item \texttt{ic\_velbx}, \texttt{ic\_velby}, \texttt{ic\_velbz}
        -- baryon velocity fields
  \item \texttt{ic\_velcx}, \texttt{ic\_velcy}, \texttt{ic\_velcz}
        -- dark matter (CDM) velocity fields
  \item \texttt{ic\_poscx}, \texttt{ic\_poscy}, \texttt{ic\_poscz}
        -- dark matter displacement fields (optional, for multi-level
        zoom-in)
  \item \texttt{ic\_tempb} -- baryon temperature perturbation
        (optional)
  \item \texttt{ic\_pvar\_00001}, \ldots\ -- passive scalar fields
        (optional, for zoom-in refinement tagging;
        see \paramref{ivar_refine})
  \item \texttt{ic\_refmap} -- refinement map (optional)
\end{itemize}
\end{quote}

\noindent\textcolor{apidefault}{\textsf{\textbf{Example:}}}
\begin{lstlisting}
initfile = '/data/IC/level_07'
         , '/data/IC/level_08'
         , '/data/IC/level_09'
\end{lstlisting}


%======================================================================
\section{\texttt{\&REFINE\_PARAMS} -- Refinement Criteria}
\label{sec:refine_params}
%======================================================================

Controls which cells are refined in the AMR hierarchy.  These
parameters are \textbf{critical for zoom-in simulations}, where
background regions must remain coarse while the zoom region refines
to high resolution.

% ------------------------------------------------------------------
\paramheader{m\_refine}{real(dp) array}{$-1$}
\label{param:m_refine}

\begin{quote}
Quasi-Lagrangian mass threshold per level.  The $i$-th entry applies
to level $\texttt{levelmin}+i-1$.  A cell is flagged for refinement
when the effective mass indicator $\phi \geq \texttt{m\_refine}(i)$.

Typical value: \textbf{8.0} for all levels (refine when the
equivalent of $\geq 8$ particles occupies a cell).  Provide one
entry for each level from \texttt{levelmin} to \texttt{levelmax}.

Interacts with \paramref{ivar_refine} and
\paramref{mass_cut_refine} to determine which particles contribute
to the density used for the refinement decision.
\end{quote}

\noindent\textcolor{apidefault}{\textsf{\textbf{Example:}}}
\begin{lstlisting}
! 6 levels of refinement (levelmin=7, levelmax=13)
m_refine = 8., 8., 8., 8., 8., 8., 8.
\end{lstlisting}

% ------------------------------------------------------------------
\paramheader{ivar\_refine}{integer}{$-1$}
\label{param:ivar_refine}

\begin{quote}
Variable index controlling the refinement criterion in
\texttt{poisson\_refine}.  This parameter fundamentally changes how
refinement regions are selected:
\begin{description}[style=nextline,leftmargin=1em]
  \item[\texttt{ivar\_refine = 0}:]
    Use \texttt{cpu\_map2} for refinement control.  During
    initialization, \texttt{cpu\_map2} is set by
    \texttt{init\_refmap} from \texttt{ic\_refmap} (if present);
    during evolution, it is updated by \texttt{rho\_fine} based on
    the local density field.  This is the standard quasi-Lagrangian
    approach.

    \apiwarning{In zoom-in simulations, this can cause uncontrolled
    AMR expansion into background regions if \texttt{cpu\_map2} is
    not properly restricted by \paramref{mass_cut_refine}.}

  \item[\texttt{ivar\_refine > 0} (e.g.\ 11):]
    During initialization, use passive scalar criterion:
    \texttt{uold(cell, ivar\_refine) / uold(cell, 1)} $>$
    \paramref{var_cut_refine}.

    \textbf{Recommended for zoom-in:} set
    \texttt{ivar\_refine=NVAR} (the last hydro variable), and create
    \texttt{ic\_pvar\_NNNNN} files with value 1.0 inside the zoom
    region and 0.0 in the background.  After initialization,
    \texttt{cpu\_map2} (set by \texttt{rho\_fine} with
    \paramref{mass_cut_refine} filtering) takes over.

  \item[\texttt{ivar\_refine < 0} (default):]
    Pure density-based refinement at both initialization and runtime.
    A cell is refined when
    \texttt{uold(cell, 1)} $\geq$ \texttt{m\_refine}
    $\times\; m_{\text{sph}} / V_{\text{cell}}$.
\end{description}
\end{quote}

% ------------------------------------------------------------------
\paramheader{var\_cut\_refine}{real(dp)}{$-1$}
\label{param:var_cut_refine}

\begin{quote}
Threshold for passive-scalar-based refinement when
\paramref{ivar_refine} $> 0$.  A cell is refined only if
\[
  \frac{\texttt{uold}(\text{cell},\;\texttt{ivar\_refine})}%
       {\texttt{uold}(\text{cell},\;1)}
  > \texttt{var\_cut\_refine}
\]
Typical value: \textbf{0.01} for zoom geometry tagging (the passive
scalar is 1.0 inside the zoom region, 0.0 outside).
\end{quote}

% ------------------------------------------------------------------
\paramheader{mass\_cut\_refine}{real(dp)}{$-1$}
\label{param:mass_cut_refine}

\begin{quote}
Particle mass threshold for quasi-Lagrangian refinement.  In
\texttt{rho\_fine}, dark matter particles with mass
$\geq \texttt{mass\_cut\_refine}$ are \emph{excluded} from the
density computation that drives cell refinement.  This prevents
heavy (coarse-level) background particles from triggering spurious
refinement.

Set this to the DM particle mass at the finest IC level.  Reference
values for a $100\,h^{-1}\,$Mpc box
($\Omega_m = 0.3$, $h = 0.68$):

\begin{center}
\small
\begin{tabular}{@{}ll@{}}
\toprule
\textbf{IC finest level} & \textbf{mass\_cut\_refine} \\
\midrule
8  & \texttt{1.19209e-07} \\
9  & \texttt{1.49012e-08} \\
10 & \texttt{1.86265e-09} \\
11 & \texttt{2.32831e-10} \\
12 & \texttt{2.91038e-11} \\
13 & \texttt{3.63798e-12} \\
\bottomrule
\end{tabular}
\end{center}

\apinote{This parameter interacts with \paramref{ivar_refine} and
\paramref{m_refine}.  All three should be set consistently for
zoom-in simulations.}
\end{quote}

% ------------------------------------------------------------------
\paramheader{interpol\_var}{integer}{0}
\label{param:interpol_var}

\begin{quote}
Interpolation variable type used when prolongating (interpolating)
data from coarse to fine grids.
\begin{description}[style=nextline,leftmargin=1.5cm,font=\ttfamily]
  \item[0] Conservative variables ($\rho$, $\rho v$, $E$).
  \item[1] Primitive variables ($\rho$, $v$, $P$).
        \textbf{Recommended} for cosmological simulations to avoid
        interpolation artefacts in low-density regions.
\end{description}
\end{quote}

% ------------------------------------------------------------------
\paramheader{interpol\_type}{integer}{1}
\label{param:interpol_type}

\begin{quote}
Interpolation slope limiter for prolongation.
\begin{description}[style=nextline,leftmargin=1.5cm,font=\ttfamily]
  \item[0] MinMod limiter -- more diffusive, more robust.
  \item[1] MonCen (monotonised central) limiter -- less diffusive.
        \textbf{Recommended}.
\end{description}
\end{quote}

% ------------------------------------------------------------------
\paramheader{sink\_refine}{logical}{.false.}
\label{param:sink_refine}

\begin{quote}
Force maximum refinement around sink particles.  When \texttt{.true.},
a contribution equal to \paramref{m_refine} is added to the
refinement indicator $\phi$ for every cell containing a sink particle,
ensuring refinement up to \paramref{levelmax}.
\end{quote}

% ------------------------------------------------------------------
\paramheader{jeans\_ncells}{real(dp)}{$-1$}
\label{param:jeans_ncells}

\begin{quote}
Jeans refinement criterion.  If $> 0$, cells are refined to resolve
the local Jeans length by at least this many cells:
\[
  \Delta x < \frac{\lambda_J}{\texttt{jeans\_ncells}}
\]
Enabling this also activates a polytropic equation-of-state floor to
prevent artificial fragmentation (Truelove criterion).  Typical
value: \textbf{4} (minimum of 4 cells per Jeans length).
\end{quote}


%======================================================================
\section{\texttt{\&HYDRO\_PARAMS} -- Hydrodynamics Solver}
\label{sec:hydro_params}
%======================================================================

Controls the gas dynamics solver configuration.

% ------------------------------------------------------------------
\paramheader{gamma}{real(dp)}{$5/3$}
\label{param:gamma}

\begin{quote}
Adiabatic index $\gamma$ of the ideal gas equation of state,
$P = (\gamma - 1) \rho e$.
Standard value: $5/3$ for a monatomic ideal gas.  Use $7/5$ for
diatomic gas or $4/3$ for radiation-dominated flow.
\end{quote}

% ------------------------------------------------------------------
\paramheader{courant\_factor}{real(dp)}{0.8}
\label{param:courant_factor}

\begin{quote}
Courant--Friedrichs--Lewy (CFL) number for time step control.  The
time step at each level is
$\Delta t = \texttt{courant\_factor} \times \Delta x / v_{\max}$.
Typical: \textbf{0.8}.  Lower values increase stability at the cost
of more time steps.
\end{quote}

% ------------------------------------------------------------------
\paramheader{scheme}{character}{`muscl'}
\label{param:scheme}

\begin{quote}
Hydrodynamics integration scheme.
\begin{description}[style=nextline,leftmargin=2.5cm,font=\ttfamily]
  \item[`muscl'] MUSCL--Hancock (Monotonic Upstream-centred Scheme
    for Conservation Laws), second-order in space and time.  This is
    the only production scheme in RAMSES.
\end{description}
\end{quote}

% ------------------------------------------------------------------
\paramheader{slope\_type}{integer}{1}
\label{param:slope_type}

\begin{quote}
Slope limiter for MUSCL piecewise-linear reconstruction.
\begin{description}[style=nextline,leftmargin=1.5cm,font=\ttfamily]
  \item[1] MinMod -- most robust, more diffusive.
  \item[2] MonCen -- monotonised central, less diffusive.
        \textbf{Recommended} for production runs.
  \item[3] Unlimited -- no limiting (unstable; testing only).
\end{description}
\end{quote}

% ------------------------------------------------------------------
\paramheader{riemann}{character}{`llf'}
\label{param:riemann}

\begin{quote}
Approximate Riemann solver for inter-cell flux computation.
\begin{description}[style=nextline,leftmargin=2.5cm,font=\ttfamily]
  \item[`llf'] Local Lax--Friedrichs (Rusanov).  Most diffusive but
    unconditionally stable.  Good default.
  \item[`hll'] Harten--Lax--van Leer.  Two-wave solver.
  \item[`hllc'] HLL with Contact restoration.  Three-wave solver,
    most accurate for contact discontinuities.
    \textbf{Recommended for cosmological simulations}.
  \item[`exact'] Exact Riemann solver (expensive; primarily for
    validation).
\end{description}
\end{quote}

% ------------------------------------------------------------------
\paramheader{pressure\_fix}{logical}{.false.}
\label{param:pressure_fix}

\begin{quote}
Enable pressure floor to prevent negative pressures in strong shocks
or highly supersonic flows.  When the internal energy becomes negative,
RAMSES falls back to a pressure estimate from the total energy.

\textbf{Recommended:} \texttt{.true.}\ for cosmological simulations.
See also \paramref{beta_fix}.
\end{quote}

% ------------------------------------------------------------------
\paramheader{beta\_fix}{real(dp)}{0.0}
\label{param:beta_fix}

\begin{quote}
Pressure fix parameter.  Controls the magnitude of the pressure floor:
$P_{\text{floor}} = \texttt{beta\_fix} \times \rho v^2 / 2$.
Typical value: \textbf{0.5} when
\paramref{pressure_fix}\texttt{=.true.}
\end{quote}

% ------------------------------------------------------------------
\paramheader{isothermal}{logical}{.false.}
\label{param:isothermal}

\begin{quote}
Isothermal mode.  When \texttt{.true.}, the energy equation is not
solved and the gas temperature remains constant.  Reduces the number
of hydro variables by one.
\end{quote}


%======================================================================
\section{\texttt{\&POISSON\_PARAMS} -- Gravity Solver}
\label{sec:poisson_params}
%======================================================================

Controls the multigrid Poisson solver for self-gravity.

% ------------------------------------------------------------------
\paramheader{epsilon}{real(dp)}{$10^{-4}$}
\label{param:epsilon}

\begin{quote}
Multigrid convergence criterion.  The V-cycle iteration at each level
stops when the residual norm satisfies
$\|r\| / \|r_0\| < \texttt{epsilon}$.
Typical value for cosmological runs: $10^{-5}$ to $10^{-4}$.
Tighter values improve force accuracy but increase iteration count.
\end{quote}

% ------------------------------------------------------------------
\paramheader{gravity\_type}{integer}{0}
\label{param:gravity_type}

\begin{quote}
Gravity model selection.
\begin{description}[style=nextline,leftmargin=1.5cm,font=\ttfamily]
  \item[0] Self-gravity (solve Poisson equation on the AMR grid).
  \item[>0] Analytical gravitational potential (e.g.\ for test
    problems with known solutions).  The integer value selects the
    specific analytical profile.
\end{description}
\end{quote}

% ------------------------------------------------------------------
\paramheader{cg\_levelmin}{integer}{999}
\label{param:cg_levelmin}

\begin{quote}
Minimum level at which the conjugate gradient (CG) fallback solver
activates.  When the multigrid solver stalls at high AMR levels, CG
provides guaranteed convergence.  Set to \paramref{levelmax} for
best convergence behaviour.

Typical: \texttt{cg\_levelmin = levelmax}.  The default (999) means
CG is effectively disabled unless \texttt{levelmax} is absurdly
large.
\end{quote}

% ------------------------------------------------------------------
\paramheader{cic\_levelmax}{integer}{0}
\label{param:cic_levelmax}

\begin{quote}
Maximum level for cloud-in-cell (CIC) particle mass deposition.
\begin{itemize}[nosep]
  \item \texttt{0} -- deposit particles at all levels (standard).
  \item $N > 0$ -- deposit particles only up to level $N$; finer
    levels inherit the coarse density by prolongation.
\end{itemize}
Rarely modified.
\end{quote}


%======================================================================
\section{\texttt{\&PHYSICS\_PARAMS} -- Sub-grid Physics}
\label{sec:physics_params}
%======================================================================

Controls cooling, star formation, stellar/AGN feedback, and
cosmological parameters.  This block is optional; omit it entirely
for adiabatic (non-radiative) simulations.

%----------------------------------------------------------------------
\subsection{Cooling and UV Background}
%----------------------------------------------------------------------

\paramheader{cooling}{logical}{.false.}
\label{param:cooling}

\begin{quote}
Enable radiative cooling with a metal-dependent cooling function.
When \texttt{.true.}, RAMSES integrates the cooling/heating rate
at each time step using tabulated cooling curves.  Requires
\paramref{hydro}\texttt{=.true.}
\end{quote}

% ------------------------------------------------------------------
\paramheader{haardt\_madau}{logical}{.false.}
\label{param:haardt_madau}

\begin{quote}
Enable the Haardt \& Madau (2012) ultraviolet background model for
cosmic reionization.  Provides a redshift-dependent photo-heating
and photo-ionization rate.  Used together with \paramref{cooling}.
\end{quote}

% ------------------------------------------------------------------
\paramheader{z\_reion}{real(dp)}{8.5}
\label{param:z_reion}

\begin{quote}
Reionization redshift.  Hydrogen reionization heating is applied
instantaneously at this redshift.  Typical range: 6--10, depending
on the reionization model.
\end{quote}

% ------------------------------------------------------------------
\paramheader{z\_ave}{real(dp)}{0.0}
\label{param:z_ave}

\begin{quote}
Initial mean metallicity of the gas in solar units ($Z_\odot$).
Applied uniformly at initialization.  Use \texttt{0.0} for
primordial composition.
\end{quote}

% ------------------------------------------------------------------
\paramheader{delayed\_cooling}{logical}{.false.}
\label{param:delayed_cooling}

\begin{quote}
Delay radiative cooling in supernova-heated gas to prevent
overcooling.  When a cell receives SN energy, cooling is suppressed
for a duration related to the Sedov--Taylor phase.  Improves the
effectiveness of stellar feedback in regulating star formation.
\end{quote}

% ------------------------------------------------------------------
\paramheader{tol}{real(dp)}{$10^{-3}$}
\label{param:tol}

\begin{quote}
Tolerance for the implicit cooling solver.  The Newton--Raphson
iteration converges when the relative temperature change
$|\Delta T / T| < \texttt{tol}$.
\end{quote}

%----------------------------------------------------------------------
\subsection{Star Formation}
%----------------------------------------------------------------------

\paramheader{n\_star}{real(dp)}{0.1}
\label{param:n_star}

\begin{quote}
Star formation hydrogen number density threshold in
$\text{H}\;\text{cm}^{-3}$.  Only gas denser than this value is
eligible for star formation.  Typical range: 0.1--10.
\end{quote}

% ------------------------------------------------------------------
\paramheader{eps\_star}{real(dp)}{0.0}
\label{param:eps_star}

\begin{quote}
Star formation efficiency per free-fall time $\epsilon_{\text{ff}}$.
The star formation rate density is
$\dot{\rho}_\star = \epsilon_{\text{ff}} \, \rho_{\text{gas}} /
t_{\text{ff}}$.
Typical value: \textbf{0.01--0.02} (1--2\%\ per free-fall time).
Set to \texttt{0.0} to disable star formation entirely.
\end{quote}

% ------------------------------------------------------------------
\paramheader{del\_star}{real(dp)}{200}
\label{param:del_star}

\begin{quote}
Overdensity threshold for star formation (in units of the cosmic
mean density).  Gas must exceed $\delta > \texttt{del\_star}$ in
addition to the density threshold \paramref{n_star}.
\end{quote}

% ------------------------------------------------------------------
\paramheader{m\_star}{real(dp)}{$-1$}
\label{param:m_star}

\begin{quote}
Minimum stellar particle mass in code units.  When a star-forming
cell would produce a particle below this mass, the event is
stochastically deferred to the next time step.
\begin{itemize}[nosep]
  \item $< 0$: use the cell gas mass (no minimum).
  \item $> 0$: explicit minimum mass.
\end{itemize}
\end{quote}

% ------------------------------------------------------------------
\paramheader{T2\_star}{real(dp)}{0}
\label{param:T2_star}

\begin{quote}
ISM polytropic equation-of-state temperature floor in Kelvin.  Gas
above the star-formation density threshold \paramref{n_star} follows
a polytropic relation:
\[
  T = T_{2,\star} \left(\frac{n}{n_\star}\right)^{\gamma_\star - 1}
\]
where $\gamma_\star$ is \paramref{g_star}.  This prevents artificial
fragmentation below the resolution limit (Jeans mass floor).
\end{quote}

% ------------------------------------------------------------------
\paramheader{g\_star}{real(dp)}{1.6}
\label{param:g_star}

\begin{quote}
Polytropic index $\gamma_\star$ for the ISM equation of state (see
\paramref{T2_star}).  Typical value: \textbf{1.6} (stiff polytrope)
or \textbf{5/3} (adiabatic floor).
\end{quote}

%----------------------------------------------------------------------
\subsection{Stellar Feedback}
%----------------------------------------------------------------------

\paramheader{f\_w}{real(dp)}{0}
\label{param:f_w}

\begin{quote}
Mass loading factor for supernova-driven winds.  The wind mass flux
is $\dot{M}_w = \texttt{f\_w} \times \dot{M}_\star$.  Set to
\texttt{0} to disable winds.  Typical range: 1--5.
\end{quote}

% ------------------------------------------------------------------
\paramheader{f\_ek}{real(dp)}{1.0}
\label{param:f_ek}

\begin{quote}
Kinetic energy fraction of supernova feedback.  Controls the
partition between kinetic (\texttt{f\_ek}) and thermal
($1 - \texttt{f\_ek}$) energy injection.  \texttt{f\_ek=1} is
purely kinetic feedback; \texttt{f\_ek=0} is purely thermal.
\end{quote}

% ------------------------------------------------------------------
\paramheader{eps\_sn1}{real(dp)}{0}
\label{param:eps_sn1}

\begin{quote}
Type Ia supernova energy per event in units of $10^{51}\,$erg.
Set to \texttt{0} to disable Type Ia SN feedback.
\end{quote}

% ------------------------------------------------------------------
\paramheader{eps\_sn2}{real(dp)}{0}
\label{param:eps_sn2}

\begin{quote}
Type II supernova energy per event in units of $10^{51}\,$erg.
Set to \texttt{0} to disable Type II SN feedback.
\end{quote}

% ------------------------------------------------------------------
\paramheader{yieldtablefilename}{character}{---}
\label{param:yieldtablefilename}

\begin{quote}
Path to the chemical yield table file for metal enrichment
calculations.  Required when metal-dependent cooling or chemical
evolution tracking is enabled.
\end{quote}

%----------------------------------------------------------------------
\subsection{Cosmological Parameters}
%----------------------------------------------------------------------

\paramheader{omega\_b}{real(dp)}{---}
\label{param:omega_b}

\begin{quote}
Baryon density parameter $\Omega_b$.  Overrides the value read from
the IC file header.  Must be consistent with the initial conditions
and other cosmological parameters ($\Omega_m$, $H_0$, etc.\ are
read from the GRAFIC2 header).
\end{quote}

%----------------------------------------------------------------------
\subsection{AGN and Sink Particle Parameters}
%----------------------------------------------------------------------

\paramheader{Mseed}{real(dp)}{---}
\label{param:Mseed}

\begin{quote}
Seed black hole mass in solar masses ($M_\odot$).  When a sink
particle forms, it is initialised with this mass.  Typical range:
$10^4$--$10^6\,M_\odot$ for cosmological simulations.
\end{quote}

% ------------------------------------------------------------------
\paramheader{sink\_AGN}{logical}{.false.}
\label{param:sink_AGN}

\begin{quote}
Enable AGN feedback from sink particles.  When \texttt{.true.}, sink
particles inject thermal and/or kinetic energy into their surroundings
based on their accretion rate.  Requires
\paramref{sink}\texttt{=.true.}
\end{quote}

% ------------------------------------------------------------------
\paramheader{bondi}{logical}{.false.}
\label{param:bondi}

\begin{quote}
Enable Bondi--Hoyle--Lyttleton accretion for sink particles.  The
accretion rate is computed from the local gas density, sound speed,
and relative velocity:
\[
  \dot{M} = \frac{4\pi G^2 M_{\text{BH}}^2 \rho}%
             {(c_s^2 + v_{\text{rel}}^2)^{3/2}}
\]
Can be boosted by \paramref{boost_acc}.
\end{quote}

% ------------------------------------------------------------------
\paramheader{drag}{logical}{.false.}
\label{param:drag}

\begin{quote}
Enable dynamical friction on sink particles.  Applies a drag force
opposing the sink's motion relative to the background gas.  Strength
can be amplified by \paramref{boost_drag}.
\end{quote}

% ------------------------------------------------------------------
\paramheader{rAGN}{real(dp)}{---}
\label{param:rAGN}

\begin{quote}
AGN feedback energy injection radius in units of the cell size at
\paramref{levelmax}.  Feedback energy is distributed over a sphere
of this radius centred on the sink particle.
\end{quote}

% ------------------------------------------------------------------
\paramheader{X\_floor}{real(dp)}{---}
\label{param:X_floor}

\begin{quote}
Hydrogen mass fraction floor.  Prevents the hydrogen fraction from
dropping below this value due to numerical artefacts.  Typical:
\texttt{0.76}.
\end{quote}

% ------------------------------------------------------------------
\paramheader{eAGN\_K}{real(dp)}{---}
\label{param:eAGN_K}

\begin{quote}
AGN kinetic feedback efficiency $\epsilon_K$.  Fraction of the
accreted rest-mass energy deposited as kinetic energy:
$\dot{E}_K = \epsilon_K \, \dot{M} c^2$.
\end{quote}

% ------------------------------------------------------------------
\paramheader{eAGN\_T}{real(dp)}{---}
\label{param:eAGN_T}

\begin{quote}
AGN thermal feedback efficiency $\epsilon_T$.  Fraction of the
accreted rest-mass energy deposited as thermal energy:
$\dot{E}_T = \epsilon_T \, \dot{M} c^2$.
\end{quote}

% ------------------------------------------------------------------
\paramheader{TAGN}{real(dp)}{---}
\label{param:TAGN}

\begin{quote}
AGN heating temperature in Kelvin.  The AGN thermal energy is
deposited by raising gas temperature toward this value within the
feedback region \paramref{rAGN}.
\end{quote}

% ------------------------------------------------------------------
\paramheader{r\_gal}{real(dp)}{---}
\label{param:r_gal}

\begin{quote}
Galaxy definition radius for AGN feedback, in code units.  Used to
compute the local galaxy properties (stellar mass, gas mass) around
a sink particle for AGN mode switching.
\end{quote}

% ------------------------------------------------------------------
\paramheader{T2maxAGN}{real(dp)}{---}
\label{param:T2maxAGN}

\begin{quote}
Maximum AGN heating temperature in Kelvin.  Caps the temperature
increase from a single AGN feedback event to prevent unphysically
hot gas.
\end{quote}

% ------------------------------------------------------------------
\paramheader{boost\_acc}{real(dp)}{---}
\label{param:boost_acc}

\begin{quote}
Bondi accretion boost factor.  Multiplies the Bondi--Hoyle accretion
rate by this factor to compensate for unresolved gas structure near
the black hole.  Typical range: 1--100.  Requires
\paramref{bondi}\texttt{=.true.}
\end{quote}

% ------------------------------------------------------------------
\paramheader{boost\_drag}{real(dp)}{---}
\label{param:boost_drag}

\begin{quote}
Dynamical friction drag boost factor.  Multiplies the drag force by
this factor.  Requires \paramref{drag}\texttt{=.true.}
\end{quote}

% ------------------------------------------------------------------
\paramheader{vrel\_merge}{logical}{---}
\label{param:vrel_merge}

\begin{quote}
Use relative velocity criterion for sink particle merging.  When
\texttt{.true.}, two sinks merge only if their relative velocity is
below the local escape velocity, in addition to the spatial proximity
criterion \paramref{rmerge}.
\end{quote}

% ------------------------------------------------------------------
\paramheader{rmerge}{real(dp)}{---}
\label{param:rmerge}

\begin{quote}
Sink merging radius in units of the cell size at
\paramref{levelmax}.  Two sink particles closer than this distance
are candidates for merging (subject to additional criteria if
\paramref{vrel_merge}\texttt{=.true.}).
\end{quote}

% ------------------------------------------------------------------
\paramheader{spin\_bh}{logical}{---}
\label{param:spin_bh}

\begin{quote}
Track black hole spin evolution.  When \texttt{.true.}, the code
evolves the dimensionless spin parameter $a_\star$ of each sink
particle based on the angular momentum of accreted gas.
\end{quote}

% ------------------------------------------------------------------
\paramheader{mad\_jet}{logical}{---}
\label{param:mad_jet}

\begin{quote}
Enable the magnetically arrested disk (MAD) jet model.  When
\texttt{.true.}, AGN kinetic feedback is launched as a collimated
bipolar jet aligned with the black hole spin axis.  Requires
\paramref{spin_bh}\texttt{=.true.}
\end{quote}


%======================================================================
\section{\texttt{\&LIGHTCONE\_PARAMS} -- Lightcone Output}
\label{sec:lightcone_params}
%======================================================================

\begin{quote}
Parameters for lightcone output mode (activated when
\paramref{lightcone}\texttt{=.true.}).  In this mode, particles
and/or cells that cross the observer's past lightcone during each
time step are written to special output files, enabling the
construction of mock galaxy surveys and weak-lensing maps without
storing full snapshots.

Configuration parameters include the observer position, opening
angle, and selection criteria.  Consult the RAMSES lightcone
documentation for the full parameter list, which varies by
application.
\end{quote}


%======================================================================
\section{\texttt{\&SPHERICAL\_REGION\_PARAMS}}
\label{sec:spherical_region_params}
%======================================================================

\paramheader{spherical\_region}{logical}{.false.}
\label{param:spherical_region}

\begin{quote}
Enable a spherical refinement region.  When \texttt{.true.}, AMR
refinement is restricted to a spherical sub-volume of the simulation
box.  This is useful for re-simulations of specific halos where a
cubic zoom region is not optimal.  Additional parameters define the
centre and radius of the sphere.
\end{quote}


%======================================================================
\section{Complete Example: Cosmological Zoom-In}
\label{sec:example}
%======================================================================

The following namelist illustrates a production cosmological zoom-in
simulation with dark matter particles, baryonic gas, self-gravity,
cooling, star formation, and AGN feedback.

\begin{lstlisting}[language=Fortran,
  caption={Cosmological zoom-in namelist},
  label={lst:zoomin}]
&RUN_PARAMS
cosmo    = .true.
pic      = .true.
poisson  = .true.
hydro    = .true.
sink     = .true.
nrestart = 0
nremap   = 5
nsubcycle = 1, 1, 1, 2, 2, 2, 2
nstepmax = 10000000
ordering = 'ksection'
memory_balance = .true.
/

&AMR_PARAMS
levelmin  = 7
levelmax  = 18
nexpand   = 1, 1, 1, 1, 1, 1, 1, 1, 1, 1, 1, 1
ngridtot  = 400000000
nparttot  = 600000000
/

&OUTPUT_PARAMS
noutput = 10
aout = 0.05, 0.1, 0.15, 0.2, 0.3, 0.4, 0.5, 0.7, 0.85, 1.0
foutput = 500
/

&INIT_PARAMS
filetype = 'grafic'
initfile = '/data/IC/level_07'
         , '/data/IC/level_08'
         , '/data/IC/level_09'
         , '/data/IC/level_10'
         , '/data/IC/level_11'
         , '/data/IC/level_12'
         , '/data/IC/level_13'
/

&REFINE_PARAMS
m_refine = 8., 8., 8., 8., 8., 8., 8., 8., 8., 8., 8., 8.
ivar_refine     = 11
var_cut_refine  = 0.01
mass_cut_refine = 3.63798e-12
interpol_var    = 1
interpol_type   = 1
/

&HYDRO_PARAMS
gamma         = 1.6666667
courant_factor = 0.8
scheme        = 'muscl'
slope_type    = 2
riemann       = 'hllc'
pressure_fix  = .true.
beta_fix      = 0.5
/

&POISSON_PARAMS
epsilon      = 1.0e-5
gravity_type = 0
cg_levelmin  = 18
/

&PHYSICS_PARAMS
cooling        = .true.
haardt_madau   = .true.
z_reion        = 8.5
n_star         = 0.1
eps_star       = 0.02
T2_star        = 1.0e4
g_star         = 1.6
del_star       = 200.0
f_ek           = 1.0
sink_AGN       = .true.
bondi          = .true.
Mseed          = 1.0e5
/
\end{lstlisting}


%======================================================================
\section{Parameter Cross-Reference Index}
\label{sec:crossref}
%======================================================================

Table~\ref{tab:crossref} lists parameters that commonly interact and
should be set consistently.

\begin{longtable}{@{}p{4cm}p{4.5cm}p{6.5cm}@{}}
\caption{Cross-reference of interacting parameters.}
\label{tab:crossref}\\
\toprule
\textbf{Parameter} & \textbf{Related Parameters} & \textbf{Notes} \\
\midrule
\endfirsthead
\toprule
\textbf{Parameter} & \textbf{Related Parameters} & \textbf{Notes} \\
\midrule
\endhead
\bottomrule
\endfoot
\texttt{cosmo}
  & \texttt{aout}, \texttt{omega\_b}
  & Cosmological mode requires scale-factor outputs \\
\addlinespace
\texttt{pic}
  & \texttt{poisson}, \texttt{nparttot}
  & Particles need gravity and memory allocation \\
\addlinespace
\texttt{ordering}
  & \texttt{memory\_balance}
  & Memory balancing requires \texttt{ksection} \\
\addlinespace
\texttt{levelmax}
  & \texttt{m\_refine}, \texttt{cg\_levelmin}
  & Set \texttt{cg\_levelmin = levelmax} \\
\addlinespace
\texttt{ivar\_refine}
  & \texttt{var\_cut\_refine}, \texttt{mass\_cut\_refine}, \texttt{m\_refine}
  & All must be consistent for zoom-in \\
\addlinespace
\texttt{mass\_cut\_refine}
  & \texttt{ivar\_refine}
  & Set to finest-level DM particle mass \\
\addlinespace
\texttt{pressure\_fix}
  & \texttt{beta\_fix}
  & \texttt{beta\_fix} only effective when fix is on \\
\addlinespace
\texttt{T2\_star}
  & \texttt{g\_star}, \texttt{n\_star}
  & Polytropic EOS parameters \\
\addlinespace
\texttt{sink}
  & \texttt{sink\_AGN}, \texttt{bondi}, \texttt{Mseed}
  & AGN feedback requires sink particles \\
\addlinespace
\texttt{sink\_AGN}
  & \texttt{eAGN\_K}, \texttt{eAGN\_T}, \texttt{TAGN}, \texttt{rAGN}
  & AGN feedback parameters \\
\addlinespace
\texttt{bondi}
  & \texttt{boost\_acc}
  & Boost factor for unresolved accretion \\
\addlinespace
\texttt{drag}
  & \texttt{boost\_drag}
  & Drag boost factor \\
\addlinespace
\texttt{spin\_bh}
  & \texttt{mad\_jet}
  & MAD jet requires spin tracking \\
\addlinespace
\texttt{cooling}
  & \texttt{haardt\_madau}, \texttt{z\_reion}
  & UV background for reionization heating \\
\addlinespace
\texttt{ngridtot}
  & \texttt{nparttot}
  & Both determine per-process memory usage \\
\end{longtable}


%======================================================================
\vfill
\begin{center}
\small\sffamily\textcolor{apitype}{%
Generated for cuRAMSES-kjhan --- \today}
\end{center}

\end{document}

%   (when included, comment out the \documentclass .. \begin{document}
%    and \end{document} lines, or wrap them with \ifx\STANDALONE\undefined)
%%%%%%%%%%%%%%%%%%%%%%%%%%%%%%%%%%%%%%%%%%%%%%%%%%%%%%%%%%%%%%%%%%%%%%%%%

\documentclass[11pt]{article}

%----------------------------------------------------------------------
%  Packages
%----------------------------------------------------------------------
\usepackage[top=2.5cm,bottom=2.5cm,left=2.8cm,right=2.8cm,a4paper]{geometry}
\usepackage[utf8]{inputenc}
\usepackage[T1]{fontenc}
\usepackage{lmodern}
\usepackage{microtype}
\usepackage{xcolor}
\usepackage{hyperref}
\usepackage{enumitem}
\usepackage{amsmath,amssymb}
\usepackage{booktabs}
\usepackage{longtable}
\usepackage{fancyhdr}
\usepackage{titlesec}
\usepackage{parskip}
\usepackage{listings}

%----------------------------------------------------------------------
%  Colour palette (CAMB-readthedocs inspired)
%----------------------------------------------------------------------
\definecolor{apibg}{RGB}{248,249,251}       % light background
\definecolor{apiborder}{RGB}{200,210,225}    % border
\definecolor{apiname}{RGB}{0,64,128}         % parameter name
\definecolor{apitype}{RGB}{100,100,100}      % type annotation
\definecolor{apidefault}{RGB}{60,130,60}     % default value
\definecolor{apinote}{RGB}{180,80,40}        % warning / note
\definecolor{sectioncolor}{RGB}{0,80,160}    % section headings
\definecolor{codebg}{RGB}{243,245,248}
\definecolor{codeframe}{RGB}{190,200,215}

%----------------------------------------------------------------------
%  Hyperref
%----------------------------------------------------------------------
\hypersetup{
  colorlinks=true,
  linkcolor=sectioncolor,
  urlcolor=sectioncolor!80!black,
  citecolor=sectioncolor,
  pdfauthor={Juhan Kim},
  pdftitle={RAMSES Namelist Parameter Reference},
}

%----------------------------------------------------------------------
%  Section formatting
%----------------------------------------------------------------------
\titleformat{\section}
  {\Large\bfseries\sffamily\color{sectioncolor}}
  {\thesection}{1em}{}
  [\vspace{-0.6em}\textcolor{sectioncolor!40}{\rule{\linewidth}{0.8pt}}]

\titleformat{\subsection}
  {\large\bfseries\sffamily\color{sectioncolor!80!black}}
  {\thesubsection}{0.8em}{}

%----------------------------------------------------------------------
%  Header / footer
%----------------------------------------------------------------------
\pagestyle{fancy}
\fancyhf{}
\fancyhead[L]{\small\sffamily\textcolor{apitype}{RAMSES Namelist Reference}}
\fancyhead[R]{\small\sffamily\textcolor{apitype}{\nouppercase{\leftmark}}}
\fancyfoot[C]{\small\sffamily\thepage}
\renewcommand{\headrulewidth}{0.4pt}

%----------------------------------------------------------------------
%  Code listings
%----------------------------------------------------------------------
\lstset{
  basicstyle=\ttfamily\small,
  backgroundcolor=\color{codebg},
  frame=single,
  rulecolor=\color{codeframe},
  framesep=4pt,
  xleftmargin=6pt,
  xrightmargin=6pt,
  breaklines=true,
  columns=fullflexible,
  keepspaces=true,
  showstringspaces=false,
}

%----------------------------------------------------------------------
%  Custom environments for parameter documentation.
%
%  We use a "paramblock" environment instead of a macro so that
%  verbatim-like content (lstlisting) can appear inside parameter
%  descriptions.
%----------------------------------------------------------------------

% --- parameter header bar (name + type + default) ---
\newcommand{\paramheader}[3]{%
  \vspace{0.7em}%
  \noindent
  \colorbox{apibg}{%
    \parbox{\dimexpr\linewidth-2\fboxsep}{%
      \vspace{4pt}%
      \hspace{2pt}%
      {\large\textcolor{apiname}{\texttt{\textbf{#1}}}}%
      \hfill
      {\small\textcolor{apitype}{\textsf{#2}}}%
      \quad
      {\small\textcolor{apidefault}{\textsf{default:~\texttt{#3}}}}%
      \vspace{4pt}%
    }%
  }%
  \nopagebreak[4]\par\nopagebreak[4]\vspace{0.25em}%
}

% --- required parameter header bar (no default) ---
\newcommand{\paramheaderreq}[2]{%
  \vspace{0.7em}%
  \noindent
  \colorbox{apibg}{%
    \parbox{\dimexpr\linewidth-2\fboxsep}{%
      \vspace{4pt}%
      \hspace{2pt}%
      {\large\textcolor{apiname}{\texttt{\textbf{#1}}}}%
      \hfill
      {\small\textcolor{apitype}{\textsf{#2}}}%
      \quad
      {\small\textcolor{apinote}{\textsf{required}}}%
      \vspace{4pt}%
    }%
  }%
  \nopagebreak[4]\par\nopagebreak[4]\vspace{0.25em}%
}

%----------------------------------------------------------------------
%  Note / Warning / Example helpers (usable at top level)
%----------------------------------------------------------------------
\newcommand{\apinote}[1]{%
  \par\vspace{0.3em}%
  \noindent\textcolor{sectioncolor}{\textsf{\textbf{Note:}}} #1%
  \par\vspace{0.2em}%
}

\newcommand{\apiwarning}[1]{%
  \par\vspace{0.3em}%
  \noindent\textcolor{apinote}{\textsf{\textbf{Warning:}}} #1%
  \par\vspace{0.2em}%
}

%----------------------------------------------------------------------
%  Cross-reference helper
%  Usage: \paramref{cosmo}  or  \paramref{memory_balance}
%  The label name uses the raw form (with literal underscores);
%  the displayed text uses \texttt with escaped underscores.
%----------------------------------------------------------------------
\makeatletter
\newcommand{\paramref}[1]{%
  \begingroup
  \def\tmp{#1}%
  \hyperref[param:\tmp]{\texttt{\detokenize{#1}}}%
  \endgroup
}
\makeatother

%======================================================================
\begin{document}

%----------------------------------------------------------------------
%  Title
%----------------------------------------------------------------------
\begin{center}
{\fontsize{28}{34}\selectfont\sffamily\bfseries
  \textcolor{sectioncolor}{RAMSES Namelist Parameter Reference}}\\[0.8cm]
{\large\sffamily cuRAMSES-kjhan -- February 2026}\\[0.3cm]
{\normalsize\sffamily Juhan Kim}\\[0.2cm]
{\small\sffamily\textcolor{apitype}{%
  Based on RAMSES by Romain Teyssier}}
\end{center}

\vspace{0.5cm}
\noindent\textcolor{sectioncolor!40}{\rule{\linewidth}{1pt}}

\vspace{0.6cm}
\noindent
This document provides a complete reference for every namelist parameter
accepted by RAMSES and the cuRAMSES-kjhan extensions.  Parameters are
grouped by their Fortran namelist block (\texttt{\&RUN\_PARAMS},
\texttt{\&AMR\_PARAMS}, etc.).  Each entry specifies the parameter name,
Fortran type, default value, and a detailed description including valid
ranges and interactions with other parameters.

The namelist file uses standard Fortran namelist syntax.  Each block
begins with \texttt{\&BLOCK\_NAME} and ends with a single~\texttt{/}.

\vspace{0.4cm}
\tableofcontents
\newpage

%======================================================================
\section{\texttt{\&RUN\_PARAMS} -- Global Run Control}
\label{sec:run_params}
%======================================================================

This mandatory block controls the physics modules to activate, restart
behaviour, domain decomposition strategy, and general simulation
parameters.

% ------------------------------------------------------------------
\paramheader{cosmo}{logical}{.false.}
\label{param:cosmo}

\begin{quote}
Enable cosmological mode.  When \texttt{.true.}, RAMSES uses comoving
(super-comoving) coordinates with the expansion factor $a(t)$ as the
time variable.  The box length is interpreted in $h^{-1}\,$Mpc.
Friedmann equations are integrated internally.

Enabling this flag also activates expansion-factor--based output
scheduling (see \paramref{aout} in
Section~\ref{sec:output_params}).  Cosmological initial conditions
(GRAFIC2 format) must be provided via \paramref{initfile}.
\end{quote}

% ------------------------------------------------------------------
\paramheader{pic}{logical}{.false.}
\label{param:pic}

\begin{quote}
Enable the Particle-In-Cell (PIC) method for collisionless $N$-body
dynamics (dark matter, stars).  Particles are deposited onto the AMR
grid using cloud-in-cell (CIC) interpolation, and forces are
interpolated back to particle positions.

Required for any simulation containing dark matter particles.  Usually
combined with \paramref{poisson}\texttt{=.true.}
\end{quote}

% ------------------------------------------------------------------
\paramheader{poisson}{logical}{.false.}
\label{param:poisson}

\begin{quote}
Enable the self-gravity Poisson solver.  RAMSES uses an adaptive
multigrid (MG) method on the AMR hierarchy with V-cycles and
red--black Gauss--Seidel smoothing.  Convergence is controlled by
\paramref{epsilon} in \texttt{\&POISSON\_PARAMS}.

Must be \texttt{.true.}\ whenever \paramref{pic}\texttt{=.true.}\ or
whenever gas self-gravity is desired.
\end{quote}

% ------------------------------------------------------------------
\paramheader{hydro}{logical}{.false.}
\label{param:hydro}

\begin{quote}
Enable the hydrodynamics (or MHD) solver.  RAMSES employs a
second-order MUSCL--Hancock scheme with approximate Riemann solvers
(see \paramref{scheme}, \paramref{riemann} in
Section~\ref{sec:hydro_params}).

Set to \texttt{.true.}\ for any simulation involving baryonic gas.
\end{quote}

% ------------------------------------------------------------------
\paramheader{nrestart}{integer}{0}
\label{param:nrestart}

\begin{quote}
Restart from checkpoint (output snapshot) number \texttt{nrestart}.
\begin{itemize}[nosep]
  \item \texttt{nrestart=0} -- fresh start from initial conditions.
  \item \texttt{nrestart=}$N$ -- load \texttt{output\_}$N$\texttt{/} and resume.
\end{itemize}
The number of MPI processes must match the run that produced the
checkpoint.  RAMSES reads all AMR, hydro, particle, and gravity data
from the snapshot directory.
\end{quote}

% ------------------------------------------------------------------
\paramheader{nremap}{integer}{5}
\label{param:nremap}

\begin{quote}
Load-balancing frequency: perform domain decomposition every
\texttt{nremap} coarse time steps.  Recommended value: \textbf{5}
(balances redistribution overhead against growing load imbalance).
Set to \texttt{0} to disable load balancing entirely.

\apinote{Benchmarks (200\,M particles, 12~ranks, 10~steps) show that
\texttt{nremap=5} reduces total runtime by 18\%\ compared to
\texttt{nremap=1}, with load-balance overhead at 6.3\%\ of wall time.
Larger values (e.g.\ 10) save overhead but allow imbalance to grow.}
\end{quote}

% ------------------------------------------------------------------
\paramheader{nsubcycle}{integer array}{1,1,2}
\label{param:nsubcycle}

\begin{quote}
Time sub-cycling factors per AMR level.  The $i$-th entry gives the
number of fine time steps per coarse step at level
$\texttt{levelmin}+i-1$.  Typical usage:
\begin{itemize}[nosep]
  \item \texttt{1} for coarse levels (no sub-cycling).
  \item \texttt{2} for fine levels (halve the time step at each
        finer level).
\end{itemize}
The array has up to \texttt{levelmax\,--\,levelmin+1} entries.  Any
unspecified trailing entries default to~\texttt{1}.
\end{quote}

\noindent\textcolor{apidefault}{\textsf{\textbf{Example:}}}
\begin{lstlisting}
nsubcycle = 1, 1, 1, 2, 2, 2, 2
\end{lstlisting}

% ------------------------------------------------------------------
\paramheader{ncontrol}{integer}{1}
\label{param:ncontrol}

\begin{quote}
Print control output (energy diagnostics, timing) every
\texttt{ncontrol} coarse time steps to standard output.
\end{quote}

% ------------------------------------------------------------------
\paramheader{nstepmax}{integer}{1000000}
\label{param:nstepmax}

\begin{quote}
Maximum number of coarse time steps.  The simulation stops when
\texttt{nstep} reaches this value, even if the final output time has
not been reached.  Useful for short test runs.
\end{quote}

% ------------------------------------------------------------------
\paramheader{ordering}{character}{`hilbert'}
\label{param:ordering}

\begin{quote}
Domain decomposition ordering strategy.
\begin{description}[style=nextline,leftmargin=2.5cm,font=\ttfamily]
  \item[`hilbert'] Hilbert space-filling curve.  Standard choice for
    moderate core counts ($\lesssim 1000$).
  \item[`ksection'] K-section tree-based decomposition.  Provides
    $O(k)$~message scaling (where $k$ is the branching factor) for
    large core counts.  Enables hierarchical MPI exchanges and
    memory-based load balancing (see \paramref{memory_balance}).
\end{description}
When \texttt{ordering='ksection'}, the communication pattern in ghost
zone exchanges, multigrid solvers, and \texttt{build\_comm} all switch
to ksection tree routing automatically.
\end{quote}

% ------------------------------------------------------------------
\paramheader{memory\_balance}{logical}{.false.}
\label{param:memory_balance}

\begin{quote}
Enable memory-based load balancing.  When \texttt{.true.}, the
bisection histogram weights each cell by its memory footprint (grid
metadata + attached particles) instead of uniform cell count.

\textbf{Requires} \paramref{ordering}\texttt{='ksection'}.

The cell cost function is:
\[
  C_{\text{cell}} = \frac{\texttt{mem\_weight\_grid}}{\texttt{twotondim}}
                + n_{\text{part}} \times
                  \frac{\texttt{mem\_weight\_part}}{\texttt{twotondim}}
\]
where $n_{\text{part}}$ is the number of particles attached to the parent
grid.  The weight parameters \texttt{mem\_weight\_grid} (default~270) and
\texttt{mem\_weight\_part} (default~12) are set in the same namelist block.

\apinote{All histogram variables (\texttt{bisec\_hist},
\texttt{bisec\_cpu\_load}, \texttt{cell\_cost}) use 64-bit integers
(\texttt{integer(i8b)}) and \texttt{MPI\_INTEGER8} to avoid overflow at
high particle counts.}
\end{quote}

% ------------------------------------------------------------------
\paramheader{sink}{logical}{.false.}
\label{param:sink}

\begin{quote}
Enable sink particle formation and evolution.  Sink particles represent
compact objects (e.g.\ black holes, protostars) that accrete gas from
their surroundings.  When active, cells exceeding a density threshold
at \texttt{levelmax} can spawn sink particles.

See also \paramref{sink_AGN}, \paramref{bondi}, \paramref{Mseed} in
Section~\ref{sec:physics_params}.
\end{quote}

% ------------------------------------------------------------------
\paramheader{sinkprops}{logical}{.false.}
\label{param:sinkprops}

\begin{quote}
Output detailed sink particle properties (mass, position, velocity,
accretion rate, spin) to dedicated files at each snapshot.
\end{quote}

% ------------------------------------------------------------------
\paramheader{lightcone}{logical}{.false.}
\label{param:lightcone}

\begin{quote}
Enable lightcone output mode.  When \texttt{.true.}, particles and/or
cells crossing the observer's past lightcone are written to special
output files during the simulation.  See
Section~\ref{sec:lightcone_params} for additional parameters.
\end{quote}

% ------------------------------------------------------------------
\paramheader{verbose}{logical}{.false.}
\label{param:verbose}

\begin{quote}
Enable verbose output during initialization and evolution.  Prints
additional diagnostics (grid counts, memory usage, load balance
statistics) to standard output at each coarse step.
\end{quote}


%======================================================================
\section{\texttt{\&AMR\_PARAMS} -- Adaptive Mesh Refinement}
\label{sec:amr_params}
%======================================================================

This mandatory block controls the AMR grid hierarchy, memory allocation
sizes, and the simulation box geometry.

% ------------------------------------------------------------------
\paramheaderreq{levelmin}{integer}
\label{param:levelmin}

\begin{quote}
Minimum (base) AMR level.  The base grid has $2^{\texttt{levelmin}}$
cells per dimension.

\noindent\textcolor{apidefault}{\textsf{\textbf{Example:}}}
\texttt{levelmin=7} produces a $128^3$ base grid.
\texttt{levelmin=9} produces a $512^3$ base grid.

This level is fully covered -- every cell at \texttt{levelmin} exists
on exactly one MPI process.
\end{quote}

% ------------------------------------------------------------------
\paramheaderreq{levelmax}{integer}
\label{param:levelmax}

\begin{quote}
Maximum AMR level.  Determines the finest attainable resolution:
\[
  \Delta x_{\min} = \frac{L_{\text{box}}}{2^{\texttt{levelmax}}}
\]
For cosmological zoom-in simulations, this controls the physical
resolution at $z=0$.  The number of refinement levels beyond
\texttt{levelmin} is \texttt{levelmax\,--\,levelmin}.

Refinement criteria (\paramref{m_refine}, \paramref{ivar_refine})
determine which cells actually refine up to this level.
\end{quote}

% ------------------------------------------------------------------
\paramheader{nexpand}{integer array}{1}
\label{param:nexpand}

\begin{quote}
Number of buffer (guard) cell layers per level to ensure smooth
transitions between refinement levels.  The $i$-th entry applies to
level $\texttt{levelmin}+i-1$.  Typical value: \texttt{1} for all
levels.

Larger values produce wider buffer zones around refined patches,
improving solution quality at the cost of more cells.
\end{quote}

% ------------------------------------------------------------------
\paramheaderreq{ngridtot}{integer(i8b)}
\label{param:ngridtot}

\begin{quote}
Total number of AMR grids (octs) allocated across all MPI processes.
Each process receives $\texttt{ngridmax} = \texttt{ngridtot} /
\texttt{ncpu}$ grids.  Each grid (oct) contains $2^{\texttt{ndim}}$
cells (8~cells in 3D).

\apiwarning{RAMSES allocates full arrays at startup based on
\texttt{ngridmax}.  The virtual memory footprint is approximately
$\texttt{ngridmax} \times 20\;\text{bytes} \times \texttt{nvar}$.
This must not exceed the system's \texttt{CommitLimit} (typically
RAM~$\times$~\texttt{overcommit\_ratio}/100).}

\textbf{Rule of thumb:} For $N$~processes on a node with $M$\,GB of
RAM,
\[
  \texttt{ngridtot} < \frac{M \times 0.5}{20 \times \texttt{nvar}}
  \times N
\]
\end{quote}

% ------------------------------------------------------------------
\paramheaderreq{nparttot}{integer(i8b)}
\label{param:nparttot}

\begin{quote}
Total particle allocation across all MPI processes.  Each process gets
$\texttt{npartmax} = \texttt{nparttot} / \texttt{ncpu}$.  Should be
at least $2\times$ the total number of DM + star particles expected
during the simulation (to accommodate load imbalance and new star
particle creation).
\end{quote}

\noindent\textcolor{apidefault}{\textsf{\textbf{Example:}}}
\begin{lstlisting}
! 100M DM particles, allow for stars
nparttot = 300000000
\end{lstlisting}

% ------------------------------------------------------------------
\paramheader{boxlen}{real(dp)}{1.0}
\label{param:boxlen}

\begin{quote}
Box length in code units.  For cosmological runs, the box size is
typically read from the IC header (in $h^{-1}\,$Mpc) and this
parameter is overridden.  For non-cosmological (idealised) setups,
\texttt{boxlen} defines the physical domain extent.
\end{quote}


%======================================================================
\section{\texttt{\&OUTPUT\_PARAMS} -- Snapshot Output}
\label{sec:output_params}
%======================================================================

Controls when and how simulation snapshots are written to disk.

% ------------------------------------------------------------------
\paramheader{noutput}{integer}{1}
\label{param:noutput}

\begin{quote}
Number of output snapshots requested.  The corresponding times or
expansion factors must be listed in \paramref{tout} (non-cosmological)
or \paramref{aout} (cosmological).
\end{quote}

% ------------------------------------------------------------------
\paramheader{aout}{real(dp) array}{---}
\label{param:aout}

\begin{quote}
Scale factors at which to write output snapshots (cosmological mode
only, i.e.\ when \paramref{cosmo}\texttt{=.true.}).  The array must
contain \paramref{noutput} entries, in ascending order.
\end{quote}

\noindent\textcolor{apidefault}{\textsf{\textbf{Example:}}}
\begin{lstlisting}
noutput = 4
aout    = 0.1, 0.2, 0.5, 1.0
! Outputs at z = 9, 4, 1, 0
\end{lstlisting}

% ------------------------------------------------------------------
\paramheader{tout}{real(dp) array}{---}
\label{param:tout}

\begin{quote}
Output times in code units (non-cosmological mode).  The array must
contain \paramref{noutput} entries, in ascending order.
\end{quote}

% ------------------------------------------------------------------
\paramheader{foutput}{integer}{1000000}
\label{param:foutput}

\begin{quote}
Write an output snapshot every \texttt{foutput} coarse time steps,
regardless of the \paramref{aout}/\paramref{tout} schedule.  Useful
for periodic checkpointing in long runs.  Set to a very large number
to effectively disable.
\end{quote}

% ------------------------------------------------------------------
\paramheader{outformat}{character}{`original'}
\label{param:outformat}

\begin{quote}
Output file format for snapshots.
\begin{description}[style=nextline,leftmargin=2.5cm,font=\ttfamily]
  \item[`original'] Standard RAMSES per-CPU binary format.  Each MPI
    process writes separate files (\texttt{amr\_NNNNN.outNNNNN},
    \texttt{hydro\_NNNNN.outNNNNN}, etc.).
  \item[`hdf5'] Single HDF5 file per snapshot (\texttt{data\_NNNNN.h5}).
    Uses MPI parallel I/O for collective writes.  The HDF5 file stores
    all AMR, hydro, gravity, particle, and sink data in a hierarchical
    group structure.  \textbf{Requires compilation with}
    \texttt{make~HDF5=1}.
\end{description}

\apinote{The standard auxiliary files (\texttt{info\_NNNNN.txt},
\texttt{header\_NNNNN.txt}, \texttt{compilation.txt},
\texttt{makefile.txt}, \texttt{namelist.txt}) are always written
regardless of \texttt{outformat}.}
\end{quote}

% ------------------------------------------------------------------
\paramheader{informat}{character}{`original'}
\label{param:informat}

\begin{quote}
Input (restart) file format.
\begin{description}[style=nextline,leftmargin=2.5cm,font=\ttfamily]
  \item[`original'] Read from standard per-CPU binary files.  The
    number of MPI processes must match the run that produced the
    checkpoint.
  \item[`hdf5'] Read from the single HDF5 file
    (\texttt{data\_NNNNN.h5}).  Currently requires the same number of
    MPI processes as the original run.  \textbf{Requires compilation
    with} \texttt{make~HDF5=1}.
\end{description}

\apinote{\texttt{informat} and \texttt{outformat} can be set
independently, allowing cross-format conversion (e.g.\ restart from
binary and output to HDF5, or vice versa).}
\end{quote}


%======================================================================
\section{\texttt{\&INIT\_PARAMS} -- Initial Conditions}
\label{sec:init_params}
%======================================================================

Specifies the format and location of initial condition files.

% ------------------------------------------------------------------
\paramheader{filetype}{character}{`grafic'}
\label{param:filetype}

\begin{quote}
Initial condition file format.
\begin{description}[style=nextline,leftmargin=2.5cm,font=\ttfamily]
  \item[`grafic'] GRAFIC2 binary format (Bertschinger 2001).  Each
    level's IC directory contains binary files for density
    perturbations, velocities, and (optionally) particle
    displacements.
  \item[`ascii'] Text-based initial conditions (for simple test
    problems).
\end{description}
\end{quote}

% ------------------------------------------------------------------
\paramheader{initfile}{character array}{---}
\label{param:initfile}

\begin{quote}
Paths to IC directories, one per AMR level.
\texttt{initfile(1)} corresponds to \paramref{levelmin},
\texttt{initfile(2)} to $\texttt{levelmin}+1$, and so on.

Each directory must contain the following binary files:
\begin{itemize}[nosep]
  \item \texttt{ic\_deltab} -- baryon density perturbation field
  \item \texttt{ic\_velbx}, \texttt{ic\_velby}, \texttt{ic\_velbz}
        -- baryon velocity fields
  \item \texttt{ic\_velcx}, \texttt{ic\_velcy}, \texttt{ic\_velcz}
        -- dark matter (CDM) velocity fields
  \item \texttt{ic\_poscx}, \texttt{ic\_poscy}, \texttt{ic\_poscz}
        -- dark matter displacement fields (optional, for multi-level
        zoom-in)
  \item \texttt{ic\_tempb} -- baryon temperature perturbation
        (optional)
  \item \texttt{ic\_pvar\_00001}, \ldots\ -- passive scalar fields
        (optional, for zoom-in refinement tagging;
        see \paramref{ivar_refine})
  \item \texttt{ic\_refmap} -- refinement map (optional)
\end{itemize}
\end{quote}

\noindent\textcolor{apidefault}{\textsf{\textbf{Example:}}}
\begin{lstlisting}
initfile = '/data/IC/level_07'
         , '/data/IC/level_08'
         , '/data/IC/level_09'
\end{lstlisting}


%======================================================================
\section{\texttt{\&REFINE\_PARAMS} -- Refinement Criteria}
\label{sec:refine_params}
%======================================================================

Controls which cells are refined in the AMR hierarchy.  These
parameters are \textbf{critical for zoom-in simulations}, where
background regions must remain coarse while the zoom region refines
to high resolution.

% ------------------------------------------------------------------
\paramheader{m\_refine}{real(dp) array}{$-1$}
\label{param:m_refine}

\begin{quote}
Quasi-Lagrangian mass threshold per level.  The $i$-th entry applies
to level $\texttt{levelmin}+i-1$.  A cell is flagged for refinement
when the effective mass indicator $\phi \geq \texttt{m\_refine}(i)$.

Typical value: \textbf{8.0} for all levels (refine when the
equivalent of $\geq 8$ particles occupies a cell).  Provide one
entry for each level from \texttt{levelmin} to \texttt{levelmax}.

Interacts with \paramref{ivar_refine} and
\paramref{mass_cut_refine} to determine which particles contribute
to the density used for the refinement decision.
\end{quote}

\noindent\textcolor{apidefault}{\textsf{\textbf{Example:}}}
\begin{lstlisting}
! 6 levels of refinement (levelmin=7, levelmax=13)
m_refine = 8., 8., 8., 8., 8., 8., 8.
\end{lstlisting}

% ------------------------------------------------------------------
\paramheader{ivar\_refine}{integer}{$-1$}
\label{param:ivar_refine}

\begin{quote}
Variable index controlling the refinement criterion in
\texttt{poisson\_refine}.  This parameter fundamentally changes how
refinement regions are selected:
\begin{description}[style=nextline,leftmargin=1em]
  \item[\texttt{ivar\_refine = 0}:]
    Use \texttt{cpu\_map2} for refinement control.  During
    initialization, \texttt{cpu\_map2} is set by
    \texttt{init\_refmap} from \texttt{ic\_refmap} (if present);
    during evolution, it is updated by \texttt{rho\_fine} based on
    the local density field.  This is the standard quasi-Lagrangian
    approach.

    \apiwarning{In zoom-in simulations, this can cause uncontrolled
    AMR expansion into background regions if \texttt{cpu\_map2} is
    not properly restricted by \paramref{mass_cut_refine}.}

  \item[\texttt{ivar\_refine > 0} (e.g.\ 11):]
    During initialization, use passive scalar criterion:
    \texttt{uold(cell, ivar\_refine) / uold(cell, 1)} $>$
    \paramref{var_cut_refine}.

    \textbf{Recommended for zoom-in:} set
    \texttt{ivar\_refine=NVAR} (the last hydro variable), and create
    \texttt{ic\_pvar\_NNNNN} files with value 1.0 inside the zoom
    region and 0.0 in the background.  After initialization,
    \texttt{cpu\_map2} (set by \texttt{rho\_fine} with
    \paramref{mass_cut_refine} filtering) takes over.

  \item[\texttt{ivar\_refine < 0} (default):]
    Pure density-based refinement at both initialization and runtime.
    A cell is refined when
    \texttt{uold(cell, 1)} $\geq$ \texttt{m\_refine}
    $\times\; m_{\text{sph}} / V_{\text{cell}}$.
\end{description}
\end{quote}

% ------------------------------------------------------------------
\paramheader{var\_cut\_refine}{real(dp)}{$-1$}
\label{param:var_cut_refine}

\begin{quote}
Threshold for passive-scalar-based refinement when
\paramref{ivar_refine} $> 0$.  A cell is refined only if
\[
  \frac{\texttt{uold}(\text{cell},\;\texttt{ivar\_refine})}%
       {\texttt{uold}(\text{cell},\;1)}
  > \texttt{var\_cut\_refine}
\]
Typical value: \textbf{0.01} for zoom geometry tagging (the passive
scalar is 1.0 inside the zoom region, 0.0 outside).
\end{quote}

% ------------------------------------------------------------------
\paramheader{mass\_cut\_refine}{real(dp)}{$-1$}
\label{param:mass_cut_refine}

\begin{quote}
Particle mass threshold for quasi-Lagrangian refinement.  In
\texttt{rho\_fine}, dark matter particles with mass
$\geq \texttt{mass\_cut\_refine}$ are \emph{excluded} from the
density computation that drives cell refinement.  This prevents
heavy (coarse-level) background particles from triggering spurious
refinement.

Set this to the DM particle mass at the finest IC level.  Reference
values for a $100\,h^{-1}\,$Mpc box
($\Omega_m = 0.3$, $h = 0.68$):

\begin{center}
\small
\begin{tabular}{@{}ll@{}}
\toprule
\textbf{IC finest level} & \textbf{mass\_cut\_refine} \\
\midrule
8  & \texttt{1.19209e-07} \\
9  & \texttt{1.49012e-08} \\
10 & \texttt{1.86265e-09} \\
11 & \texttt{2.32831e-10} \\
12 & \texttt{2.91038e-11} \\
13 & \texttt{3.63798e-12} \\
\bottomrule
\end{tabular}
\end{center}

\apinote{This parameter interacts with \paramref{ivar_refine} and
\paramref{m_refine}.  All three should be set consistently for
zoom-in simulations.}
\end{quote}

% ------------------------------------------------------------------
\paramheader{interpol\_var}{integer}{0}
\label{param:interpol_var}

\begin{quote}
Interpolation variable type used when prolongating (interpolating)
data from coarse to fine grids.
\begin{description}[style=nextline,leftmargin=1.5cm,font=\ttfamily]
  \item[0] Conservative variables ($\rho$, $\rho v$, $E$).
  \item[1] Primitive variables ($\rho$, $v$, $P$).
        \textbf{Recommended} for cosmological simulations to avoid
        interpolation artefacts in low-density regions.
\end{description}
\end{quote}

% ------------------------------------------------------------------
\paramheader{interpol\_type}{integer}{1}
\label{param:interpol_type}

\begin{quote}
Interpolation slope limiter for prolongation.
\begin{description}[style=nextline,leftmargin=1.5cm,font=\ttfamily]
  \item[0] MinMod limiter -- more diffusive, more robust.
  \item[1] MonCen (monotonised central) limiter -- less diffusive.
        \textbf{Recommended}.
\end{description}
\end{quote}

% ------------------------------------------------------------------
\paramheader{sink\_refine}{logical}{.false.}
\label{param:sink_refine}

\begin{quote}
Force maximum refinement around sink particles.  When \texttt{.true.},
a contribution equal to \paramref{m_refine} is added to the
refinement indicator $\phi$ for every cell containing a sink particle,
ensuring refinement up to \paramref{levelmax}.
\end{quote}

% ------------------------------------------------------------------
\paramheader{jeans\_ncells}{real(dp)}{$-1$}
\label{param:jeans_ncells}

\begin{quote}
Jeans refinement criterion.  If $> 0$, cells are refined to resolve
the local Jeans length by at least this many cells:
\[
  \Delta x < \frac{\lambda_J}{\texttt{jeans\_ncells}}
\]
Enabling this also activates a polytropic equation-of-state floor to
prevent artificial fragmentation (Truelove criterion).  Typical
value: \textbf{4} (minimum of 4 cells per Jeans length).
\end{quote}


%======================================================================
\section{\texttt{\&HYDRO\_PARAMS} -- Hydrodynamics Solver}
\label{sec:hydro_params}
%======================================================================

Controls the gas dynamics solver configuration.

% ------------------------------------------------------------------
\paramheader{gamma}{real(dp)}{$5/3$}
\label{param:gamma}

\begin{quote}
Adiabatic index $\gamma$ of the ideal gas equation of state,
$P = (\gamma - 1) \rho e$.
Standard value: $5/3$ for a monatomic ideal gas.  Use $7/5$ for
diatomic gas or $4/3$ for radiation-dominated flow.
\end{quote}

% ------------------------------------------------------------------
\paramheader{courant\_factor}{real(dp)}{0.8}
\label{param:courant_factor}

\begin{quote}
Courant--Friedrichs--Lewy (CFL) number for time step control.  The
time step at each level is
$\Delta t = \texttt{courant\_factor} \times \Delta x / v_{\max}$.
Typical: \textbf{0.8}.  Lower values increase stability at the cost
of more time steps.
\end{quote}

% ------------------------------------------------------------------
\paramheader{scheme}{character}{`muscl'}
\label{param:scheme}

\begin{quote}
Hydrodynamics integration scheme.
\begin{description}[style=nextline,leftmargin=2.5cm,font=\ttfamily]
  \item[`muscl'] MUSCL--Hancock (Monotonic Upstream-centred Scheme
    for Conservation Laws), second-order in space and time.  This is
    the only production scheme in RAMSES.
\end{description}
\end{quote}

% ------------------------------------------------------------------
\paramheader{slope\_type}{integer}{1}
\label{param:slope_type}

\begin{quote}
Slope limiter for MUSCL piecewise-linear reconstruction.
\begin{description}[style=nextline,leftmargin=1.5cm,font=\ttfamily]
  \item[1] MinMod -- most robust, more diffusive.
  \item[2] MonCen -- monotonised central, less diffusive.
        \textbf{Recommended} for production runs.
  \item[3] Unlimited -- no limiting (unstable; testing only).
\end{description}
\end{quote}

% ------------------------------------------------------------------
\paramheader{riemann}{character}{`llf'}
\label{param:riemann}

\begin{quote}
Approximate Riemann solver for inter-cell flux computation.
\begin{description}[style=nextline,leftmargin=2.5cm,font=\ttfamily]
  \item[`llf'] Local Lax--Friedrichs (Rusanov).  Most diffusive but
    unconditionally stable.  Good default.
  \item[`hll'] Harten--Lax--van Leer.  Two-wave solver.
  \item[`hllc'] HLL with Contact restoration.  Three-wave solver,
    most accurate for contact discontinuities.
    \textbf{Recommended for cosmological simulations}.
  \item[`exact'] Exact Riemann solver (expensive; primarily for
    validation).
\end{description}
\end{quote}

% ------------------------------------------------------------------
\paramheader{pressure\_fix}{logical}{.false.}
\label{param:pressure_fix}

\begin{quote}
Enable pressure floor to prevent negative pressures in strong shocks
or highly supersonic flows.  When the internal energy becomes negative,
RAMSES falls back to a pressure estimate from the total energy.

\textbf{Recommended:} \texttt{.true.}\ for cosmological simulations.
See also \paramref{beta_fix}.
\end{quote}

% ------------------------------------------------------------------
\paramheader{beta\_fix}{real(dp)}{0.0}
\label{param:beta_fix}

\begin{quote}
Pressure fix parameter.  Controls the magnitude of the pressure floor:
$P_{\text{floor}} = \texttt{beta\_fix} \times \rho v^2 / 2$.
Typical value: \textbf{0.5} when
\paramref{pressure_fix}\texttt{=.true.}
\end{quote}

% ------------------------------------------------------------------
\paramheader{isothermal}{logical}{.false.}
\label{param:isothermal}

\begin{quote}
Isothermal mode.  When \texttt{.true.}, the energy equation is not
solved and the gas temperature remains constant.  Reduces the number
of hydro variables by one.
\end{quote}


%======================================================================
\section{\texttt{\&POISSON\_PARAMS} -- Gravity Solver}
\label{sec:poisson_params}
%======================================================================

Controls the multigrid Poisson solver for self-gravity.

% ------------------------------------------------------------------
\paramheader{epsilon}{real(dp)}{$10^{-4}$}
\label{param:epsilon}

\begin{quote}
Multigrid convergence criterion.  The V-cycle iteration at each level
stops when the residual norm satisfies
$\|r\| / \|r_0\| < \texttt{epsilon}$.
Typical value for cosmological runs: $10^{-5}$ to $10^{-4}$.
Tighter values improve force accuracy but increase iteration count.
\end{quote}

% ------------------------------------------------------------------
\paramheader{gravity\_type}{integer}{0}
\label{param:gravity_type}

\begin{quote}
Gravity model selection.
\begin{description}[style=nextline,leftmargin=1.5cm,font=\ttfamily]
  \item[0] Self-gravity (solve Poisson equation on the AMR grid).
  \item[>0] Analytical gravitational potential (e.g.\ for test
    problems with known solutions).  The integer value selects the
    specific analytical profile.
\end{description}
\end{quote}

% ------------------------------------------------------------------
\paramheader{cg\_levelmin}{integer}{999}
\label{param:cg_levelmin}

\begin{quote}
Minimum level at which the conjugate gradient (CG) fallback solver
activates.  When the multigrid solver stalls at high AMR levels, CG
provides guaranteed convergence.  Set to \paramref{levelmax} for
best convergence behaviour.

Typical: \texttt{cg\_levelmin = levelmax}.  The default (999) means
CG is effectively disabled unless \texttt{levelmax} is absurdly
large.
\end{quote}

% ------------------------------------------------------------------
\paramheader{cic\_levelmax}{integer}{0}
\label{param:cic_levelmax}

\begin{quote}
Maximum level for cloud-in-cell (CIC) particle mass deposition.
\begin{itemize}[nosep]
  \item \texttt{0} -- deposit particles at all levels (standard).
  \item $N > 0$ -- deposit particles only up to level $N$; finer
    levels inherit the coarse density by prolongation.
\end{itemize}
Rarely modified.
\end{quote}


%======================================================================
\section{\texttt{\&PHYSICS\_PARAMS} -- Sub-grid Physics}
\label{sec:physics_params}
%======================================================================

Controls cooling, star formation, stellar/AGN feedback, and
cosmological parameters.  This block is optional; omit it entirely
for adiabatic (non-radiative) simulations.

%----------------------------------------------------------------------
\subsection{Cooling and UV Background}
%----------------------------------------------------------------------

\paramheader{cooling}{logical}{.false.}
\label{param:cooling}

\begin{quote}
Enable radiative cooling with a metal-dependent cooling function.
When \texttt{.true.}, RAMSES integrates the cooling/heating rate
at each time step using tabulated cooling curves.  Requires
\paramref{hydro}\texttt{=.true.}
\end{quote}

% ------------------------------------------------------------------
\paramheader{haardt\_madau}{logical}{.false.}
\label{param:haardt_madau}

\begin{quote}
Enable the Haardt \& Madau (2012) ultraviolet background model for
cosmic reionization.  Provides a redshift-dependent photo-heating
and photo-ionization rate.  Used together with \paramref{cooling}.
\end{quote}

% ------------------------------------------------------------------
\paramheader{z\_reion}{real(dp)}{8.5}
\label{param:z_reion}

\begin{quote}
Reionization redshift.  Hydrogen reionization heating is applied
instantaneously at this redshift.  Typical range: 6--10, depending
on the reionization model.
\end{quote}

% ------------------------------------------------------------------
\paramheader{z\_ave}{real(dp)}{0.0}
\label{param:z_ave}

\begin{quote}
Initial mean metallicity of the gas in solar units ($Z_\odot$).
Applied uniformly at initialization.  Use \texttt{0.0} for
primordial composition.
\end{quote}

% ------------------------------------------------------------------
\paramheader{delayed\_cooling}{logical}{.false.}
\label{param:delayed_cooling}

\begin{quote}
Delay radiative cooling in supernova-heated gas to prevent
overcooling.  When a cell receives SN energy, cooling is suppressed
for a duration related to the Sedov--Taylor phase.  Improves the
effectiveness of stellar feedback in regulating star formation.
\end{quote}

% ------------------------------------------------------------------
\paramheader{tol}{real(dp)}{$10^{-3}$}
\label{param:tol}

\begin{quote}
Tolerance for the implicit cooling solver.  The Newton--Raphson
iteration converges when the relative temperature change
$|\Delta T / T| < \texttt{tol}$.
\end{quote}

%----------------------------------------------------------------------
\subsection{Star Formation}
%----------------------------------------------------------------------

\paramheader{n\_star}{real(dp)}{0.1}
\label{param:n_star}

\begin{quote}
Star formation hydrogen number density threshold in
$\text{H}\;\text{cm}^{-3}$.  Only gas denser than this value is
eligible for star formation.  Typical range: 0.1--10.
\end{quote}

% ------------------------------------------------------------------
\paramheader{eps\_star}{real(dp)}{0.0}
\label{param:eps_star}

\begin{quote}
Star formation efficiency per free-fall time $\epsilon_{\text{ff}}$.
The star formation rate density is
$\dot{\rho}_\star = \epsilon_{\text{ff}} \, \rho_{\text{gas}} /
t_{\text{ff}}$.
Typical value: \textbf{0.01--0.02} (1--2\%\ per free-fall time).
Set to \texttt{0.0} to disable star formation entirely.
\end{quote}

% ------------------------------------------------------------------
\paramheader{del\_star}{real(dp)}{200}
\label{param:del_star}

\begin{quote}
Overdensity threshold for star formation (in units of the cosmic
mean density).  Gas must exceed $\delta > \texttt{del\_star}$ in
addition to the density threshold \paramref{n_star}.
\end{quote}

% ------------------------------------------------------------------
\paramheader{m\_star}{real(dp)}{$-1$}
\label{param:m_star}

\begin{quote}
Minimum stellar particle mass in code units.  When a star-forming
cell would produce a particle below this mass, the event is
stochastically deferred to the next time step.
\begin{itemize}[nosep]
  \item $< 0$: use the cell gas mass (no minimum).
  \item $> 0$: explicit minimum mass.
\end{itemize}
\end{quote}

% ------------------------------------------------------------------
\paramheader{T2\_star}{real(dp)}{0}
\label{param:T2_star}

\begin{quote}
ISM polytropic equation-of-state temperature floor in Kelvin.  Gas
above the star-formation density threshold \paramref{n_star} follows
a polytropic relation:
\[
  T = T_{2,\star} \left(\frac{n}{n_\star}\right)^{\gamma_\star - 1}
\]
where $\gamma_\star$ is \paramref{g_star}.  This prevents artificial
fragmentation below the resolution limit (Jeans mass floor).
\end{quote}

% ------------------------------------------------------------------
\paramheader{g\_star}{real(dp)}{1.6}
\label{param:g_star}

\begin{quote}
Polytropic index $\gamma_\star$ for the ISM equation of state (see
\paramref{T2_star}).  Typical value: \textbf{1.6} (stiff polytrope)
or \textbf{5/3} (adiabatic floor).
\end{quote}

%----------------------------------------------------------------------
\subsection{Stellar Feedback}
%----------------------------------------------------------------------

\paramheader{f\_w}{real(dp)}{0}
\label{param:f_w}

\begin{quote}
Mass loading factor for supernova-driven winds.  The wind mass flux
is $\dot{M}_w = \texttt{f\_w} \times \dot{M}_\star$.  Set to
\texttt{0} to disable winds.  Typical range: 1--5.
\end{quote}

% ------------------------------------------------------------------
\paramheader{f\_ek}{real(dp)}{1.0}
\label{param:f_ek}

\begin{quote}
Kinetic energy fraction of supernova feedback.  Controls the
partition between kinetic (\texttt{f\_ek}) and thermal
($1 - \texttt{f\_ek}$) energy injection.  \texttt{f\_ek=1} is
purely kinetic feedback; \texttt{f\_ek=0} is purely thermal.
\end{quote}

% ------------------------------------------------------------------
\paramheader{eps\_sn1}{real(dp)}{0}
\label{param:eps_sn1}

\begin{quote}
Type Ia supernova energy per event in units of $10^{51}\,$erg.
Set to \texttt{0} to disable Type Ia SN feedback.
\end{quote}

% ------------------------------------------------------------------
\paramheader{eps\_sn2}{real(dp)}{0}
\label{param:eps_sn2}

\begin{quote}
Type II supernova energy per event in units of $10^{51}\,$erg.
Set to \texttt{0} to disable Type II SN feedback.
\end{quote}

% ------------------------------------------------------------------
\paramheader{yieldtablefilename}{character}{---}
\label{param:yieldtablefilename}

\begin{quote}
Path to the chemical yield table file for metal enrichment
calculations.  Required when metal-dependent cooling or chemical
evolution tracking is enabled.
\end{quote}

%----------------------------------------------------------------------
\subsection{Cosmological Parameters}
%----------------------------------------------------------------------

\paramheader{omega\_b}{real(dp)}{---}
\label{param:omega_b}

\begin{quote}
Baryon density parameter $\Omega_b$.  Overrides the value read from
the IC file header.  Must be consistent with the initial conditions
and other cosmological parameters ($\Omega_m$, $H_0$, etc.\ are
read from the GRAFIC2 header).
\end{quote}

%----------------------------------------------------------------------
\subsection{AGN and Sink Particle Parameters}
%----------------------------------------------------------------------

\paramheader{Mseed}{real(dp)}{---}
\label{param:Mseed}

\begin{quote}
Seed black hole mass in solar masses ($M_\odot$).  When a sink
particle forms, it is initialised with this mass.  Typical range:
$10^4$--$10^6\,M_\odot$ for cosmological simulations.
\end{quote}

% ------------------------------------------------------------------
\paramheader{sink\_AGN}{logical}{.false.}
\label{param:sink_AGN}

\begin{quote}
Enable AGN feedback from sink particles.  When \texttt{.true.}, sink
particles inject thermal and/or kinetic energy into their surroundings
based on their accretion rate.  Requires
\paramref{sink}\texttt{=.true.}
\end{quote}

% ------------------------------------------------------------------
\paramheader{bondi}{logical}{.false.}
\label{param:bondi}

\begin{quote}
Enable Bondi--Hoyle--Lyttleton accretion for sink particles.  The
accretion rate is computed from the local gas density, sound speed,
and relative velocity:
\[
  \dot{M} = \frac{4\pi G^2 M_{\text{BH}}^2 \rho}%
             {(c_s^2 + v_{\text{rel}}^2)^{3/2}}
\]
Can be boosted by \paramref{boost_acc}.
\end{quote}

% ------------------------------------------------------------------
\paramheader{drag}{logical}{.false.}
\label{param:drag}

\begin{quote}
Enable dynamical friction on sink particles.  Applies a drag force
opposing the sink's motion relative to the background gas.  Strength
can be amplified by \paramref{boost_drag}.
\end{quote}

% ------------------------------------------------------------------
\paramheader{rAGN}{real(dp)}{---}
\label{param:rAGN}

\begin{quote}
AGN feedback energy injection radius in units of the cell size at
\paramref{levelmax}.  Feedback energy is distributed over a sphere
of this radius centred on the sink particle.
\end{quote}

% ------------------------------------------------------------------
\paramheader{X\_floor}{real(dp)}{---}
\label{param:X_floor}

\begin{quote}
Hydrogen mass fraction floor.  Prevents the hydrogen fraction from
dropping below this value due to numerical artefacts.  Typical:
\texttt{0.76}.
\end{quote}

% ------------------------------------------------------------------
\paramheader{eAGN\_K}{real(dp)}{---}
\label{param:eAGN_K}

\begin{quote}
AGN kinetic feedback efficiency $\epsilon_K$.  Fraction of the
accreted rest-mass energy deposited as kinetic energy:
$\dot{E}_K = \epsilon_K \, \dot{M} c^2$.
\end{quote}

% ------------------------------------------------------------------
\paramheader{eAGN\_T}{real(dp)}{---}
\label{param:eAGN_T}

\begin{quote}
AGN thermal feedback efficiency $\epsilon_T$.  Fraction of the
accreted rest-mass energy deposited as thermal energy:
$\dot{E}_T = \epsilon_T \, \dot{M} c^2$.
\end{quote}

% ------------------------------------------------------------------
\paramheader{TAGN}{real(dp)}{---}
\label{param:TAGN}

\begin{quote}
AGN heating temperature in Kelvin.  The AGN thermal energy is
deposited by raising gas temperature toward this value within the
feedback region \paramref{rAGN}.
\end{quote}

% ------------------------------------------------------------------
\paramheader{r\_gal}{real(dp)}{---}
\label{param:r_gal}

\begin{quote}
Galaxy definition radius for AGN feedback, in code units.  Used to
compute the local galaxy properties (stellar mass, gas mass) around
a sink particle for AGN mode switching.
\end{quote}

% ------------------------------------------------------------------
\paramheader{T2maxAGN}{real(dp)}{---}
\label{param:T2maxAGN}

\begin{quote}
Maximum AGN heating temperature in Kelvin.  Caps the temperature
increase from a single AGN feedback event to prevent unphysically
hot gas.
\end{quote}

% ------------------------------------------------------------------
\paramheader{boost\_acc}{real(dp)}{---}
\label{param:boost_acc}

\begin{quote}
Bondi accretion boost factor.  Multiplies the Bondi--Hoyle accretion
rate by this factor to compensate for unresolved gas structure near
the black hole.  Typical range: 1--100.  Requires
\paramref{bondi}\texttt{=.true.}
\end{quote}

% ------------------------------------------------------------------
\paramheader{boost\_drag}{real(dp)}{---}
\label{param:boost_drag}

\begin{quote}
Dynamical friction drag boost factor.  Multiplies the drag force by
this factor.  Requires \paramref{drag}\texttt{=.true.}
\end{quote}

% ------------------------------------------------------------------
\paramheader{vrel\_merge}{logical}{---}
\label{param:vrel_merge}

\begin{quote}
Use relative velocity criterion for sink particle merging.  When
\texttt{.true.}, two sinks merge only if their relative velocity is
below the local escape velocity, in addition to the spatial proximity
criterion \paramref{rmerge}.
\end{quote}

% ------------------------------------------------------------------
\paramheader{rmerge}{real(dp)}{---}
\label{param:rmerge}

\begin{quote}
Sink merging radius in units of the cell size at
\paramref{levelmax}.  Two sink particles closer than this distance
are candidates for merging (subject to additional criteria if
\paramref{vrel_merge}\texttt{=.true.}).
\end{quote}

% ------------------------------------------------------------------
\paramheader{spin\_bh}{logical}{---}
\label{param:spin_bh}

\begin{quote}
Track black hole spin evolution.  When \texttt{.true.}, the code
evolves the dimensionless spin parameter $a_\star$ of each sink
particle based on the angular momentum of accreted gas.
\end{quote}

% ------------------------------------------------------------------
\paramheader{mad\_jet}{logical}{---}
\label{param:mad_jet}

\begin{quote}
Enable the magnetically arrested disk (MAD) jet model.  When
\texttt{.true.}, AGN kinetic feedback is launched as a collimated
bipolar jet aligned with the black hole spin axis.  Requires
\paramref{spin_bh}\texttt{=.true.}
\end{quote}


%======================================================================
\section{\texttt{\&LIGHTCONE\_PARAMS} -- Lightcone Output}
\label{sec:lightcone_params}
%======================================================================

\begin{quote}
Parameters for lightcone output mode (activated when
\paramref{lightcone}\texttt{=.true.}).  In this mode, particles
and/or cells that cross the observer's past lightcone during each
time step are written to special output files, enabling the
construction of mock galaxy surveys and weak-lensing maps without
storing full snapshots.

Configuration parameters include the observer position, opening
angle, and selection criteria.  Consult the RAMSES lightcone
documentation for the full parameter list, which varies by
application.
\end{quote}


%======================================================================
\section{\texttt{\&SPHERICAL\_REGION\_PARAMS}}
\label{sec:spherical_region_params}
%======================================================================

\paramheader{spherical\_region}{logical}{.false.}
\label{param:spherical_region}

\begin{quote}
Enable a spherical refinement region.  When \texttt{.true.}, AMR
refinement is restricted to a spherical sub-volume of the simulation
box.  This is useful for re-simulations of specific halos where a
cubic zoom region is not optimal.  Additional parameters define the
centre and radius of the sphere.
\end{quote}


%======================================================================
\section{Complete Example: Cosmological Zoom-In}
\label{sec:example}
%======================================================================

The following namelist illustrates a production cosmological zoom-in
simulation with dark matter particles, baryonic gas, self-gravity,
cooling, star formation, and AGN feedback.

\begin{lstlisting}[language=Fortran,
  caption={Cosmological zoom-in namelist},
  label={lst:zoomin}]
&RUN_PARAMS
cosmo    = .true.
pic      = .true.
poisson  = .true.
hydro    = .true.
sink     = .true.
nrestart = 0
nremap   = 5
nsubcycle = 1, 1, 1, 2, 2, 2, 2
nstepmax = 10000000
ordering = 'ksection'
memory_balance = .true.
/

&AMR_PARAMS
levelmin  = 7
levelmax  = 18
nexpand   = 1, 1, 1, 1, 1, 1, 1, 1, 1, 1, 1, 1
ngridtot  = 400000000
nparttot  = 600000000
/

&OUTPUT_PARAMS
noutput = 10
aout = 0.05, 0.1, 0.15, 0.2, 0.3, 0.4, 0.5, 0.7, 0.85, 1.0
foutput = 500
/

&INIT_PARAMS
filetype = 'grafic'
initfile = '/data/IC/level_07'
         , '/data/IC/level_08'
         , '/data/IC/level_09'
         , '/data/IC/level_10'
         , '/data/IC/level_11'
         , '/data/IC/level_12'
         , '/data/IC/level_13'
/

&REFINE_PARAMS
m_refine = 8., 8., 8., 8., 8., 8., 8., 8., 8., 8., 8., 8.
ivar_refine     = 11
var_cut_refine  = 0.01
mass_cut_refine = 3.63798e-12
interpol_var    = 1
interpol_type   = 1
/

&HYDRO_PARAMS
gamma         = 1.6666667
courant_factor = 0.8
scheme        = 'muscl'
slope_type    = 2
riemann       = 'hllc'
pressure_fix  = .true.
beta_fix      = 0.5
/

&POISSON_PARAMS
epsilon      = 1.0e-5
gravity_type = 0
cg_levelmin  = 18
/

&PHYSICS_PARAMS
cooling        = .true.
haardt_madau   = .true.
z_reion        = 8.5
n_star         = 0.1
eps_star       = 0.02
T2_star        = 1.0e4
g_star         = 1.6
del_star       = 200.0
f_ek           = 1.0
sink_AGN       = .true.
bondi          = .true.
Mseed          = 1.0e5
/
\end{lstlisting}


%======================================================================
\section{Parameter Cross-Reference Index}
\label{sec:crossref}
%======================================================================

Table~\ref{tab:crossref} lists parameters that commonly interact and
should be set consistently.

\begin{longtable}{@{}p{4cm}p{4.5cm}p{6.5cm}@{}}
\caption{Cross-reference of interacting parameters.}
\label{tab:crossref}\\
\toprule
\textbf{Parameter} & \textbf{Related Parameters} & \textbf{Notes} \\
\midrule
\endfirsthead
\toprule
\textbf{Parameter} & \textbf{Related Parameters} & \textbf{Notes} \\
\midrule
\endhead
\bottomrule
\endfoot
\texttt{cosmo}
  & \texttt{aout}, \texttt{omega\_b}
  & Cosmological mode requires scale-factor outputs \\
\addlinespace
\texttt{pic}
  & \texttt{poisson}, \texttt{nparttot}
  & Particles need gravity and memory allocation \\
\addlinespace
\texttt{ordering}
  & \texttt{memory\_balance}
  & Memory balancing requires \texttt{ksection} \\
\addlinespace
\texttt{levelmax}
  & \texttt{m\_refine}, \texttt{cg\_levelmin}
  & Set \texttt{cg\_levelmin = levelmax} \\
\addlinespace
\texttt{ivar\_refine}
  & \texttt{var\_cut\_refine}, \texttt{mass\_cut\_refine}, \texttt{m\_refine}
  & All must be consistent for zoom-in \\
\addlinespace
\texttt{mass\_cut\_refine}
  & \texttt{ivar\_refine}
  & Set to finest-level DM particle mass \\
\addlinespace
\texttt{pressure\_fix}
  & \texttt{beta\_fix}
  & \texttt{beta\_fix} only effective when fix is on \\
\addlinespace
\texttt{T2\_star}
  & \texttt{g\_star}, \texttt{n\_star}
  & Polytropic EOS parameters \\
\addlinespace
\texttt{sink}
  & \texttt{sink\_AGN}, \texttt{bondi}, \texttt{Mseed}
  & AGN feedback requires sink particles \\
\addlinespace
\texttt{sink\_AGN}
  & \texttt{eAGN\_K}, \texttt{eAGN\_T}, \texttt{TAGN}, \texttt{rAGN}
  & AGN feedback parameters \\
\addlinespace
\texttt{bondi}
  & \texttt{boost\_acc}
  & Boost factor for unresolved accretion \\
\addlinespace
\texttt{drag}
  & \texttt{boost\_drag}
  & Drag boost factor \\
\addlinespace
\texttt{spin\_bh}
  & \texttt{mad\_jet}
  & MAD jet requires spin tracking \\
\addlinespace
\texttt{cooling}
  & \texttt{haardt\_madau}, \texttt{z\_reion}
  & UV background for reionization heating \\
\addlinespace
\texttt{ngridtot}
  & \texttt{nparttot}
  & Both determine per-process memory usage \\
\end{longtable}


%======================================================================
\vfill
\begin{center}
\small\sffamily\textcolor{apitype}{%
Generated for cuRAMSES-kjhan --- \today}
\end{center}

\end{document}
